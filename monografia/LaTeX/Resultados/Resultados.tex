\chapter{Resultados}

\section{Testes Unitários}

\subsection{Leituras Espúrias nos Sonares}
% citar leituras espúrias encontradas no teste de din Vs est
Ao serem feitos testes unitários no \textit{software} de controle dos sensores ultrassônicos, notou-se a existência de leituras espúrias, em 
condições em que não havia obstáculos dentro da região visível do dispositivo.
A fim de documentar o problema, posicionou-se o veículo num ambiente em que não havia qualquer obstáculo num raio de 5 metros e foram feitas medições 
com intervalos de 30ms, 45ms e 60ms; no casa deste teste, não faz diferença se as leituras são dinâmicas ou estáticas pois em todas leituras ao menos 
um dos sonares não conseguia detectar obstáculos, fazendo com que os intervalos de medida estático e dinâmico fossem iguais sempre.

Os resultados obtidos no teste mostraram que as leituras espúrias são esparsas e, em sua maioria, apontam valores menores do que 10cm.
A solução concebida para remediar este problema foi basear o comportamento a ser adotado pelo veículo na média aritmética das últimas 5 leituras, 
excluindo os valores extremos. 
Desta forma, reduz-se problemas de leituras errôneas em que o sensor falha em detectar o obstáculo dentro do alcance tanto quanto quando detecta-se 
empecilhos onde não há.
Com este novo comportamento, houve redução de mais de 40\% do número de medidas espúrias, conforme ilustra a Tabela \ref{vazio}.
% Please add the following required packages to your document preamble:
% \usepackage{multirow}
\begin{table}[H]
\centering
\caption{Teste sem Obstáculos ao Alcance}
\label{vazio}
\begin{tabular}{|c|c|c|ccccc|}\hline
Dados & Contagem & Intervalo & Sensor 0 & Sensor 1 & Sensor 2 & Sensor 3 & Sensor 4 \\ \hline
\multirow{6}{*}{crus} & \multirow{3}{*}{$\neq 400$} & 30ms & 0\% & 1,22\% & 0\% & 1,48\% & 0\% \\
 &  & 45ms & 0,67\% & 0,92\% & 0\% & 1,58\% & 0\% \\
 &  & 60ms & 3,17\% & 0,58\% & 0\% & 2,67\% & 0\% \\ \cline{2-8} 
 & \multirow{3}{*}{\textless \hspace{0,5mm} 10} & 30ms & 0\% & 0\% & 0\% & 0,78\% & 0\% \\
 &  & 45ms & 0,67\% & 0\% & 0\% & 0,83\% & 0\% \\
 &  & 60ms & 3,17\% & 0\% & 0\% & 1,08\% & 0\% \\ \hline
\multirow{6}{*}{processados} & \multirow{3}{*}{$\neq 400$} & 30ms & 0\% & 0,70\% & 0\% & 0,35\% & 0\% \\
 &  & 45ms & 0\% & 0,25\% & 0\% & 0,33\% & 0\% \\
 &  & 60ms & 0\% & 0,08\% & 0\% & 1,09\% & 0\% \\ \cline{2-8} 
 & \multirow{3}{*}{\textless \hspace{0,5mm} 10} & 30ms & 0\% & 0\% & 0\% & 0\% & 0\% \\
 &  & 45ms & 0\% & 0\% & 0\% & 0\% & 0\% \\
 &  & 60ms & 0\% & 0\% & 0\% & 0\% & 0\% \\ \hline
\end{tabular}
\end{table}

Uma outra solução plausível para remediar as medidas espúrias que apontam valores muito próximos ao veículo seria aumentar o \textit{blanking time} 
dos sonares, conforme \citeonline{siegwart}.

\subsection{Comparação entre Intervalos de Medição Dinâmico e Estático}
A fim de verificar a viabilidade de empregar-se intervalos dinâmicos na percepção do veículo, este foi disposto de frente a uma quina, buscando 
manter o sensor frontal o mais alinhado ao plano bissetor da junção das duas paredes, perpendiculares entre si, de forma que os sensores das 
extremidades estivessem aproximadamente equidistantes da respectiva parede mais próxima. 
Para isso, foram marcados com uma trena pontos no chão que distavam de ambas paredes 30cm, 50cm, 100cm, 150cm e 200cm; para valores 
maiores do que 200cm não era possível sensibilizar todos sonares ao mesmo tempo, de modo que ao menos um deles não conseguia detectar obstáculos, o 
que não faria sentido para o teste em questão.
Em cada um destes pontos foram feitas medições estáticas e dinâmicas com intervalos de 30ms, 45ms e 60ms. Numa situação ideal, em que todos sonares 
conseguissem sempre encontrar os obstáculos, não faria diferença para as medidas dinâmicas qual o intervalo estipulado, pois a medição seria 
encerrada assim que todos sonares houvessem sido sensibilizados; no entanto, em algumas leituras houve falhas na detecção, apontando erroneamente a 
inexistência de obstáculo.
Em cada um dos testes foram armazenadas 150 leituras consecutivas em um \textit{buffer} local que, ao ser preenchido, cessava o acionamento dos 
sonares e imprimia os dados via interface serial antes de começar a próxima medição.
Nos dados obtidos em cada uma dessas medições foram feitos cálculos de média e desvio padrão sujos e limpos\footnote{Medidas limpas são aquelas cuja 
métrica foi calculada descartando-se as leituras em que o respectivo sonar falhou em encontrar o obstáculo. No caso do teste de 30cm, foram 
excluídas medidas abaixo de 10cm também.}, contagem dos resultados, e expresso em termos percentuais nas tabelas a seguir, que desviassem em mais do 
que 15 centímetros em relação à média limpa das leituras, exceto para o caso do teste feito a 30cm das paredes, em que foram contadas medidas que 
desviaram mais do que 5 centímetros da média limpa.

\subsubsection{Teste 30cm}
Neste teste as leituras dinâmicas se mostraram consideravelmente mais imprecisas, chegando a apresentar mais de 10\% de erros em alguns testes; no 
entanto, é preciso levar em consideração o quão mais rápidas são as medições dinâmicas que, no caso deste teste, durariam por volta de 2ms.
Apesar de todas as leituras de todos sonares terem encontrado obstáculo, o desvio padrão das leituras com intervalo dinâmico chegou a mais de 8cm 
para o sensor frontal em decorrência da aparição de medidas espúrias bem abaixo da média, da ordem de grandeza de 10cm. 
Este tipo de erro de leitura constitui um problema sério para a aplicação em pauta neste projeto, pois a detecção de falsos obstáculos tão próximos 
do veículo causaria o desligamento dos motores para, logo em seguida, serem ligados novamente como consequência das leituras corretas; como resultado 
deste erro, a trajetória acaba sendo imprevisível, pois os motores reagiriam diferentemente ao comando de parada, já que podem estar em velocidades 
diferentes entre si. Isto é, caso no momento dessa leitura espúria um dos motores esteja com a rotação mais baixa, pode ser que ele trave a roda mais 
rapidamente do que o outro, fazendo com que o veículo faça uma curva brusca que possa causar uma colisão, por exemplo.
 % Please add the following required packages to your document preamble:
% \usepackage{multirow}
\begin{center}
\begin{longtable}{|c|c|c|ccccc|}
\label{30cm} \\
\caption[30cm]{30cm} \endfirsthead \\
\caption[]{30cm} \\
\hline \multicolumn{3}{|c|}{{Continua na próxima página.}} \\ \hline
\endfoot \hline
\hline \hline
\endlastfoot
\hline
\textbf{Medida} & \textbf{Intervalo} & \textbf{Leitura} & \textbf{Sensor 0} & \textbf{Sensor 1} & \textbf{Sensor 2} & \textbf{Sensor 3} & \textbf{Sensor 4} \\ \hline
\multirow{12}{*}{\begin{tabular}[c]{@{}c@{}}Média\\ {[}cm{]}\end{tabular}} & \multirow{4}{*}{30ms} & Est. & 26,43 & 30,13 & 31,14 & 35,62 & 37,19 \\
 &  & Din. & 26,58 & 30,57 & 30,55 & 35,67 & 37,51 \\
 &  & Est. & 26,37 & 30,12 & 30,99 & 35,66 & 37,21 \\
 &  & Din. & 26,42 & 30,60 & 32,17 & 35,61 & 37,52 \\ \cline{2-8} 
 & \multirow{4}{*}{45ms} & Est. & 26,36 & 30,51 & 32,61 & 35,67 & 37,34 \\
 &  & Din. & 26,39 & 30,86 & 32,31 & 35,76 & 37,73 \\
 &  & Est. & 26,45 & 30,26 & 31,59 & 35,75 & 37,17 \\
 &  & Din. & 26,42 & 30,29 & 29,11 & 35,65 & 37,36 \\ \cline{2-8} 
 & \multirow{4}{*}{60ms} & Est. & 26,41 & 30,12 & 31,28 & 35,67 & 37,21 \\
 &  & Din. & 26,72 & 30,48 & 31,11 & 35,71 & 37,65 \\
 &  & Est. & 26,46 & 30,01 & 30,38 & 35,61 & 37,15 \\
 &  & Din. & 26,52 & 30,35 & 28,47 & 35,50 & 37,33 \\ \hline
\multirow{12}{*}{\begin{tabular}[c]{@{}c@{}}Média\\ Limpa\\ {[}cm{]}\end{tabular}} & \multirow{4}{*}{30ms} & Est. & 26,43 & 30,13 & 31,14 & 35,62 & 37,19 \\
 &  & Din. & 26,58 & 30,57 & 31,16 & 35,67 & 37,51 \\
 &  & Est. & 26,37 & 30,12 & 30,99 & 35,66 & 37,21 \\
 &  & Din. & 26,42 & 30,60 & 32,69 & 35,61 & 37,52 \\ \cline{2-8} 
 & \multirow{4}{*}{45ms} & Est. & 26,36 & 30,51 & 32,61 & 35,67 & 37,34 \\
 &  & Din. & 26,39 & 30,86 & 32,82 & 35,76 & 37,73 \\
 &  & Est. & 26,45 & 30,26 & 31,59 & 35,75 & 37,17 \\
 &  & Din. & 26,42 & 30,29 & 31,91 & 35,65 & 37,36 \\ \cline{2-8} 
 & \multirow{4}{*}{60ms} & Est. & 26,41 & 30,12 & 31,28 & 35,67 & 37,21 \\
 &  & Din. & 26,72 & 30,48 & 31,59 & 35,71 & 37,65 \\
 &  & Est. & 26,46 & 30,01 & 30,38 & 35,61 & 37,15 \\
 &  & Din. & 26,52 & 30,35 & 30,84 & 35,50 & 37,33 \\ \hline \pagebreak
\multirow{12}{*}{\begin{tabular}[c]{@{}c@{}}Desvio\\ Padrão\\ {[}cm{]}\end{tabular}} & \multirow{4}{*}{30ms} & Est. & 0,50 & 0,34 & 1,51 & 0,49 & 0,41 \\
 &  & Din. & 0,50 & 0,50 & 4,19 & 0,55 & 0,53 \\
 &  & Est. & 0,48 & 0,33 & 1,19 & 0,48 & 0,43 \\
 &  & Din. & 0,50 & 0,49 & 4,35 & 0,59 & 0,55 \\ \cline{2-8} 
 & \multirow{4}{*}{45ms} & Est. & 0,48 & 0,50 & 1,68 & 0,47 & 0,52 \\
 &  & Din. & 0,49 & 0,35 & 4,12 & 0,54 & 0,62 \\
 &  & Est. & 0,50 & 0,44 & 1,63 & 0,44 & 0,41 \\
 &  & Din. & 0,50 & 0,46 & 8,18 & 0,48 & 0,63 \\ \cline{2-8} 
 & \multirow{4}{*}{60ms} & Est. & 0,49 & 0,33 & 1,42 & 0,47 & 0,41 \\
 &  & Din. & 0,45 & 0,50 & 4,10 & 0,50 & 0,69 \\
 &  & Est. & 0,50 & 0,12 & 0,70 & 0,49 & 0,38 \\
 &  & Din. & 0,50 & 0,48 & 7,47 & 0,50 & 0,49 \\ \hline
\multirow{12}{*}{\textless avg - 15} & \multirow{4}{*}{30ms} & Est. & 0,00\% & 0,00\% & 0,00\% & 0,00\% & 0,00\% \\
 &  & Din. & 0,00\% & 0,00\% & 2,67\% & 0,00\% & 0,00\% \\
 &  & Est. & 0,00\% & 0,00\% & 0,00\% & 0,00\% & 0,00\% \\
 &  & Din. & 0,00\% & 0,00\% & 2,00\% & 0,00\% & 0,00\% \\ \cline{2-8} 
 & \multirow{4}{*}{45ms} & Est. & 0,00\% & 0,00\% & 0,00\% & 0,00\% & 0,00\% \\
 &  & Din. & 0,00\% & 0,00\% & 2,00\% & 0,00\% & 0,00\% \\
 &  & Est. & 0,00\% & 0,00\% & 0,00\% & 0,00\% & 0,00\% \\
 &  & Din. & 0,00\% & 0,00\% & 11,33\% & 0,00\% & 0,00\% \\ \cline{2-8} 
 & \multirow{4}{*}{60ms} & Est. & 0,00\% & 0,00\% & 0,00\% & 0,00\% & 0,00\% \\
 &  & Din. & 0,00\% & 0,00\% & 2,00\% & 0,00\% & 0,00\% \\
 &  & Est. & 0,00\% & 0,00\% & 0,00\% & 0,00\% & 0,00\% \\
 &  & Din. & 0,00\% & 0,00\% & 10,00\% & 0,00\% & 0,00\% \\ \hline
\multirow{12}{*}{\textgreater avg + 15} & \multirow{4}{*}{30ms} & Est. & 0,00\% & 0,00\% & 1,33\% & 0,00\% & 0,00\% \\
 &  & Din. & 0,00\% & 0,00\% & 4,00\% & 0,00\% & 0,00\% \\
 &  & Est. & 0,00\% & 0,00\% & 0,67\% & 0,00\% & 0,00\% \\
 &  & Din. & 0,00\% & 0,00\% & 4,67\% & 0,00\% & 0,00\% \\ \cline{2-8} 
 & \multirow{4}{*}{45ms} & Est. & 0,00\% & 0,00\% & 0,00\% & 0,00\% & 0,00\% \\
 &  & Din. & 0,00\% & 0,00\% & 1,33\% & 0,00\% & 0,00\% \\
 &  & Est. & 0,00\% & 0,00\% & 0,00\% & 0,00\% & 0,00\% \\
 &  & Din. & 0,00\% & 0,00\% & 6,67\% & 0,00\% & 0,00\% \\ \cline{2-8} 
 & \multirow{4}{*}{60ms} & Est. & 0,00\% & 0,00\% & 0,00\% & 0,00\% & 0,00\% \\
 &  & Din. & 0,00\% & 0,00\% & 7,33\% & 0,00\% & 0,00\% \\
 &  & Est. & 0,00\% & 0,00\% & 0,00\% & 0,00\% & 0,00\% \\
 &  & Din. & 0,00\% & 0,00\% & 6,67\% & 0,00\% & 0,00\% \\ \hline
\multirow{12}{*}{= 400} & \multirow{4}{*}{30ms} & Est. & 0,00\% & 0,00\% & 0,00\% & 0,00\% & 0,00\% \\
 &  & Din. & 0,00\% & 0,00\% & 0,00\% & 0,00\% & 0,00\% \\
 &  & Est. & 0,00\% & 0,00\% & 0,00\% & 0,00\% & 0,00\% \\
 &  & Din. & 0,00\% & 0,00\% & 0,00\% & 0,00\% & 0,00\% \\ \cline{2-8} 
 & \multirow{4}{*}{45ms} & Est. & 0,00\% & 0,00\% & 0,00\% & 0,00\% & 0,00\% \\
 &  & Din. & 0,00\% & 0,00\% & 0,00\% & 0,00\% & 0,00\% \\
 &  & Est. & 0,00\% & 0,00\% & 0,00\% & 0,00\% & 0,00\% \\
 &  & Din. & 0,00\% & 0,00\% & 0,00\% & 0,00\% & 0,00\% \\ \cline{2-8} 
 & \multirow{4}{*}{60ms} & Est. & 0,00\% & 0,00\% & 0,00\% & 0,00\% & 0,00\% \\
 &  & Din. & 0,00\% & 0,00\% & 0,00\% & 0,00\% & 0,00\% \\
 &  & Est. & 0,00\% & 0,00\% & 0,00\% & 0,00\% & 0,00\% \\
 &  & Din. & 0,00\% & 0,00\% & 0,00\% & 0,00\% & 0,00\% \\ \hline
\end{longtable}
\end{center}

% \pagebreak
\subsubsection{Teste 50cm}
Neste teste, houve grande incidência de medidas espúrias da ordem de grandeza de 100cm no sensor mais à direita do veículo, denominado Sensor 4 na 
Tabela \ref{50cm}, causando desvios padrão maiores do que 25cm para determinados testes, mesmo tendo sido encontrado obstáculo em todas medidas.
Para os demais sensores, os resultados obtidos foram satisfatórios, pois as medidas espúrias que surgiram foram esparsas e, desta vez, apareceram 
mais frequentemente nos testes com leituras estáticas.
  % Please add the following required packages to your document preamble:
% \usepackage{multirow}
\begin{center}
\begin{longtable}{|c|c|c|ccccc|}
\label{50cm} \\
\caption[50cm]{50cm} \endfirsthead \\
\caption[]{50cm} \\
\hline \multicolumn{3}{|c|}{{Continua na próxima página.}} \\ \hline
\endfoot \hline
\hline \hline
\endlastfoot
\hline
\textbf{Medida} & \textbf{Intervalo} & \textbf{Leitura} & \textbf{Sensor 0} & \textbf{Sensor 1} & \textbf{Sensor 2} & \textbf{Sensor 3} & \textbf{Sensor 4} \\ \hline
\multirow{12}{*}{\begin{tabular}[c]{@{}c@{}}Média\\ {[}cm{]}\end{tabular}} & \multirow{4}{*}{30ms} & Est. & 44,80 & 47,77 & 50,86 & 48,39 & 53,25 \\
 &  & Din. & 44,57 & 47,87 & 50,13 & 48,59 & 53,65 \\
 &  & Est. & 45,21 & 47,91 & 50,86 & 48,35 & 52,49 \\
 &  & Din. & 44,95 & 47,83 & 50,65 & 48,77 & 52,65 \\ \cline{2-8} 
 & \multirow{4}{*}{45ms} & Est. & 45,35 & 47,46 & 50,45 & 48,53 & 53,61 \\
 &  & Din. & 44,85 & 47,85 & 52,61 & 48,92 & 63,21 \\
 &  & Est. & 44,70 & 47,72 & 51,35 & 48,56 & 53,06 \\
 &  & Din. & 44,79 & 47,91 & 50,87 & 48,67 & 60,42 \\ \cline{2-8} 
 & \multirow{4}{*}{60ms} & Est. & 45,86 & 47,76 & 50,81 & 48,49 & 51,69 \\
 &  & Din. & 45,27 & 47,85 & 50,74 & 48,94 & 52,55 \\
 &  & Est. & 45,68 & 47,89 & 51,54 & 48,61 & 53,26 \\
 &  & Din. & 44,78 & 47,63 & 51,29 & 48,73 & 51,33 \\ \hline  \pagebreak
\multirow{12}{*}{\begin{tabular}[c]{@{}c@{}}Média\\ Limpa\\ {[}cm{]}\end{tabular}} & \multirow{4}{*}{30ms} & Est. & 44,80 & 47,77 & 50,86 & 48,39 & 53,25 \\
 &  & Din. & 44,57 & 47,87 & 50,13 & 48,59 & 53,65 \\
 &  & Est. & 45,21 & 47,91 & 50,86 & 48,35 & 52,49 \\
 &  & Din. & 44,95 & 47,83 & 50,65 & 48,77 & 52,65 \\ \cline{2-8} 
 & \multirow{4}{*}{45ms} & Est. & 45,35 & 47,46 & 50,45 & 48,53 & 53,61 \\
 &  & Din. & 44,85 & 47,85 & 52,61 & 48,92 & 63,21 \\
 &  & Est. & 44,70 & 47,72 & 51,35 & 48,56 & 53,06 \\
 &  & Din. & 44,79 & 47,91 & 50,87 & 48,67 & 60,42 \\ \cline{2-8} 
 & \multirow{4}{*}{60ms} & Est. & 45,86 & 47,76 & 50,81 & 48,49 & 51,69 \\
 &  & Din. & 45,27 & 47,85 & 50,74 & 48,94 & 52,55 \\
 &  & Est. & 45,68 & 47,89 & 51,54 & 48,61 & 53,26 \\
 &  & Din. & 44,78 & 47,63 & 51,29 & 48,73 & 51,33 \\ \hline
\multirow{12}{*}{\begin{tabular}[c]{@{}c@{}}Desvio\\ Padrão\\ {[}cm{]}\end{tabular}} & \multirow{4}{*}{30ms} & Est. & 0,65 & 0,42 & 0,74 & 0,50 & 11,99 \\
 &  & Din. & 0,56 & 0,33 & 1,48 & 0,53 & 12,93 \\
 &  & Est. & 2,96 & 0,29 & 0,74 & 0,48 & 11,24 \\
 &  & Din. & 1,00 & 0,37 & 1,17 & 0,49 & 11,11 \\ \cline{2-8} 
 & \multirow{4}{*}{45ms} & Est. & 3,38 & 0,50 & 0,97 & 0,51 & 13,46 \\
 &  & Din. & 0,63 & 0,36 & 1,16 & 0,54 & 28,66 \\
 &  & Est. & 0,76 & 0,45 & 0,81 & 0,50 & 12,03 \\
 &  & Din. & 0,53 & 0,28 & 0,70 & 0,51 & 26,26 \\ \cline{2-8} 
 & \multirow{4}{*}{60ms} & Est. & 7,40 & 0,43 & 0,87 & 0,50 & 7,44 \\
 &  & Din. & 2,98 & 0,35 & 0,81 & 0,48 & 8,89 \\
 &  & Est. & 6,25 & 0,31 & 0,81 & 0,50 & 11,92 \\
 &  & Din. & 0,53 & 0,48 & 1,30 & 0,49 & 5,41 \\ \hline
\multirow{12}{*}{\textless avg - 15} & \multirow{4}{*}{30ms} & Est. & 0,00\% & 0,00\% & 0,00\% & 0,00\% & 0,00\% \\
 &  & Din. & 0,00\% & 0,00\% & 0,00\% & 0,00\% & 0,00\% \\
 &  & Est. & 0,00\% & 0,00\% & 0,00\% & 0,00\% & 0,00\% \\
 &  & Din. & 0,00\% & 0,00\% & 0,00\% & 0,00\% & 0,00\% \\ \cline{2-8} 
 & \multirow{4}{*}{45ms} & Est. & 0,00\% & 0,00\% & 0,00\% & 0,00\% & 0,00\% \\
 &  & Din. & 0,00\% & 0,00\% & 0,00\% & 0,00\% & 0,00\% \\
 &  & Est. & 0,00\% & 0,00\% & 0,00\% & 0,00\% & 0,00\% \\
 &  & Din. & 0,00\% & 0,00\% & 0,00\% & 0,00\% & 0,00\% \\ \cline{2-8} 
 & \multirow{4}{*}{60ms} & Est. & 0,00\% & 0,00\% & 0,00\% & 0,00\% & 0,00\% \\
 &  & Din. & 0,00\% & 0,00\% & 0,00\% & 0,00\% & 0,00\% \\
 &  & Est. & 0,00\% & 0,00\% & 0,00\% & 0,00\% & 0,00\% \\
 &  & Din. & 0,00\% & 0,00\% & 0,00\% & 0,00\% & 0,00\% \\ \hline
\multirow{12}{*}{\textgreater avg + 15} & \multirow{4}{*}{30ms} & Est. & 0,00\% & 0,00\% & 0,00\% & 0,00\% & 4,00\% \\
 &  & Din. & 0,00\% & 0,00\% & 0,00\% & 0,00\% & 5,33\% \\
 &  & Est. & 0,67\% & 0,00\% & 0,00\% & 0,00\% & 2,67\% \\
 &  & Din. & 0,00\% & 0,00\% & 0,00\% & 0,00\% & 2,67\% \\ \cline{2-8} 
 & \multirow{4}{*}{45ms} & Est. & 0,67\% & 0,00\% & 0,00\% & 0,00\% & 4,00\% \\
 &  & Din. & 0,00\% & 0,00\% & 0,00\% & 0,00\% & 16,00\% \\
 &  & Est. & 0,00\% & 0,00\% & 0,00\% & 0,00\% & 3,33\% \\
 &  & Din. & 0,00\% & 0,00\% & 0,00\% & 0,00\% & 12,67\% \\ \cline{2-8} 
 & \multirow{4}{*}{60ms} & Est. & 1,33\% & 0,00\% & 0,00\% & 0,00\% & 1,33\% \\
 &  & Din. & 0,67\% & 0,00\% & 0,00\% & 0,00\% & 2,67\% \\
 &  & Est. & 1,33\% & 0,00\% & 0,00\% & 0,00\% & 4,67\% \\
 &  & Din. & 0,00\% & 0,00\% & 0,00\% & 0,00\% & 0,67\% \\ \hline
\multirow{12}{*}{= 400} & \multirow{4}{*}{30ms} & Est. & 0,00\% & 0,00\% & 0,00\% & 0,00\% & 0,00\% \\
 &  & Din. & 0,00\% & 0,00\% & 0,00\% & 0,00\% & 0,00\% \\
 &  & Est. & 0,00\% & 0,00\% & 0,00\% & 0,00\% & 0,00\% \\
 &  & Din. & 0,00\% & 0,00\% & 0,00\% & 0,00\% & 0,00\% \\ \cline{2-8} 
 & \multirow{4}{*}{45ms} & Est. & 0,00\% & 0,00\% & 0,00\% & 0,00\% & 0,00\% \\
 &  & Din. & 0,00\% & 0,00\% & 0,00\% & 0,00\% & 0,00\% \\
 &  & Est. & 0,00\% & 0,00\% & 0,00\% & 0,00\% & 0,00\% \\
 &  & Din. & 0,00\% & 0,00\% & 0,00\% & 0,00\% & 0,00\% \\ \cline{2-8} 
 & \multirow{4}{*}{60ms} & Est. & 0,00\% & 0,00\% & 0,00\% & 0,00\% & 0,00\% \\
 &  & Din. & 0,00\% & 0,00\% & 0,00\% & 0,00\% & 0,00\% \\
 &  & Est. & 0,00\% & 0,00\% & 0,00\% & 0,00\% & 0,00\% \\
 &  & Din. & 0,00\% & 0,00\% & 0,00\% & 0,00\% & 0,00\% \\ \hline
\end{longtable}
\end{center}

\subsubsection{Teste 100cm}
Neste teste, não houve diferença significativa dos resultados obtidos anteriormente, houve apenas uma leitura destoante de mais do que 15cm 
da média limpa: 116cm. 
  % Please add the following required packages to your document preamble:
% \usepackage{multirow}
\begin{center}
\begin{longtable}{|c|c|c|ccccc|}
\label{100cm} \\
\caption[100cm]{100cm} \endfirsthead \\
\caption[]{100cm} \\
\hline \multicolumn{3}{|c|}{{Continua na próxima página.}} \\ \hline
\endfoot \hline
\hline \hline
\endlastfoot
\hline
\textbf{Medida} & \textbf{Intervalo} & \textbf{Leitura} & \textbf{Sensor 0} & \textbf{Sensor 1} & \textbf{Sensor 2} & \textbf{Sensor 3} & \textbf{Sensor 4} \\ \hline
\multirow{12}{*}{\begin{tabular}[c]{@{}c@{}}Média\\ {[}cm{]}\end{tabular}} & \multirow{4}{*}{30ms} & Est. & 102,49 & 106,83 & 96,83 & 92,05 & 94,19 \\
 &  & Din. & 102,71 & 107,05 & 97,07 & 92,46 & 94,74 \\
 &  & Est. & 102,41 & 106,85 & 96,83 & 92,21 & 94,49 \\
 &  & Din. & 102,46 & 106,91 & 97,19 & 92,54 & 94,54 \\ \cline{2-8} 
 & \multirow{4}{*}{45ms} & Est. & 102,78 & 106,86 & 96,96 & 92,05 & 94,29 \\
 &  & Din. & 102,31 & 106,82 & 97,10 & 92,37 & 94,49 \\
 &  & Est. & 102,41 & 106,89 & 97,07 & 92,20 & 94,27 \\
 &  & Din. & 102,37 & 106,83 & 97,07 & 92,23 & 94,20 \\ \cline{2-8} 
 & \multirow{4}{*}{60ms} & Est. & 102,63 & 106,96 & 97,46 & 92,14 & 94,24 \\
 &  & Din. & 102,34 & 106,81 & 97,23 & 92,46 & 94,53 \\
 &  & Est. & 102,59 & 106,90 & 97,07 & 92,43 & 94,60 \\
 &  & Din. & 102,07 & 106,77 & 97,18 & 92,51 & 94,65 \\ \hline
\multirow{12}{*}{\begin{tabular}[c]{@{}c@{}}Média\\ Limpa\\ {[}cm{]}\end{tabular}} & \multirow{4}{*}{30ms} & Est. & 102,49 & 106,83 & 96,83 & 92,05 & 94,19 \\
 &  & Din. & 102,71 & 107,05 & 97,07 & 92,46 & 94,74 \\
 &  & Est. & 102,41 & 106,85 & 96,83 & 92,21 & 94,49 \\
 &  & Din. & 102,46 & 106,91 & 97,19 & 92,54 & 94,54 \\ \cline{2-8} 
 & \multirow{4}{*}{45ms} & Est. & 102,78 & 106,86 & 96,96 & 92,05 & 94,29 \\
 &  & Din. & 102,31 & 106,82 & 97,10 & 92,37 & 94,49 \\
 &  & Est. & 102,41 & 106,89 & 97,07 & 92,20 & 94,27 \\
 &  & Din. & 102,37 & 106,83 & 97,07 & 92,23 & 94,20 \\ \cline{2-8} 
 & \multirow{4}{*}{60ms} & Est. & 102,63 & 106,96 & 97,46 & 92,14 & 94,24 \\
 &  & Din. & 102,34 & 106,81 & 97,23 & 92,46 & 94,53 \\
 &  & Est. & 102,59 & 106,90 & 97,07 & 92,43 & 94,60 \\
 &  & Din. & 102,07 & 106,77 & 97,18 & 92,51 & 94,65 \\ \hline
\multirow{12}{*}{\begin{tabular}[c]{@{}c@{}}Desvio\\ Padrão\\ {[}cm{]}\end{tabular}} & \multirow{4}{*}{30ms} & Est. & 0,66 & 0,39 & 0,78 & 0,24 & 1,02 \\
 &  & Din. & 0,99 & 0,44 & 0,73 & 0,53 & 1,12 \\
 &  & Est. & 0,66 & 0,38 & 0,77 & 0,41 & 1,01 \\
 &  & Din. & 0,66 & 0,29 & 0,75 & 0,57 & 0,98 \\ \cline{2-8} 
 & \multirow{4}{*}{45ms} & Est. & 0,83 & 0,35 & 0,72 & 0,33 & 1,01 \\
 &  & Din. & 0,60 & 0,39 & 0,78 & 0,56 & 0,99 \\
 &  & Est. & 0,66 & 0,31 & 0,77 & 0,42 & 0,83 \\
 &  & Din. & 0,66 & 0,40 & 0,81 & 0,44 & 0,62 \\ \cline{2-8}  \pagebreak
 & \multirow{4}{*}{60ms} & Est. & 0,69 & 0,33 & 2,52 & 0,37 & 1,07 \\
 &  & Din. & 0,61 & 0,39 & 0,76 & 0,54 & 0,95 \\
 &  & Est. & 0,73 & 0,34 & 0,72 & 0,50 & 1,22 \\
 &  & Din. & 0,53 & 0,42 & 0,71 & 0,58 & 0,99 \\ \hline
\multirow{12}{*}{\textless avg - 15} & \multirow{4}{*}{30ms} & Est. &0\% &0\% &0\% &0\% &0\% \\
 &  & Din. &0\% &0\% &0\% &0\% &0\% \\
 &  & Est. &0\% &0\% &0\% &0\% &0\% \\
 &  & Din. &0\% &0\% &0\% &0\% &0\% \\ \cline{2-8} 
 & \multirow{4}{*}{45ms} & Est. &0\% &0\% &0\% &0\% &0\% \\
 &  & Din. &0\% &0\% &0\% &0\% &0\% \\
 &  & Est. &0\% &0\% &0\% &0\% &0\% \\
 &  & Din. &0\% &0\% &0\% &0\% &0\% \\ \cline{2-8} 
 & \multirow{4}{*}{60ms} & Est. &0\% &0\% &0\% &0\% &0\% \\
 &  & Din. &0\% &0\% &0\% &0\% &0\% \\
 &  & Est. &0\% &0\% &0\% &0\% &0\% \\
 &  & Din. &0\% &0\% &0\% &0\% &0\% \\ \hline
\multirow{12}{*}{\textgreater avg + 15} & \multirow{4}{*}{30ms} & Est. &0\% &0\% &0\% &0\% &0\% \\
 &  & Din. &0\% &0\% &0\% &0\% &0\% \\
 &  & Est. &0\% &0\% &0\% &0\% &0\% \\
 &  & Din. &0\% &0\% &0\% &0\% &0\% \\ \cline{2-8} 
 & \multirow{4}{*}{45ms} & Est. &0\% &0\% &0\% &0\% &0\% \\
 &  & Din. &0\% &0\% &0\% &0\% &0\% \\
 &  & Est. &0\% &0\% &0\% &0\% &0\% \\
 &  & Din. &0\% &0\% &0\% &0\% &0\% \\ \cline{2-8} 
 & \multirow{4}{*}{60ms} & Est. &0\% &0\% & 0,67\% &0\% &0\% \\
 &  & Din. &0\% &0\% &0\% &0\% &0\% \\
 &  & Est. &0\% &0\% &0\% &0\% &0\% \\
 &  & Din. &0\% &0\% &0\% &0\% &0\% \\ \hline
\multirow{12}{*}{= 400} & \multirow{4}{*}{30ms} & Est. &0\% &0\% &0\% &0\% &0\% \\
 &  & Din. &0\% &0\% &0\% &0\% &0\% \\
 &  & Est. &0\% &0\% &0\% &0\% &0\% \\
 &  & Din. &0\% &0\% &0\% &0\% &0\% \\ \cline{2-8} 
 & \multirow{4}{*}{45ms} & Est. &0\% &0\% &0\% &0\% &0\% \\
 &  & Din. &0\% &0\% &0\% &0\% &0\% \\
 &  & Est. &0\% &0\% &0\% &0\% &0\% \\
 &  & Din. &0\% &0\% &0\% &0\% &0\% \\ \cline{2-8} 
 & \multirow{4}{*}{60ms} & Est. &0\% &0\% &0\% &0\% &0\% \\
 &  & Din. &0\% &0\% &0\% &0\% &0\% \\
 &  & Est. &0\% &0\% &0\% &0\% &0\% \\
 &  & Din. &0\% &0\% &0\% &0\% &0\% \\ \hline
\end{longtable}
\end{center}

\subsubsection{Teste 150cm}
Pela primeira vez surgiram leituras em que algum sonar não encontrou obstáculos, todas ocorreram no sensor da extremidade direita, no entanto, não 
houve distinção notória de incidência deste fenômeno dos intervalos de medições fixos em relação aos variáveis, de modo que, para a discussão em 
pauta, essas medidas não agregam valor significativo.
Tal qual ocorreu no teste de 100cm, as leituras de intervalo dinâmico apresentaram um desempenho levemente superior em relação a medidas destoantes.
  % Please add the following required packages to your document preamble:
% \usepackage{multirow}
\begin{center}
\begin{longtable}{|c|c|c|ccccc|}
\label{150cm} \\
\caption[150cm]{150cm} \endfirsthead \\
\caption[]{150cm} \\
\hline \multicolumn{3}{|c|}{{Continua na próxima página.}} \\ \hline
\endfoot \hline
\hline \hline
\endlastfoot
\hline
\textbf{Medida} & \textbf{Intervalo} & \textbf{Leitura} & \textbf{Sensor 0} & \textbf{Sensor 1} & \textbf{Sensor 2} & \textbf{Sensor 3} & \textbf{Sensor 4} \\ \hline
\multirow{12}{*}{\begin{tabular}[c]{@{}c@{}}Média\\ {[}cm{]}\end{tabular}} & \multirow{4}{*}{30ms} & Est. & 148,52 & 152,17 & 152,97 & 156,65 & 230,77 \\
 &  & Din. & 148,70 & 153,18 & 153,45 & 156,41 & 174,46 \\
 &  & Est. & 148,97 & 153,53 & 153,65 & 156,53 & 196,81 \\
 &  & Din. & 149,05 & 153,01 & 153,63 & 156,95 & 176,01 \\ \cline{2-8} 
 & \multirow{4}{*}{45ms} & Est. & 149,13 & 153,13 & 153,95 & 156,42 & 184,03 \\
 &  & Din. & 148,62 & 153,01 & 153,10 & 156,97 & 226,32 \\
 &  & Est. & 148,63 & 151,45 & 153,23 & 157,11 & 210,04 \\
 &  & Din. & 148,90 & 153,04 & 154,24 & 157,13 & 192,36 \\ \cline{2-8} 
 & \multirow{4}{*}{60ms} & Est. & 148,84 & 153,09 & 153,42 & 156,79 & 256,77 \\
 &  & Din. & 148,53 & 152,96 & 153,33 & 157,28 & 161,61 \\
 &  & Est. & 148,49 & 152,95 & 153,03 & 156,81 & 164,53 \\
 &  & Din. & 148,45 & 152,57 & 152,18 & 156,33 & 229,58 \\ \hline \multirow{12}{*}{\begin{tabular}[c]{@{}c@{}}Média\\ Limpa\\ {[}cm{]}\end{tabular}} & \multirow{4}{*}{30ms} & Est. & 148,52 & 152,17 & 152,97 & 156,65 & 158,25 \\
 &  & Din. & 148,70 & 153,18 & 153,45 & 156,41 & 158,35 \\
 &  & Est. & 148,97 & 153,53 & 153,65 & 156,53 & 158,10 \\
 &  & Din. & 149,05 & 153,01 & 153,63 & 156,95 & 158,28 \\ \cline{2-8} 
 & \multirow{4}{*}{45ms} & Est. & 149,13 & 153,13 & 153,95 & 156,42 & 158,25 \\
 &  & Din. & 148,62 & 153,01 & 153,10 & 156,97 & 158,78 \\
 &  & Est. & 148,63 & 151,45 & 153,23 & 157,11 & 158,53 \\
 &  & Din. & 148,90 & 153,04 & 154,24 & 157,13 & 158,56 \\ \cline{2-8} 
 & \multirow{4}{*}{60ms} & Est. & 148,84 & 153,09 & 153,42 & 156,79 & 158,60 \\
 &  & Din. & 148,53 & 152,96 & 153,33 & 157,28 & 158,39 \\
 &  & Est. & 148,49 & 152,95 & 153,03 & 156,81 & 158,08 \\
 &  & Din. & 148,45 & 152,57 & 152,18 & 156,33 & 158,84 \\ \hline \pagebreak
\multirow{12}{*}{\begin{tabular}[c]{@{}c@{}}Desvio\\ Padrão\\ {[}cm{]}\end{tabular}} & \multirow{4}{*}{30ms} & Est. & 0,60 & 7,48 & 6,44 & 0,76 & 111,16 \\
 &  & Din. & 0,63 & 0,39 & 0,64 & 0,88 & 60,49 \\
 &  & Est. & 0,52 & 0,50 & 0,65 & 0,68 & 88,98 \\
 &  & Din. & 0,47 & 0,08 & 0,61 & 0,68 & 63,23 \\ \cline{2-8} 
 & \multirow{4}{*}{45ms} & Est. & 0,42 & 0,33 & 0,73 & 0,74 & 74,88 \\
 &  & Din. & 0,63 & 0,23 & 0,86 & 0,82 & 108,67 \\
 &  & Est. & 0,61 & 9,46 & 0,89 & 0,64 & 99,26 \\
 &  & Din. & 0,47 & 0,30 & 6,01 & 0,72 & 84,06 \\ \cline{2-8} 
 & \multirow{4}{*}{60ms} & Est. & 0,57 & 0,28 & 0,63 & 0,57 & 118,98 \\
 &  & Din. & 0,62 & 0,20 & 0,62 & 0,65 & 27,81 \\
 &  & Est. & 0,54 & 0,21 & 0,68 & 0,61 & 39,11 \\
 &  & Din. & 0,59 & 4,90 & 5,68 & 0,89 & 110,18 \\ \hline
\multirow{12}{*}{\textless avg - 15} & \multirow{4}{*}{30ms} & Est. & 0,00\% & 1,33\% & 0,67\% & 0,00\% & 0,00\% \\
 &  & Din. & 0,00\% & 0,00\% & 0,00\% & 0,00\% & 0,00\% \\
 &  & Est. & 0,00\% & 0,00\% & 0,00\% & 0,00\% & 0,00\% \\
 &  & Din. & 0,00\% & 0,00\% & 0,00\% & 0,00\% & 0,00\% \\ \cline{2-8} 
 & \multirow{4}{*}{45ms} & Est. & 0,00\% & 0,00\% & 0,00\% & 0,00\% & 0,00\% \\
 &  & Din. & 0,00\% & 0,00\% & 0,00\% & 0,00\% & 0,00\% \\
 &  & Est. & 0,00\% & 2,67\% & 0,00\% & 0,00\% & 0,00\% \\
 &  & Din. & 0,00\% & 0,00\% & 0,00\% & 0,00\% & 0,00\% \\ \cline{2-8} 
 & \multirow{4}{*}{60ms} & Est. & 0,00\% & 0,00\% & 0,00\% & 0,00\% & 0,00\% \\
 &  & Din. & 0,00\% & 0,00\% & 0,00\% & 0,00\% & 0,00\% \\
 &  & Est. & 0,00\% & 0,00\% & 0,00\% & 0,00\% & 0,00\% \\
 &  & Din. & 0,00\% & 0,67\% & 1,33\% & 0,00\% & 0,00\% \\ \hline
\multirow{12}{*}{\textgreater avg + 15} & \multirow{4}{*}{30ms} & Est. & 0,00\% & 0,00\% & 0,00\% & 0,00\% & 0,00\% \\
 &  & Din. & 0,00\% & 0,00\% & 0,00\% & 0,00\% & 0,00\% \\
 &  & Est. & 0,00\% & 0,00\% & 0,00\% & 0,00\% & 0,00\% \\
 &  & Din. & 0,00\% & 0,00\% & 0,00\% & 0,00\% & 0,00\% \\ \cline{2-8} 
 & \multirow{4}{*}{45ms} & Est. & 0,00\% & 0,00\% & 0,00\% & 0,00\% & 0,00\% \\
 &  & Din. & 0,00\% & 0,00\% & 0,00\% & 0,00\% & 0,00\% \\
 &  & Est. & 0,00\% & 0,00\% & 0,00\% & 0,00\% & 0,00\% \\
 &  & Din. & 0,00\% & 0,00\% & 0,67\% & 0,00\% & 0,00\% \\ \cline{2-8} 
 & \multirow{4}{*}{60ms} & Est. & 0,00\% & 0,00\% & 0,00\% & 0,00\% & 0,00\% \\
 &  & Din. & 0,00\% & 0,00\% & 0,00\% & 0,00\% & 0,00\% \\
 &  & Est. & 0,00\% & 0,00\% & 0,00\% & 0,00\% & 0,00\% \\
 &  & Din. & 0,00\% & 0,00\% & 0,00\% & 0,00\% & 0,00\% \\ \hline
\multirow{12}{*}{= 400} & \multirow{4}{*}{30ms} & Est. & 0,00\% & 0,00\% & 0,00\% & 0,00\% & 30,00\% \\
 &  & Din. & 0,00\% & 0,00\% & 0,00\% & 0,00\% & 6,67\% \\
 &  & Est. & 0,00\% & 0,00\% & 0,00\% & 0,00\% & 16,00\% \\
 &  & Din. & 0,00\% & 0,00\% & 0,00\% & 0,00\% & 7,33\% \\ \cline{2-8} 
 & \multirow{4}{*}{45ms} & Est. & 0,00\% & 0,00\% & 0,00\% & 0,00\% & 10,67\% \\
 &  & Din. & 0,00\% & 0,00\% & 0,00\% & 0,00\% & 28,00\% \\
 &  & Est. & 0,00\% & 0,00\% & 0,00\% & 0,00\% & 21,33\% \\
 &  & Din. & 0,00\% & 0,00\% & 0,00\% & 0,00\% & 14,00\% \\ \cline{2-8} 
 & \multirow{4}{*}{60ms} & Est. & 0,00\% & 0,00\% & 0,00\% & 0,00\% & 40,67\% \\
 &  & Din. & 0,00\% & 0,00\% & 0,00\% & 0,00\% & 1,33\% \\
 &  & Est. & 0,00\% & 0,00\% & 0,00\% & 0,00\% & 2,67\% \\
 &  & Din. & 0,00\% & 0,00\% & 0,00\% & 0,00\% & 29,33\% \\ \hline
\end{longtable}
\end{center}


\subsubsection{Teste 200cm}
No teste em que o carrinho estava a 200 cm das paredes, foram registradas no sensor frontal, em sua grande maioria, medidas em dois intervalos 
estreitos: entre 185cm e 195cm, e de 265cm a 275cm, o que fez com que as medidas de desvio padrão fossem bem altas, chegando a mais de 40cm.
No entanto, é interessante notar que o segundo intervalo é aproximadamente o primeiro multiplicado pela raiz quadrada de dois, logo, é possível que 
as medidas obtidas no segundo intervalo sejam dos pacotes de ondas acústicas emitidas pelo sonar frontal, enquanto as do primeiro intervalo seriam 
decorrência de \textit{crosstalk}.
De qualquer forma, neste teste, no que se refere ao sensor frontal, não é possível extrair grandes informações no que concerne a qual tipo de medição 
é melhor, logo, ele será desconsiderado na análise a seguir.

No sensor da extremidade esquerda, denominado Sensor 0 na Tabela \ref{200cm}, houve diversas medidas em que não foi detectado o obstáculo, no 
entanto, não houve discrepância considerável na ocorrência deste fenômeno dentre os dois tipos de intervalos de medição.

De um modo geral, a partir dos dados coletados neste teste, nota-se que não há deterioração das medidas obtidas dos sonares quando utiliza-se 
intervalos dinâmicos que, no caso deste, durariam em média menos do que 15ms por leitura, o que possibilitaria pelo menos o dobro de medidas 
para um mesmo período.
  % Please add the following required packages to your document preamble:
% \usepackage{multirow}
\begin{center}
\begin{longtable}{|c|c|c|ccccc|}
\label{200cm} \\
\caption[200cm]{200cm} \endfirsthead \\
\caption[]{200cm} \\
\hline \multicolumn{3}{|c|}{{Continua na próxima página.}} \\ \hline
\endfoot \hline
\hline \hline
\endlastfoot
\hline
\textbf{Medida} & \textbf{Intervalo} & \textbf{Leitura} & \textbf{Sensor 0} & \textbf{Sensor 1} & \textbf{Sensor 2} & \textbf{Sensor 3} & \textbf{Sensor 4} \\ \hline
\multirow{12}{*}{\begin{tabular}[c]{@{}c@{}}Média\\ {[}cm{]}\end{tabular}} & \multirow{4}{*}{30ms} & Est. & 197,58 & 189,40 & 192,26 & 182,45 & 182,72 \\
 &  & Din. & 209,51 & 189,56 & 213,76 & 182,13 & 182,67 \\
 &  & Est. & 216,61 & 189,42 & 206,54 & 182,01 & 182,48 \\
 &  & Din. & 191,33 & 189,25 & 202,63 & 182,43 & 182,72 \\ \cline{2-8} 
 & \multirow{4}{*}{45ms} & Est. & 196,53 & 189,63 & 220,25 & 182,33 & 182,75 \\
 &  & Din. & 239,21 & 189,51 & 208,81 & 182,87 & 183,20 \\
 &  & Est. & 201,98 & 189,55 & 209,55 & 182,76 & 183,12 \\
 &  & Din. & 242,49 & 188,22 & 192,27 & 182,97 & 183,50 \\ \cline{2-8} 
 & \multirow{4}{*}{60ms} & Est. & 199,45 & 189,49 & 189,49 & 182,46 & 186,07 \\
 &  & Din. & 197,21 & 189,61 & 212,97 & 182,71 & 183,24 \\
 &  & Est. & 237,66 & 189,77 & 196,97 & 182,60 & 182,85 \\
\multirow{12}{*}{\begin{tabular}[c]{@{}c@{}}Média\\ Limpa\\ {[}cm{]}\end{tabular}} & \multirow{4}{*}{30ms} & Est. & 194,84 & 189,40 & 192,26 & 182,45 & 182,72 \\
 &  & Din. & 200,18 & 189,56 & 211,24 & 182,13 & 182,67 \\
 &  & Est. & 194,72 & 189,42 & 206,54 & 182,01 & 182,48 \\
 &  & Din. & 191,33 & 189,25 & 202,63 & 182,43 & 182,72 \\ \cline{2-8} 
 & \multirow{4}{*}{45ms} & Est. & 196,53 & 189,63 & 220,25 & 182,33 & 182,75 \\
 &  & Din. & 197,32 & 189,51 & 208,81 & 182,87 & 183,20 \\
 &  & Est. & 193,73 & 189,55 & 209,55 & 182,76 & 183,12 \\
 &  & Din. & 198,06 & 188,22 & 192,27 & 182,97 & 183,50 \\ \cline{2-8} 
 & \multirow{4}{*}{60ms} & Est. & 191,09 & 189,49 & 189,49 & 182,46 & 183,18 \\
 &  & Din. & 191,65 & 189,61 & 212,97 & 182,71 & 183,24 \\
 &  & Est. & 190,08 & 189,77 & 196,97 & 182,60 & 182,85 \\
 &  & Din. & 189,75 & 188,77 & 192,45 & 182,79 & 182,97 \\ \hline
\multirow{12}{*}{\begin{tabular}[c]{@{}c@{}}Desvio\\ Padrão\\ {[}cm{]}\end{tabular}} & \multirow{4}{*}{30ms} & Est. & 25,47 & 0,51 & 16,06 & 0,56 & 0,72 \\
 &  & Din. & 43,75 & 0,50 & 42,00 & 0,65 & 0,71 \\
 &  & Est. & 64,48 & 0,51 & 32,18 & 0,74 & 0,66 \\
 &  & Din. & 5,85 & 0,43 & 28,28 & 0,82 & 0,77 \\ \cline{2-8} 
 & \multirow{4}{*}{45ms} & Est. & 10,08 & 0,49 & 39,51 & 0,48 & 0,69 \\
 &  & Din. & 82,97 & 0,53 & 34,85 & 0,75 & 0,93 \\
 &  & Est. & 41,49 & 0,53 & 33,54 & 0,65 & 0,90 \\
 &  & Din. & 84,75 & 9,48 & 14,72 & 0,79 & 1,24 \\ \cline{2-8} 
 & \multirow{4}{*}{60ms} & Est. & 41,69 & 0,53 & 6,76 & 0,54 & 24,97 \\
 &  & Din. & 33,94 & 0,50 & 35,49 & 0,64 & 1,14 \\
 &  & Est. & 88,21 & 0,42 & 22,80 & 0,67 & 0,83 \\
 &  & Din. & 41,38 & 7,62 & 13,07 & 0,70 & 0,90 \\ \hline
\multirow{12}{*}{\textless avg - 15} & \multirow{4}{*}{30ms} & Est. &0\% &0\% &0\% &0\% &0\% \\
 &  & Din. &0\% &0\% & 71,33\% &0\% &0\% \\
 &  & Est. &0\% &0\% & 54,00\% &0\% &0\% \\
 &  & Din. &0\% &0\% & 1,33\% &0\% &0\% \\ \cline{2-8} 
 & \multirow{4}{*}{45ms} & Est. &0\% &0\% & 62,00\% &0\% &0\% \\
 &  & Din. &0\% &0\% & 73,33\% &0\% &0\% \\
 &  & Est. &0\% &0\% & 75,33\% &0\% &0\% \\
 &  & Din. &0\% & 2,67\% &0\% &0\% &0\% \\ \cline{2-8} 
 & \multirow{4}{*}{60ms} & Est. &0\% &0\% &0\% &0\% &0\% \\
 &  & Din. &0\% &0\% & 72,00\% &0\% &0\% \\
 &  & Est. &0\% &0\% &0\% &0\% &0\% \\
 &  & Din. &0\% & 2,00\% &0\% &0\% &0\% \\ \hline
\multirow{12}{*}{\textgreater avg + 15} & \multirow{4}{*}{30ms} & Est. & 18,67\% &0\% & 4,00\% &0\% &0\% \\
 &  & Din. & 0,67\% &0\% & 26,67\% &0\% &0\% \\
 &  & Est. & 16,00\% &0\% & 20,00\% &0\% &0\% \\
 &  & Din. & 4,67\% &0\% & 14,67\% &0\% &0\% \\ \cline{2-8} 
 & \multirow{4}{*}{45ms} & Est. & 25,33\% &0\% & 38,00\% &0\% &0\% \\
 &  & Din. & 11,33\% &0\% & 24,00\% &0\% &0\% \\
 &  & Est. & 10,00\% &0\% & 22,67\% &0\% &0\% \\
 &  & Din. & 15,33\% &0\% & 3,33\% &0\% &0\% \\ \cline{2-8} 
 & \multirow{4}{*}{60ms} & Est. & 6,00\% &0\% & 0,67\% &0\% &0\% \\
 &  & Din. & 2,67\% &0\% & 27,33\% &0\% &0\% \\
 &  & Est. &0\% &0\% & 8,67\% &0\% &0\% \\
 &  & Din. &0\% &0\% & 2,67\% &0\% &0\% \\ \hline
\multirow{12}{*}{= 400} & \multirow{4}{*}{30ms} & Est. & 1,33\% &0\% &0\% &0\% &0\% \\
 &  & Din. & 4,67\% &0\% & 1,33\% &0\% &0\% \\
 &  & Est. & 10,67\% &0\% &0\% &0\% &0\% \\
 &  & Din. &0\% &0\% &0\% &0\% &0\% \\ \cline{2-8} 
 & \multirow{4}{*}{45ms} & Est. &0\% &0\% &0\% &0\% &0\% \\
 &  & Din. & 20,67\% &0\% &0\% &0\% &0\% \\
 &  & Est. & 4,00\% &0\% &0\% &0\% &0\% \\
 &  & Din. & 22,00\% &0\% &0\% &0\% &0\% \\ \cline{2-8} 
 & \multirow{4}{*}{60ms} & Est. & 4,00\% &0\% &0\% &0\% & 1,33\% \\
 &  & Din. & 2,67\% &0\% &0\% &0\% &0\% \\
 &  & Est. & 22,67\% &0\% &0\% &0\% &0\% \\
 &  & Din. & 4,00\% &0\% &0\% &0\% &0\% \\ \hline
\end{longtable}
\end{center}



\section{Testes de Integração}
\subsection{Interferência dos BLDC nos Sonares}
Notou-se que em determinadas condições o robô ia de encontro ao obstáculo ao invés de efetuar o desvio. 
Após serem analisados os dados dos sensores nessas circunstâncias, observou-se a existência de ruídos nos sensores, que causavam a adoção 
destes comportamentos errados.

Ao perceber o problema, novos testes foram engendrados a fim de descobrir a natureza da falha.
As hipóteses concebidas eram as seguintes: \textit{crosstalk}, curto circuito entre pinos do Arduino, falha na lógica do \textit{software} ou 
interferência dos motores nos sensores.

Para eliminar a possibilidade de que uma das portas da placa de prototipação estivesse interferindo na outra de alguma maneira, a 
ligação dos sensores ultrassônicos no Arduino foram mudadas mas o problema se manteve.
O mesmo foi feito no \textit{software}, i.e. foi alterada a disposição dos sensores ultrassônicos no código. 
Especificamente falando, foram trocados os parâmetros que correlacionam a ligação física do pino de \textit{echo} dos dispositivos a sua variável 
correspondente na matriz de estruturas do tipo \textit{sensor\_t}, denominada no programa por USS, e, mais uma vez, o defeito persistiu.
Adicionalmente a essa modificação no \textit{software}, foram feitas medições com os motores desligados, nas quais a falha em questão não ocorreu, 
evidenciando que a natureza do problema não era do código e nem os pinos da placa de prototipação.

O teste seguinte consistiu em desacoplar os sensores ultrassônicos da carcaça do veículo apontando-os para um lugar livre de obstáculos e ligar os 
motores com os sonares sendo segurados na mão. 
No resultado, observou-se a desaparição total das leituras espúrias, confirmando a hipótese de que os BLDC causavam de alguma maneira a distorção na 
percepção do robô.
Em vista disso, supôs-se que a natureza da interferência seria em razão da proximidade entre os dispositivos, como interferência eletromagnética nos 
pinos de \textit{echo} dos sonares ou então de ondas acústica na banda de operação dos sensores, e optou-se por erguer os sensores a uma altura 
na qual não houvesse interferência suficiente a ponto de provocar erros de medição.

Em seguida foram feitas modificações na estrutura do veículo para distanciar os sonares dos motores: foram fixadas três hastes metálicas de 30 
centímetros na carcaça do veículo, nestas foram fixados suportes nos quais foi parafusado o para-choque do veículo, onde estavam colados com 
cola epóxi os 5 sensores ultrassônicos. Além disso, as ligações dos sonares entre si e com o Arduino foram refeitas, soldando de fato os fios ao 
invés de serem utilizados \textit{jumpers} como outrora, com o intuito de reduzir ao máximo qualquer tipo de ruído causado por mal contato.

Após isso, foi engendrado um novo teste a fim de constatar se com a nova disposição o problema havia sido sanado: cada uma das combinações de 
velocidades dos motores foi mantida por dez segundos enquanto os sonares faziam as medições.
Os dados obtidos, vide Tabela \ref{teste_4}, apontaram que o problema persistia, apresentando inclusive detecções espúrias de obstáculos a mais 
do que 400cm, distância que se encontra além do alcance dos sonares.

% Please add the following required packages to your document preamble:
% \usepackage{multirow}
\begin{table}[H]
\centering
\caption{Teste com os Sonares Acoplados ao Veículo}
\label{teste_4}
\begin{tabular}{|c|c|ccccc|}
\hline
\textbf{Motores}                & \textbf{Medida}  & \textbf{Sensor 0} & \textbf{Sensor 1} & \textbf{Sensor 2} & \textbf{Sensor 3} & \textbf{Sensor 4} \\ \hline
\multirow{3}{*}{STP}            & Média            & 400               & 400               & 400               & 398               & 400               \\
                                & \textless 400    & 0,23\%            & 0\%            & 0\%            & 0,70\%            & 0\%            \\
                                & \textgreater 400 & 0\%            & 0\%            & 0\%            & 0\%            & 0\%            \\ \hline
\multirow{3}{*}{E\_L}           & Média            & 163               & 222               & 400               & 90                & 232               \\
                                & \textless 400    & 91,58\%           & 66,34\%           & 0\%            & 98,51\%           & 71,29\%           \\
                                & \textgreater 400 & 3,47\%            & 0\%            & 0\%            & 0\%            & 4,46\%            \\ \hline
\multirow{3}{*}{E\_M}           & Média            & 143               & 196               & 400               & 90                & 209               \\
                                & \textless 400    & 98,48\%           & 77,78\%           & 0\%            & 100\%          & 83,33\%           \\
                                & \textgreater 400 & 1,01\%            & 0\%            & 0\%            & 0\%            & 4,55\%            \\ \hline
\multirow{3}{*}{E\_F}           & Média            & 176               & 198               & 400               & 91                & 250               \\
                                & \textless 400    & 93,43\%           & 77,27\%           & 0\%            & 100\%          & 61,62\%           \\
                                & \textgreater 400 & 2,02\%            & 0\%            & 0\%            & 0\%            & 2,53\%            \\ \hline
\multirow{3}{*}{D\_L}           & Média            & 174               & 229               & 400               & 84                & 238               \\
                                & \textless 400    & 89,95\%           & 62,81\%           & 0\%            & 100\%          & 66,33\%           \\
                                & \textgreater 400 & 2,51\%            & 0\%            & 0\%            & 0\%            & 4,02\%            \\ \hline
\multirow{3}{*}{D\_M}           & Média            & 180               & 332               & 400               & 99                & 369               \\
                                & \textless 400    & 94,95\%           & 27,27\%           & 0\%            & 100\%          & 21,21\%           \\
                                & \textgreater 400 & 1,52\%            & 0\%            & 0\%            & 0\%            & 3,03\%            \\ \hline
\multirow{3}{*}{D\_F}           & Média            & 292               & 386               & 400               & 132               & 380               \\
                                & \textless 400    & 50,25\%           & 6,03\%            & 0\%            & 99,50\%           & 14,57\%           \\
                                & \textgreater 400 & 5,53\%            & 0\%            & 0\%            & 0,50\%            & 2,01\%            \\ \hline
\multirow{3}{*}{FS}             & Média            & 192               & 232               & 400               & 90                & 253               \\
                                & \textless 400    & 77,89\%           & 60,80\%           & 0\%            & 99,50\%           & 57,29\%           \\
                                & \textgreater 400 & 0,50\%            & 0\%            & 0\%            & 0,50\%            & 0,50\%            \\ \hline
\multirow{3}{*}{Fre}            & Média            & 181               & 252               & 400               & 86                & 274               \\
                                & \textless 400    & 81,82\%           & 53,03\%           & 0\%            & 100\%          & 56,06\%           \\
                                & \textgreater 400 & 2,02\%            & 0\%            & 0\%            & 0\%            & 2,53\%            \\ \hline
\multirow{4}{*}{\textbf{TOTAL}} & Média            & 233               & 287               & 400               & 160               & 302               \\
                                & Desvio Padrão    & 132               & 134               & 0                 & 128               & 128               \\
                                & \textless 400    & 66,80\%           & 42,45\%           & 0\%            & 78,62\%           & 42,50\%           \\
                                & \textgreater 400 & 1,83\%            & 0\%            & 0\%            & 0,10\%            & 2,33\%            \\ \hline
\end{tabular}
\end{table}


Enquanto era realizado o teste, notou-se que os fios oscilavam em decorrência da vibração dos motores, o que levou a cogitar a hipótese de que essa 
seria de fato a natureza da interferência entre os sensores ultrassônicos e os motores, i.e. as vibrações mecânicas dos BLDC ressoavam nos fios 
provocando alguma espécie de mal contato, seja pela solda mal feita ou fios partidos.
Diante desse cenário, foi realizado um novo teste nas mesmas condições em que desparafusou-se o para-choque no qual os sensores estão fixos, das 
hastes metálicas e foram colhidos dados  com ele apoiado no chão. Nestas circunstâncias houve redução considerável do ruído que, de um modo geral, se 
restringiu a um único sonar, conforme a Tabela \ref{teste_2}.


% Please add the following required packages to your document preamble:
% \usepackage{multirow}
\begin{table}[]
\centering
\caption{My caption}
\label{my-label}
\begin{tabular}{|c|c|ccccc|}
\hline
\textbf{Motores}                & \textbf{Medida}  & \textbf{Sensor 0} & \textbf{Sensor 1} & \textbf{Sensor 2} & \textbf{Sensor 3} & \textbf{Sensor 4} \\ \hline
\multirow{3}{*}{STP}            & Média            & 400               & 373               & 400               & 400               & 400               \\
                                & \textless 400    & 0\%            & 11,09\%           & 0\%            & 0\%            & 0\%            \\
                                & \textgreater 400 & 0\%            & 0\%            & 0\%            & 0\%            & 0\%            \\ \hline
\multirow{3}{*}{E\_L}           & Média            & 400               & 141               & 400               & 400               & 400               \\
                                & \textless 400    & 0\%            & 89,55\%           & 0\%            & 0\%            & 0\%            \\
                                & \textgreater 400 & 0\%            & 0\%            & 0\%            & 0\%            & 0\%            \\ \hline
\multirow{3}{*}{E\_M}           & Média            & 400               & 121               & 400               & 400               & 400               \\
                                & \textless 400    & 0\%            & 96,97\%           & 0\%            & 0\%            & 0\%            \\
                                & \textgreater 400 & 0\%            & 0\%            & 0\%            & 0\%            & 0\%            \\ \hline
\multirow{3}{*}{E\_F}           & Média            & 400               & 116               & 400               & 400               & 400               \\
                                & \textless 400    & 0\%            & 98,48\%           & 0\%            & 0\%            & 0\%            \\
                                & \textgreater 400 & 0\%            & 0\%            & 0\%            & 0\%            & 0\%            \\ \hline
\multirow{3}{*}{D\_L}           & Média            & 400               & 207               & 400               & 400               & 400               \\
                                & \textless 400    & 0\%            & 67,84\%           & 0\%            & 0\%            & 0\%            \\
                                & \textgreater 400 & 0\%            & 0\%            & 0\%            & 0\%            & 0\%            \\ \hline
\multirow{3}{*}{D\_M}           & Média            & 400               & 148               & 400               & 400               & 400               \\
                                & \textless 400    & 0\%            & 87,94\%           & 0\%            & 0\%            & 0\%            \\
                                & \textgreater 400 & 0\%            & 0\%            & 0\%            & 0\%            & 0\%            \\ \hline
\multirow{3}{*}{D\_F}           & Média            & 400               & 242               & 400               & 400               & 400               \\
                                & \textless 400    & 0\%            & 56,57\%           & 0\%            & 0\%            & 0\%            \\
                                & \textgreater 400 & 0\%            & 0\%            & 0\%            & 0\%            & 0\%            \\ \hline
\multirow{3}{*}{FS}             & Média            & 400               & 218               & 400               & 400               & 400               \\
                                & \textless 400    & 0\%            & 64,65\%           & 0\%            & 0\%            & 0\%            \\
                                & \textgreater 400 & 0\%            & 0\%            & 0\%            & 0\%            & 0\%            \\ \hline
\multirow{3}{*}{Fre}            & Média            & 400               & 123               & 400               & 400               & 400               \\
                                & \textless 400    & 0\%            & 96,48\%           & 0\%            & 0\%            & 0\%            \\
                                & \textgreater 400 & 0\%            & 0\%            & 0\%            & 0\%            & 0\%            \\ \hline
\multirow{4}{*}{\textbf{TOTAL}} & Média            & 400               & 210               & 400               & 400               & 400               \\
                                & Desvio Padrão    & 0                 & 135               & 0                 & 0                 & 0                 \\
                                & \textless 400    & 0\%            & 66,83\%           & 0\%            & 0\%            & 0\%            \\
                                & \textgreater 400 & 0\%            & 0\%            & 0\%            & 0\%            & 0\%            \\ \hline
\end{tabular}
\end{table}
