%% USPSC-Anexos.tex
% ---
% Inicia os anexos
% ---
\begin{anexosenv}

% Imprime uma página indicando o início dos anexos
\partanexos

% ---
\chapter{Exemplo de anexo}
% ---
Elemento opcional, que consiste em um texto ou documento não elaborado pelo autor, que serve de fundamentação, comprovação e ilustração, conforme a ABNT NBR 14724. \cite{nbr14724}.

O \textbf{ANEXO B} exemplifica como incluir um anexo em pdf.

\chapter{Acentuação (modo texto - \LaTeX)}
\begin{figure}[H]
	\begin{center}
	\caption{\label{fig_anexob}Acentuação (modo texto - \LaTeX)}
	\includegraphics[scale=1.0]{USPSC-AcentuacaoLaTeX.png} \\
	Fonte: \citeonline{comandos}
	\end{center}	
\end{figure}

\chapter{Símbolos úteis em \LaTeX}
\begin{figure}[H]
	\begin{center}
		\caption{\label{fig_anexoc}Símbolos úteis em \LaTeX}
		\includegraphics[scale=1.0]{USPSC-SimbolosUteis.png} \\
		Fonte: \citeonline{comandos}
	\end{center}	
\end{figure}


\chapter{Letras gregas em \LaTeX}
\begin{figure}[H]
	\begin{center}
		\caption{\label{fig_anexod}Letras gregas em \LaTeX}
		\includegraphics[scale=1.0]{USPSC-LetrasGregas.png} \\
		Fonte: \citeonline{comandos}
	\end{center}	
\end{figure}

\end{anexosenv}
