%% Abstract.tex
% ---
% Abstract
% ---
\autor{Gordo, J.V.B.}
 \begin{resumo}[Abstract]
  \begin{otherlanguage*}{english}
 A high speed autonomous mobile robot with a purely reactive obstacle avoidance technique was implemented. Its purpose is to serve as a navigation 
test platform for UVAs in MOSA architecture which decouples the critical from the non-critical portion in the embedded system at real time. The 
vehicle’s perception is composed of an array of five ultrasonic sensors, the locomotion subsystem operates by front-wheel drive made through brushless 
DC motors, the communication subsystem is made through radio and its functions are: obtaining remote internal data during the autonomous navigation as 
well as a command interface which could also enable changes on the internal variables of the vehicle without reprograming the microcontroller and, 
most important, endorse the vehicle’s navigation based on the status of a security button attached to another radio. The navigation strategy consists 
in classifying the readings from each of the sonars in three regions: distant, warning and close. Based on the region where the sonar is located, a 
decision is made on which evasion measure is to be carried out, i.e., in the microcontroller’s memory was recorded a table that correlates each 
combinations of regions to an ordered pair of speed which has to be imposed to the motors so they are able to deviate from the obstacle. However, the 
readings obtained by the sonars were strongly affected by the mechanical vibrations from the motors, causing spurious readings which led the vehicle 
to adopt wrong behaviors.

    \vspace{\onelineskip}
  
    \noindent 
    \textbf{Key-words}: autonomous navigation. reactive systems. ultrasonic sensors. obstacle avoidance
  \end{otherlanguage*}
 \end{resumo}
