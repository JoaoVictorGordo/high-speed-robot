
% resumo em português
\setlength{\absparsep}{18pt} % ajusta o espaçamento dos parágrafos do resumo
\begin{resumo}
Foi implementado um veículo de navegação autônoma de alta velocidade, cuja técnica de desvio de obstáculos é puramente reativa, com o propósito de 
servir como uma plataforma de testes de navegação de VANTs desenvolvidos na arquitetura MOSA, que desacopla a porção crítica da não crítica do 
sistema embarcado de tempo real.
A percepção do veículo consiste em uma matriz de cinco sensores ultrassônicos, o subsistema de locomoção funciona por tração dianteira feita por 
motores de corrente contínua sem escovas, o subsistema de comunicação é feito via rádio com as funcionalidades de: obtenção remota de dados 
internos do veículo durante a navegação autônoma, de uma interface de comando que também possibilitasse a mudança de variáveis internas do veículo sem 
a necessidade de reprogramar o microcontrolador e, o mais importante, de dar ou não ao veículo o aval para navegar com base no \textit{status} de um 
botão de segurança acoplado a outro rádio.
A estratégia de navegação consiste em classificar as leituras de cada um dos sonares em três regiões: distante, atenção e próxima; com base em qual 
região encontra-se cada sonar, toma-se a decisão de qual a medida de evasão a ser realizada, isto é, foi gravada na memória do microcontrolador uma 
tabela que correlaciona cada uma das combinações de regiões a um par ordenado de velocidades que deve ser imposta aos motores para desviar do 
obstáculo em questão.
Contudo, as leituras obtidas pelos sonares foram fortemente afetadas em decorrência de vibrações mecânicas dos motores, que causam leituras 
espúrias, o que, por sua vez, provoca a adoção de comportamentos errados pelo veículo.

 \textbf{Palavras-chaves}: navegação autônoma. sistemas reativos. sensores ultrassônicos. desvio de obstáculos.
\end{resumo}
