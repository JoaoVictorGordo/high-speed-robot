\chapter{Apêndices}

 \section{Diagrama da Classe RF24}
  \begin{figure}[H]
    \centering
    \includegraphics[width=\linewidth]{../../Imagens/RF24_class.png}
    \caption{Diagrama da Classe RF24} %% TODO certificar se realmente é uma WBS!!!
    \label{RF24_ClassDiag}
  \end{figure}

 \section{Estrutura Analítica do Projeto}
  \begin{figure}[H] %% TODO certificar se realmente é uma WBS!!!
    \centering
    \includegraphics[width=\linewidth]{../../Imagens/WBS.png}
    \caption{Estrutura Analítica do Projeto} %% TODO certificar se realmente é uma WBS!!!
    \label{WBS}
  \end{figure}
  
 \section{Esquemático do Robô}
  \begin{figure}[H]
    \centering
    \includegraphics[width=\linewidth]{../../Imagens/robot_schem.png}
    \caption{Diagrama Elétrico} %% TODO: é um bom nome??
    \label{fritzing}
  \end{figure}
  
 \section{Testes}
 \subsection{Leituras Espúrias}
% Please add the following required packages to your document preamble:
% \usepackage{multirow}
\begin{table}[H]
\centering
\caption{Teste sem Obstáculos ao Alcance}
\label{vazio}
\begin{tabular}{|c|c|c|ccccc|}\hline
Dados & Contagem & Intervalo & Sensor 0 & Sensor 1 & Sensor 2 & Sensor 3 & Sensor 4 \\ \hline
\multirow{6}{*}{crus} & \multirow{3}{*}{$\neq 400$} & 30ms & 0\% & 1,22\% & 0\% & 1,48\% & 0\% \\
 &  & 45ms & 0,67\% & 0,92\% & 0\% & 1,58\% & 0\% \\
 &  & 60ms & 3,17\% & 0,58\% & 0\% & 2,67\% & 0\% \\ \cline{2-8} 
 & \multirow{3}{*}{\textless \hspace{0,5mm} 10} & 30ms & 0\% & 0\% & 0\% & 0,78\% & 0\% \\
 &  & 45ms & 0,67\% & 0\% & 0\% & 0,83\% & 0\% \\
 &  & 60ms & 3,17\% & 0\% & 0\% & 1,08\% & 0\% \\ \hline
\multirow{6}{*}{processados} & \multirow{3}{*}{$\neq 400$} & 30ms & 0\% & 0,70\% & 0\% & 0,35\% & 0\% \\
 &  & 45ms & 0\% & 0,25\% & 0\% & 0,33\% & 0\% \\
 &  & 60ms & 0\% & 0,08\% & 0\% & 1,09\% & 0\% \\ \cline{2-8} 
 & \multirow{3}{*}{\textless \hspace{0,5mm} 10} & 30ms & 0\% & 0\% & 0\% & 0\% & 0\% \\
 &  & 45ms & 0\% & 0\% & 0\% & 0\% & 0\% \\
 &  & 60ms & 0\% & 0\% & 0\% & 0\% & 0\% \\ \hline
\end{tabular}
\end{table}

  \pagebreak

 \subsection{Leituras Dinâmicas x Estáticas}
 % Please add the following required packages to your document preamble:
% \usepackage{multirow}
\begin{center}
\begin{longtable}{|c|c|c|ccccc|}
\label{30cm} \\
\caption[30cm]{30cm} \endfirsthead \\
\caption[]{30cm} \\
\hline \multicolumn{3}{|c|}{{Continua na próxima página.}} \\ \hline
\endfoot \hline
\hline \hline
\endlastfoot
\hline
\textbf{Medida} & \textbf{Intervalo} & \textbf{Leitura} & \textbf{Sensor 0} & \textbf{Sensor 1} & \textbf{Sensor 2} & \textbf{Sensor 3} & \textbf{Sensor 4} \\ \hline
\multirow{12}{*}{\begin{tabular}[c]{@{}c@{}}Média\\ {[}cm{]}\end{tabular}} & \multirow{4}{*}{30ms} & Est. & 26,43 & 30,13 & 31,14 & 35,62 & 37,19 \\
 &  & Din. & 26,58 & 30,57 & 30,55 & 35,67 & 37,51 \\
 &  & Est. & 26,37 & 30,12 & 30,99 & 35,66 & 37,21 \\
 &  & Din. & 26,42 & 30,60 & 32,17 & 35,61 & 37,52 \\ \cline{2-8} 
 & \multirow{4}{*}{45ms} & Est. & 26,36 & 30,51 & 32,61 & 35,67 & 37,34 \\
 &  & Din. & 26,39 & 30,86 & 32,31 & 35,76 & 37,73 \\
 &  & Est. & 26,45 & 30,26 & 31,59 & 35,75 & 37,17 \\
 &  & Din. & 26,42 & 30,29 & 29,11 & 35,65 & 37,36 \\ \cline{2-8} 
 & \multirow{4}{*}{60ms} & Est. & 26,41 & 30,12 & 31,28 & 35,67 & 37,21 \\
 &  & Din. & 26,72 & 30,48 & 31,11 & 35,71 & 37,65 \\
 &  & Est. & 26,46 & 30,01 & 30,38 & 35,61 & 37,15 \\
 &  & Din. & 26,52 & 30,35 & 28,47 & 35,50 & 37,33 \\ \hline
\multirow{12}{*}{\begin{tabular}[c]{@{}c@{}}Média\\ Limpa\\ {[}cm{]}\end{tabular}} & \multirow{4}{*}{30ms} & Est. & 26,43 & 30,13 & 31,14 & 35,62 & 37,19 \\
 &  & Din. & 26,58 & 30,57 & 31,16 & 35,67 & 37,51 \\
 &  & Est. & 26,37 & 30,12 & 30,99 & 35,66 & 37,21 \\
 &  & Din. & 26,42 & 30,60 & 32,69 & 35,61 & 37,52 \\ \cline{2-8} 
 & \multirow{4}{*}{45ms} & Est. & 26,36 & 30,51 & 32,61 & 35,67 & 37,34 \\
 &  & Din. & 26,39 & 30,86 & 32,82 & 35,76 & 37,73 \\
 &  & Est. & 26,45 & 30,26 & 31,59 & 35,75 & 37,17 \\
 &  & Din. & 26,42 & 30,29 & 31,91 & 35,65 & 37,36 \\ \cline{2-8} 
 & \multirow{4}{*}{60ms} & Est. & 26,41 & 30,12 & 31,28 & 35,67 & 37,21 \\
 &  & Din. & 26,72 & 30,48 & 31,59 & 35,71 & 37,65 \\
 &  & Est. & 26,46 & 30,01 & 30,38 & 35,61 & 37,15 \\
 &  & Din. & 26,52 & 30,35 & 30,84 & 35,50 & 37,33 \\ \hline \pagebreak
\multirow{12}{*}{\begin{tabular}[c]{@{}c@{}}Desvio\\ Padrão\\ {[}cm{]}\end{tabular}} & \multirow{4}{*}{30ms} & Est. & 0,50 & 0,34 & 1,51 & 0,49 & 0,41 \\
 &  & Din. & 0,50 & 0,50 & 4,19 & 0,55 & 0,53 \\
 &  & Est. & 0,48 & 0,33 & 1,19 & 0,48 & 0,43 \\
 &  & Din. & 0,50 & 0,49 & 4,35 & 0,59 & 0,55 \\ \cline{2-8} 
 & \multirow{4}{*}{45ms} & Est. & 0,48 & 0,50 & 1,68 & 0,47 & 0,52 \\
 &  & Din. & 0,49 & 0,35 & 4,12 & 0,54 & 0,62 \\
 &  & Est. & 0,50 & 0,44 & 1,63 & 0,44 & 0,41 \\
 &  & Din. & 0,50 & 0,46 & 8,18 & 0,48 & 0,63 \\ \cline{2-8} 
 & \multirow{4}{*}{60ms} & Est. & 0,49 & 0,33 & 1,42 & 0,47 & 0,41 \\
 &  & Din. & 0,45 & 0,50 & 4,10 & 0,50 & 0,69 \\
 &  & Est. & 0,50 & 0,12 & 0,70 & 0,49 & 0,38 \\
 &  & Din. & 0,50 & 0,48 & 7,47 & 0,50 & 0,49 \\ \hline
\multirow{12}{*}{\textless avg - 15} & \multirow{4}{*}{30ms} & Est. & 0,00\% & 0\% & 0\% & 0\% & 0\% \\
 &  & Din. & 0\% & 0\% & 2,67\% & 0\% & 0\% \\
 &  & Est. & 0\% & 0\% & 0\% & 0\% & 0\% \\
 &  & Din. & 0\% & 0\% & 2,00\% & 0\% & 0\% \\ \cline{2-8} 
 & \multirow{4}{*}{45ms} & Est. & 0\% & 0\% & 0\% & 0\% & 0\% \\
 &  & Din. & 0\% & 0\% & 2,00\% & 0\% & 0\% \\
 &  & Est. & 0\% & 0\% & 0\% & 0\% & 0\% \\
 &  & Din. & 0\% & 0\% & 11,33\% & 0\% & 0\% \\ \cline{2-8} 
 & \multirow{4}{*}{60ms} & Est. & 0\% & 0\% & 0\% & 0\% & 0\% \\
 &  & Din. & 0\% & 0\% & 2,00\% & 0\% & 0\% \\
 &  & Est. & 0\% & 0\% & 0\% & 0\% & 0\% \\
 &  & Din. & 0\% & 0\% & 10,00\% & 0\% & 0\% \\ \hline
\multirow{12}{*}{\textgreater avg + 15} & \multirow{4}{*}{30ms} & Est. & 0\% & 0\% & 1,33\% & 0\% & 0\% \\
 &  & Din. & 0\% & 0\% & 4,00\% & 0\% & 0\% \\
 &  & Est. & 0\% & 0\% & 0,67\% & 0\% & 0\% \\
 &  & Din. & 0\% & 0\% & 4,67\% & 0\% & 0\% \\ \cline{2-8} 
 & \multirow{4}{*}{45ms} & Est. & 0\% & 0\% & 0\% & 0\% & 0\% \\
 &  & Din. & 0\% & 0\% & 1,33\% & 0\% & 0\% \\
 &  & Est. & 0\% & 0\% & 0\% & 0\% & 0\% \\
 &  & Din. & 0\% & 0\% & 6,67\% & 0\% & 0\% \\ \cline{2-8} 
 & \multirow{4}{*}{60ms} & Est. & 0\% & 0\% & 0\% & 0\% & 0\% \\
 &  & Din. & 0\% & 0\% & 7,33\% & 0\% & 0\% \\
 &  & Est. & 0\% & 0\% & 0\% & 0\% & 0\% \\
 &  & Din. & 0\% & 0\% & 6,67\% & 0\% & 0\% \\ \hline
\multirow{12}{*}{= 400} & \multirow{4}{*}{30ms} & Est. & 0\% & 0\% & 0\% & 0\% & 0\% \\
 &  & Din. & 0\% & 0\% & 0\% & 0\% & 0\% \\
 &  & Est. & 0\% & 0\% & 0\% & 0\% & 0\% \\
 &  & Din. & 0\% & 0\% & 0\% & 0\% & 0\% \\ \cline{2-8} 
 & \multirow{4}{*}{45ms} & Est. & 0\% & 0\% & 0\% & 0\% & 0\% \\
 &  & Din. & 0\% & 0\% & 0\% & 0\% & 0\% \\
 &  & Est. & 0\% & 0\% & 0\% & 0\% & 0\% \\
 &  & Din. & 0\% & 0\% & 0\% & 0\% & 0\% \\ \cline{2-8} 
 & \multirow{4}{*}{60ms} & Est. & 0\% & 0\% & 0\% & 0\% & 0\% \\
 &  & Din. & 0\% & 0\% & 0\% & 0\% & 0\% \\
 &  & Est. & 0\% & 0\% & 0\% & 0\% & 0\% \\
 &  & Din. & 0\% & 0\% & 0\% & 0\% & 0\% \\ \hline
\end{longtable}
\end{center}

  % Please add the following required packages to your document preamble:
% \usepackage{multirow}
\begin{center}
\begin{longtable}{|c|c|c|ccccc|}
\label{50cm} \\
\caption[50cm]{50cm} \endfirsthead \\
\caption[]{50cm} \\
\hline \multicolumn{3}{|c|}{{Continua na próxima página.}} \\ \hline
\endfoot \hline
\hline \hline
\endlastfoot
\hline
\textbf{Medida} & \textbf{Intervalo} & \textbf{Leitura} & \textbf{Sensor 0} & \textbf{Sensor 1} & \textbf{Sensor 2} & \textbf{Sensor 3} & \textbf{Sensor 4} \\ \hline
\multirow{12}{*}{\begin{tabular}[c]{@{}c@{}}Média\\ {[}cm{]}\end{tabular}} & \multirow{4}{*}{30ms} & Est. & 44,80 & 47,77 & 50,86 & 48,39 & 53,25 \\
 &  & Din. & 44,57 & 47,87 & 50,13 & 48,59 & 53,65 \\
 &  & Est. & 45,21 & 47,91 & 50,86 & 48,35 & 52,49 \\
 &  & Din. & 44,95 & 47,83 & 50,65 & 48,77 & 52,65 \\ \cline{2-8} 
 & \multirow{4}{*}{45ms} & Est. & 45,35 & 47,46 & 50,45 & 48,53 & 53,61 \\
 &  & Din. & 44,85 & 47,85 & 52,61 & 48,92 & 63,21 \\
 &  & Est. & 44,70 & 47,72 & 51,35 & 48,56 & 53,06 \\
 &  & Din. & 44,79 & 47,91 & 50,87 & 48,67 & 60,42 \\ \cline{2-8} 
 & \multirow{4}{*}{60ms} & Est. & 45,86 & 47,76 & 50,81 & 48,49 & 51,69 \\
 &  & Din. & 45,27 & 47,85 & 50,74 & 48,94 & 52,55 \\
 &  & Est. & 45,68 & 47,89 & 51,54 & 48,61 & 53,26 \\
 &  & Din. & 44,78 & 47,63 & 51,29 & 48,73 & 51,33 \\ \hline  
\multirow{12}{*}{\begin{tabular}[c]{@{}c@{}}Média\\ Limpa\\ {[}cm{]}\end{tabular}} & \multirow{4}{*}{30ms} & Est. & 44,80 & 47,77 & 50,86 & 48,39 & 53,25 \\
 &  & Din. & 44,57 & 47,87 & 50,13 & 48,59 & 53,65 \\
 &  & Est. & 45,21 & 47,91 & 50,86 & 48,35 & 52,49 \\
 &  & Din. & 44,95 & 47,83 & 50,65 & 48,77 & 52,65 \\ \cline{2-8} 
 & \multirow{4}{*}{45ms} & Est. & 45,35 & 47,46 & 50,45 & 48,53 & 53,61 \\
 &  & Din. & 44,85 & 47,85 & 52,61 & 48,92 & 63,21 \\
 &  & Est. & 44,70 & 47,72 & 51,35 & 48,56 & 53,06 \\
 &  & Din. & 44,79 & 47,91 & 50,87 & 48,67 & 60,42 \\ \cline{2-8} 
 & \multirow{4}{*}{60ms} & Est. & 45,86 & 47,76 & 50,81 & 48,49 & 51,69 \\
 &  & Din. & 45,27 & 47,85 & 50,74 & 48,94 & 52,55 \\
 &  & Est. & 45,68 & 47,89 & 51,54 & 48,61 & 53,26 \\
 &  & Din. & 44,78 & 47,63 & 51,29 & 48,73 & 51,33 \\ \hline
\multirow{12}{*}{\begin{tabular}[c]{@{}c@{}}Desvio\\ Padrão\\ {[}cm{]}\end{tabular}} & \multirow{4}{*}{30ms} & Est. & 0,65 & 0,42 & 0,74 & 0,50 & 11,99 \\
 &  & Din. & 0,56 & 0,33 & 1,48 & 0,53 & 12,93 \\
 &  & Est. & 2,96 & 0,29 & 0,74 & 0,48 & 11,24 \\
 &  & Din. & 1,00 & 0,37 & 1,17 & 0,49 & 11,11 \\ \cline{2-8} 
 & \multirow{4}{*}{45ms} & Est. & 3,38 & 0,50 & 0,97 & 0,51 & 13,46 \\
 &  & Din. & 0,63 & 0,36 & 1,16 & 0,54 & 28,66 \\
 &  & Est. & 0,76 & 0,45 & 0,81 & 0,50 & 12,03 \\
 &  & Din. & 0,53 & 0,28 & 0,70 & 0,51 & 26,26 \\ \cline{2-8} 
 & \multirow{4}{*}{60ms} & Est. & 7,40 & 0,43 & 0,87 & 0,50 & 7,44 \\
 &  & Din. & 2,98 & 0,35 & 0,81 & 0,48 & 8,89 \\
 &  & Est. & 6,25 & 0,31 & 0,81 & 0,50 & 11,92 \\
 &  & Din. & 0,53 & 0,48 & 1,30 & 0,49 & 5,41 \\ \hline
\multirow{12}{*}{\textless avg - 15} & \multirow{4}{*}{30ms} & Est. &0\% &0\% &0\% &0\% &0\% \\
 &  & Din. &0\% &0\% &0\% &0\% &0\% \\
 &  & Est. &0\% &0\% &0\% &0\% &0\% \\
 &  & Din. &0\% &0\% &0\% &0\% &0\% \\ \cline{2-8} 
 & \multirow{4}{*}{45ms} & Est. &0\% &0\% &0\% &0\% &0\% \\
 &  & Din. &0\% &0\% &0\% &0\% &0\% \\
 &  & Est. &0\% &0\% &0\% &0\% &0\% \\
 &  & Din. &0\% &0\% &0\% &0\% &0\% \\ \cline{2-8} 
 & \multirow{4}{*}{60ms} & Est. &0\% &0\% &0\% &0\% &0\% \\
 &  & Din. &0\% &0\% &0\% &0\% &0\% \\
 &  & Est. &0\% &0\% &0\% &0\% &0\% \\
 &  & Din. &0\% &0\% &0\% &0\% &0\% \\ \hline
\multirow{12}{*}{\textgreater avg + 15} & \multirow{4}{*}{30ms} & Est. &0\% &0\% &0\% &0\% & 4,00\% \\
 &  & Din. &0\% &0\% &0\% &0\% & 5,33\% \\
 &  & Est. & 0,67\% &0\% &0\% &0\% & 2,67\% \\
 &  & Din. &0\% &0\% &0\% &0\% & 2,67\% \\ \cline{2-8} 
 & \multirow{4}{*}{45ms} & Est. & 0,67\% &0\% &0\% &0\% & 4,00\% \\
 &  & Din. &0\% &0\% &0\% &0\% & 16,00\% \\
 &  & Est. &0\% &0\% &0\% &0\% & 3,33\% \\
 &  & Din. &0\% &0\% &0\% &0\% & 12,67\% \\ \cline{2-8} 
 & \multirow{4}{*}{60ms} & Est. & 1,33\% &0\% &0\% &0\% & 1,33\% \\
 &  & Din. & 0,67\% &0\% &0\% &0\% & 2,67\% \\
 &  & Est. & 1,33\% &0\% &0\% &0\% & 4,67\% \\
 &  & Din. &0\% &0\% &0\% &0\% & 0,67\% \\ \hline
\multirow{12}{*}{= 400} & \multirow{4}{*}{30ms} & Est. &0\% &0\% &0\% &0\% &0\% \\
 &  & Din. &0\% &0\% &0\% &0\% &0\% \\
 &  & Est. &0\% &0\% &0\% &0\% &0\% \\
 &  & Din. &0\% &0\% &0\% &0\% &0\% \\ \cline{2-8} 
 & \multirow{4}{*}{45ms} & Est. &0\% &0\% &0\% &0\% &0\% \\
 &  & Din. &0\% &0\% &0\% &0\% &0\% \\
 &  & Est. &0\% &0\% &0\% &0\% &0\% \\
 &  & Din. &0\% &0\% &0\% &0\% &0\% \\ \cline{2-8} 
 & \multirow{4}{*}{60ms} & Est. &0\% &0\% &0\% &0\% &0\% \\
 &  & Din. &0\% &0\% &0\% &0\% &0\% \\
 &  & Est. &0\% &0\% &0\% &0\% &0\% \\
 &  & Din. &0\% &0\% &0\% &0\% &0\% \\ \hline
\end{longtable}
\end{center}

\pagebreak
  % Please add the following required packages to your document preamble:
% \usepackage{multirow}
\begin{center}
\begin{longtable}{|c|c|c|ccccc|}
\label{100cm} \\
\caption[100cm]{100cm} \endfirsthead \\
\caption[]{100cm} \\
\hline \multicolumn{3}{|c|}{{Continua na próxima página.}} \\ \hline
\endfoot \hline
\hline \hline
\endlastfoot
\hline
\textbf{Medida} & \textbf{Intervalo} & \textbf{Leitura} & \textbf{Sensor 0} & \textbf{Sensor 1} & \textbf{Sensor 2} & \textbf{Sensor 3} & \textbf{Sensor 4} \\ \hline
\multirow{12}{*}{\begin{tabular}[c]{@{}c@{}}Média\\ {[}cm{]}\end{tabular}} & \multirow{4}{*}{30ms} & Est. & 102,49 & 106,83 & 96,83 & 92,05 & 94,19 \\
 &  & Din. & 102,71 & 107,05 & 97,07 & 92,46 & 94,74 \\
 &  & Est. & 102,41 & 106,85 & 96,83 & 92,21 & 94,49 \\
 &  & Din. & 102,46 & 106,91 & 97,19 & 92,54 & 94,54 \\ \cline{2-8} 
 & \multirow{4}{*}{45ms} & Est. & 102,78 & 106,86 & 96,96 & 92,05 & 94,29 \\
 &  & Din. & 102,31 & 106,82 & 97,10 & 92,37 & 94,49 \\
 &  & Est. & 102,41 & 106,89 & 97,07 & 92,20 & 94,27 \\
 &  & Din. & 102,37 & 106,83 & 97,07 & 92,23 & 94,20 \\ \cline{2-8} 
 & \multirow{4}{*}{60ms} & Est. & 102,63 & 106,96 & 97,46 & 92,14 & 94,24 \\
 &  & Din. & 102,34 & 106,81 & 97,23 & 92,46 & 94,53 \\
 &  & Est. & 102,59 & 106,90 & 97,07 & 92,43 & 94,60 \\
 &  & Din. & 102,07 & 106,77 & 97,18 & 92,51 & 94,65 \\ \hline
\multirow{12}{*}{\begin{tabular}[c]{@{}c@{}}Média\\ Limpa\\ {[}cm{]}\end{tabular}} & \multirow{4}{*}{30ms} & Est. & 102,49 & 106,83 & 96,83 & 92,05 & 94,19 \\
 &  & Din. & 102,71 & 107,05 & 97,07 & 92,46 & 94,74 \\
 &  & Est. & 102,41 & 106,85 & 96,83 & 92,21 & 94,49 \\
 &  & Din. & 102,46 & 106,91 & 97,19 & 92,54 & 94,54 \\ \cline{2-8} 
 & \multirow{4}{*}{45ms} & Est. & 102,78 & 106,86 & 96,96 & 92,05 & 94,29 \\
 &  & Din. & 102,31 & 106,82 & 97,10 & 92,37 & 94,49 \\
 &  & Est. & 102,41 & 106,89 & 97,07 & 92,20 & 94,27 \\
 &  & Din. & 102,37 & 106,83 & 97,07 & 92,23 & 94,20 \\ \cline{2-8} 
 & \multirow{4}{*}{60ms} & Est. & 102,63 & 106,96 & 97,46 & 92,14 & 94,24 \\
 &  & Din. & 102,34 & 106,81 & 97,23 & 92,46 & 94,53 \\
 &  & Est. & 102,59 & 106,90 & 97,07 & 92,43 & 94,60 \\
 &  & Din. & 102,07 & 106,77 & 97,18 & 92,51 & 94,65 \\ \hline
\multirow{12}{*}{\begin{tabular}[c]{@{}c@{}}Desvio\\ Padrão\\ {[}cm{]}\end{tabular}} & \multirow{4}{*}{30ms} & Est. & 0,66 & 0,39 & 0,78 & 0,24 & 1,02 \\
 &  & Din. & 0,99 & 0,44 & 0,73 & 0,53 & 1,12 \\
 &  & Est. & 0,66 & 0,38 & 0,77 & 0,41 & 1,01 \\
 &  & Din. & 0,66 & 0,29 & 0,75 & 0,57 & 0,98 \\ \cline{2-8} 
 & \multirow{4}{*}{45ms} & Est. & 0,83 & 0,35 & 0,72 & 0,33 & 1,01 \\
 &  & Din. & 0,60 & 0,39 & 0,78 & 0,56 & 0,99 \\
 &  & Est. & 0,66 & 0,31 & 0,77 & 0,42 & 0,83 \\
 &  & Din. & 0,66 & 0,40 & 0,81 & 0,44 & 0,62 \\ \cline{2-8}  \pagebreak
 & \multirow{4}{*}{60ms} & Est. & 0,69 & 0,33 & 2,52 & 0,37 & 1,07 \\
 &  & Din. & 0,61 & 0,39 & 0,76 & 0,54 & 0,95 \\
 &  & Est. & 0,73 & 0,34 & 0,72 & 0,50 & 1,22 \\
 &  & Din. & 0,53 & 0,42 & 0,71 & 0,58 & 0,99 \\ \hline
\multirow{12}{*}{\textless avg - 15} & \multirow{4}{*}{30ms} & Est. & 0,00\% & 0,00\% & 0,00\% & 0,00\% & 0,00\% \\
 &  & Din. & 0,00\% & 0,00\% & 0,00\% & 0,00\% & 0,00\% \\
 &  & Est. & 0,00\% & 0,00\% & 0,00\% & 0,00\% & 0,00\% \\
 &  & Din. & 0,00\% & 0,00\% & 0,00\% & 0,00\% & 0,00\% \\ \cline{2-8} 
 & \multirow{4}{*}{45ms} & Est. & 0,00\% & 0,00\% & 0,00\% & 0,00\% & 0,00\% \\
 &  & Din. & 0,00\% & 0,00\% & 0,00\% & 0,00\% & 0,00\% \\
 &  & Est. & 0,00\% & 0,00\% & 0,00\% & 0,00\% & 0,00\% \\
 &  & Din. & 0,00\% & 0,00\% & 0,00\% & 0,00\% & 0,00\% \\ \cline{2-8} 
 & \multirow{4}{*}{60ms} & Est. & 0,00\% & 0,00\% & 0,00\% & 0,00\% & 0,00\% \\
 &  & Din. & 0,00\% & 0,00\% & 0,00\% & 0,00\% & 0,00\% \\
 &  & Est. & 0,00\% & 0,00\% & 0,00\% & 0,00\% & 0,00\% \\
 &  & Din. & 0,00\% & 0,00\% & 0,00\% & 0,00\% & 0,00\% \\ \hline
\multirow{12}{*}{\textgreater avg + 15} & \multirow{4}{*}{30ms} & Est. & 0,00\% & 0,00\% & 0,00\% & 0,00\% & 0,00\% \\
 &  & Din. & 0,00\% & 0,00\% & 0,00\% & 0,00\% & 0,00\% \\
 &  & Est. & 0,00\% & 0,00\% & 0,00\% & 0,00\% & 0,00\% \\
 &  & Din. & 0,00\% & 0,00\% & 0,00\% & 0,00\% & 0,00\% \\ \cline{2-8} 
 & \multirow{4}{*}{45ms} & Est. & 0,00\% & 0,00\% & 0,00\% & 0,00\% & 0,00\% \\
 &  & Din. & 0,00\% & 0,00\% & 0,00\% & 0,00\% & 0,00\% \\
 &  & Est. & 0,00\% & 0,00\% & 0,00\% & 0,00\% & 0,00\% \\
 &  & Din. & 0,00\% & 0,00\% & 0,00\% & 0,00\% & 0,00\% \\ \cline{2-8} 
 & \multirow{4}{*}{60ms} & Est. & 0,00\% & 0,00\% & 0,67\% & 0,00\% & 0,00\% \\
 &  & Din. & 0,00\% & 0,00\% & 0,00\% & 0,00\% & 0,00\% \\
 &  & Est. & 0,00\% & 0,00\% & 0,00\% & 0,00\% & 0,00\% \\
 &  & Din. & 0,00\% & 0,00\% & 0,00\% & 0,00\% & 0,00\% \\ \hline
\multirow{12}{*}{= 400} & \multirow{4}{*}{30ms} & Est. & 0,00\% & 0,00\% & 0,00\% & 0,00\% & 0,00\% \\
 &  & Din. & 0,00\% & 0,00\% & 0,00\% & 0,00\% & 0,00\% \\
 &  & Est. & 0,00\% & 0,00\% & 0,00\% & 0,00\% & 0,00\% \\
 &  & Din. & 0,00\% & 0,00\% & 0,00\% & 0,00\% & 0,00\% \\ \cline{2-8} 
 & \multirow{4}{*}{45ms} & Est. & 0,00\% & 0,00\% & 0,00\% & 0,00\% & 0,00\% \\
 &  & Din. & 0,00\% & 0,00\% & 0,00\% & 0,00\% & 0,00\% \\
 &  & Est. & 0,00\% & 0,00\% & 0,00\% & 0,00\% & 0,00\% \\
 &  & Din. & 0,00\% & 0,00\% & 0,00\% & 0,00\% & 0,00\% \\ \cline{2-8} 
 & \multirow{4}{*}{60ms} & Est. & 0,00\% & 0,00\% & 0,00\% & 0,00\% & 0,00\% \\
 &  & Din. & 0,00\% & 0,00\% & 0,00\% & 0,00\% & 0,00\% \\
 &  & Est. & 0,00\% & 0,00\% & 0,00\% & 0,00\% & 0,00\% \\
 &  & Din. & 0,00\% & 0,00\% & 0,00\% & 0,00\% & 0,00\% \\ \hline
\end{longtable}
\end{center}

  % Please add the following required packages to your document preamble:
% \usepackage{multirow}
\begin{center}
\begin{longtable}{|c|c|c|ccccc|}
\label{150cm} \\
\caption[150cm]{150cm} \endfirsthead \\
\caption[]{150cm} \\
\hline \multicolumn{3}{|c|}{{Continua na próxima página.}} \\ \hline
\endfoot \hline
\hline \hline
\endlastfoot
\hline
\textbf{Medida} & \textbf{Intervalo} & \textbf{Leitura} & \textbf{Sensor 0} & \textbf{Sensor 1} & \textbf{Sensor 2} & \textbf{Sensor 3} & \textbf{Sensor 4} \\ \hline
\multirow{12}{*}{\begin{tabular}[c]{@{}c@{}}Média\\ {[}cm{]}\end{tabular}} & \multirow{4}{*}{30ms} & Est. & 148,52 & 152,17 & 152,97 & 156,65 & 230,77 \\
 &  & Din. & 148,70 & 153,18 & 153,45 & 156,41 & 174,46 \\
 &  & Est. & 148,97 & 153,53 & 153,65 & 156,53 & 196,81 \\
 &  & Din. & 149,05 & 153,01 & 153,63 & 156,95 & 176,01 \\ \cline{2-8} 
 & \multirow{4}{*}{45ms} & Est. & 149,13 & 153,13 & 153,95 & 156,42 & 184,03 \\
 &  & Din. & 148,62 & 153,01 & 153,10 & 156,97 & 226,32 \\
 &  & Est. & 148,63 & 151,45 & 153,23 & 157,11 & 210,04 \\
 &  & Din. & 148,90 & 153,04 & 154,24 & 157,13 & 192,36 \\ \cline{2-8} 
 & \multirow{4}{*}{60ms} & Est. & 148,84 & 153,09 & 153,42 & 156,79 & 256,77 \\
 &  & Din. & 148,53 & 152,96 & 153,33 & 157,28 & 161,61 \\
 &  & Est. & 148,49 & 152,95 & 153,03 & 156,81 & 164,53 \\
 &  & Din. & 148,45 & 152,57 & 152,18 & 156,33 & 229,58 \\ \hline \multirow{12}{*}{\begin{tabular}[c]{@{}c@{}}Média\\ Limpa\\ {[}cm{]}\end{tabular}} & \multirow{4}{*}{30ms} & Est. & 148,52 & 152,17 & 152,97 & 156,65 & 158,25 \\
 &  & Din. & 148,70 & 153,18 & 153,45 & 156,41 & 158,35 \\
 &  & Est. & 148,97 & 153,53 & 153,65 & 156,53 & 158,10 \\
 &  & Din. & 149,05 & 153,01 & 153,63 & 156,95 & 158,28 \\ \cline{2-8} 
 & \multirow{4}{*}{45ms} & Est. & 149,13 & 153,13 & 153,95 & 156,42 & 158,25 \\
 &  & Din. & 148,62 & 153,01 & 153,10 & 156,97 & 158,78 \\
 &  & Est. & 148,63 & 151,45 & 153,23 & 157,11 & 158,53 \\
 &  & Din. & 148,90 & 153,04 & 154,24 & 157,13 & 158,56 \\ \cline{2-8} 
 & \multirow{4}{*}{60ms} & Est. & 148,84 & 153,09 & 153,42 & 156,79 & 158,60 \\
 &  & Din. & 148,53 & 152,96 & 153,33 & 157,28 & 158,39 \\
 &  & Est. & 148,49 & 152,95 & 153,03 & 156,81 & 158,08 \\
\multirow{12}{*}{\begin{tabular}[c]{@{}c@{}}Desvio\\ Padrão\\ {[}cm{]}\end{tabular}} & \multirow{4}{*}{30ms} & Est. & 0,60 & 7,48 & 6,44 & 0,76 & 111,16 \\
 &  & Din. & 0,63 & 0,39 & 0,64 & 0,88 & 60,49 \\
 &  & Est. & 0,52 & 0,50 & 0,65 & 0,68 & 88,98 \\
 &  & Din. & 0,47 & 0,08 & 0,61 & 0,68 & 63,23 \\ \cline{2-8} 
 & \multirow{4}{*}{45ms} & Est. & 0,42 & 0,33 & 0,73 & 0,74 & 74,88 \\
 &  & Din. & 0,63 & 0,23 & 0,86 & 0,82 & 108,67 \\
 &  & Est. & 0,61 & 9,46 & 0,89 & 0,64 & 99,26 \\
 &  & Din. & 0,47 & 0,30 & 6,01 & 0,72 & 84,06 \\ \cline{2-8} 
 & \multirow{4}{*}{60ms} & Est. & 0,57 & 0,28 & 0,63 & 0,57 & 118,98 \\
 &  & Din. & 0,62 & 0,20 & 0,62 & 0,65 & 27,81 \\
 &  & Est. & 0,54 & 0,21 & 0,68 & 0,61 & 39,11 \\
 &  & Din. & 0,59 & 4,90 & 5,68 & 0,89 & 110,18 \\ \hline
\multirow{12}{*}{\textless avg - 15} & \multirow{4}{*}{30ms} & Est. &0\% & 1,33\% & 0,67\% &0\% &0\% \\
 &  & Din. &0\% &0\% &0\% &0\% &0\% \\
 &  & Est. &0\% &0\% &0\% &0\% &0\% \\
 &  & Din. &0\% &0\% &0\% &0\% &0\% \\ \cline{2-8} 
 & \multirow{4}{*}{45ms} & Est. &0\% &0\% &0\% &0\% &0\% \\
 &  & Din. &0\% &0\% &0\% &0\% &0\% \\
 &  & Est. &0\% & 2,67\% &0\% &0\% &0\% \\
 &  & Din. &0\% &0\% &0\% &0\% &0\% \\ \cline{2-8} 
 & \multirow{4}{*}{60ms} & Est. &0\% &0\% &0\% &0\% &0\% \\
 &  & Din. &0\% &0\% &0\% &0\% &0\% \\
 &  & Est. &0\% &0\% &0\% &0\% &0\% \\
 &  & Din. &0\% & 0,67\% & 1,33\% &0\% &0\% \\ \hline
\multirow{12}{*}{\textgreater avg + 15} & \multirow{4}{*}{30ms} & Est. &0\% &0\% &0\% &0\% &0\% \\
 &  & Din. &0\% &0\% &0\% &0\% &0\% \\
 &  & Est. &0\% &0\% &0\% &0\% &0\% \\
 &  & Din. &0\% &0\% &0\% &0\% &0\% \\ \cline{2-8} 
 & \multirow{4}{*}{45ms} & Est. &0\% &0\% &0\% &0\% &0\% \\
 &  & Din. &0\% &0\% &0\% &0\% &0\% \\
 &  & Est. &0\% &0\% &0\% &0\% &0\% \\
 &  & Din. &0\% &0\% & 0,67\% &0\% &0\% \\ \cline{2-8} 
 & \multirow{4}{*}{60ms} & Est. &0\% &0\% &0\% &0\% &0\% \\
 &  & Din. &0\% &0\% &0\% &0\% &0\% \\
 &  & Est. &0\% &0\% &0\% &0\% &0\% \\
 &  & Din. &0\% &0\% &0\% &0\% &0\% \\ \hline
\multirow{12}{*}{= 400} & \multirow{4}{*}{30ms} & Est. &0\% &0\% &0\% &0\% & 30,00\% \\
 &  & Din. &0\% &0\% &0\% &0\% & 6,67\% \\
 &  & Est. &0\% &0\% &0\% &0\% & 16,00\% \\
 &  & Din. &0\% &0\% &0\% &0\% & 7,33\% \\ \cline{2-8} 
 & \multirow{4}{*}{45ms} & Est. &0\% &0\% &0\% &0\% & 10,67\% \\
 &  & Din. &0\% &0\% &0\% &0\% & 28,00\% \\
 &  & Est. &0\% &0\% &0\% &0\% & 21,33\% \\
 &  & Din. &0\% &0\% &0\% &0\% & 14,00\% \\ \cline{2-8} 
 & \multirow{4}{*}{60ms} & Est. &0\% &0\% &0\% &0\% & 40,67\% \\
 &  & Din. &0\% &0\% &0\% &0\% & 1,33\% \\
 &  & Est. &0\% &0\% &0\% &0\% & 2,67\% \\
 &  & Din. &0\% &0\% &0\% &0\% & 29,33\% \\ \hline
\end{longtable}
\end{center}


  % Please add the following required packages to your document preamble:
% \usepackage{multirow}
\begin{center}
\begin{longtable}{|c|c|c|ccccc|}
\label{200cm} \\
\caption[200cm]{200cm} \endfirsthead \\
\caption[]{200cm} \\
\hline \multicolumn{3}{|c|}{{Continua na próxima página.}} \\ \hline
\endfoot \hline
\hline \hline
\endlastfoot
\hline
\textbf{Medida} & \textbf{Intervalo} & \textbf{Leitura} & \textbf{Sensor 0} & \textbf{Sensor 1} & \textbf{Sensor 2} & \textbf{Sensor 3} & \textbf{Sensor 4} \\ \hline
\multirow{12}{*}{\begin{tabular}[c]{@{}c@{}}Média\\ {[}cm{]}\end{tabular}} & \multirow{4}{*}{30ms} & Est. & 197,58 & 189,40 & 192,26 & 182,45 & 182,72 \\
 &  & Din. & 209,51 & 189,56 & 213,76 & 182,13 & 182,67 \\
 &  & Est. & 216,61 & 189,42 & 206,54 & 182,01 & 182,48 \\
 &  & Din. & 191,33 & 189,25 & 202,63 & 182,43 & 182,72 \\ \cline{2-8} 
 & \multirow{4}{*}{45ms} & Est. & 196,53 & 189,63 & 220,25 & 182,33 & 182,75 \\
 &  & Din. & 239,21 & 189,51 & 208,81 & 182,87 & 183,20 \\
 &  & Est. & 201,98 & 189,55 & 209,55 & 182,76 & 183,12 \\
 &  & Din. & 242,49 & 188,22 & 192,27 & 182,97 & 183,50 \\ \cline{2-8} 
 & \multirow{4}{*}{60ms} & Est. & 199,45 & 189,49 & 189,49 & 182,46 & 186,07 \\
 &  & Din. & 197,21 & 189,61 & 212,97 & 182,71 & 183,24 \\
 &  & Est. & 237,66 & 189,77 & 196,97 & 182,60 & 182,85 \\
\multirow{12}{*}{\begin{tabular}[c]{@{}c@{}}Média\\ Limpa\\ {[}cm{]}\end{tabular}} & \multirow{4}{*}{30ms} & Est. & 194,84 & 189,40 & 192,26 & 182,45 & 182,72 \\
 &  & Din. & 200,18 & 189,56 & 211,24 & 182,13 & 182,67 \\
 &  & Est. & 194,72 & 189,42 & 206,54 & 182,01 & 182,48 \\
 &  & Din. & 191,33 & 189,25 & 202,63 & 182,43 & 182,72 \\ \cline{2-8} 
 & \multirow{4}{*}{45ms} & Est. & 196,53 & 189,63 & 220,25 & 182,33 & 182,75 \\
 &  & Din. & 197,32 & 189,51 & 208,81 & 182,87 & 183,20 \\
 &  & Est. & 193,73 & 189,55 & 209,55 & 182,76 & 183,12 \\
 &  & Din. & 198,06 & 188,22 & 192,27 & 182,97 & 183,50 \\ \cline{2-8} 
 & \multirow{4}{*}{60ms} & Est. & 191,09 & 189,49 & 189,49 & 182,46 & 183,18 \\
 &  & Din. & 191,65 & 189,61 & 212,97 & 182,71 & 183,24 \\
 &  & Est. & 190,08 & 189,77 & 196,97 & 182,60 & 182,85 \\
 &  & Din. & 189,75 & 188,77 & 192,45 & 182,79 & 182,97 \\ \hline
\multirow{12}{*}{\begin{tabular}[c]{@{}c@{}}Desvio\\ Padrão\\ {[}cm{]}\end{tabular}} & \multirow{4}{*}{30ms} & Est. & 25,47 & 0,51 & 16,06 & 0,56 & 0,72 \\
 &  & Din. & 43,75 & 0,50 & 42,00 & 0,65 & 0,71 \\
 &  & Est. & 64,48 & 0,51 & 32,18 & 0,74 & 0,66 \\
 &  & Din. & 5,85 & 0,43 & 28,28 & 0,82 & 0,77 \\ \cline{2-8} 
 & \multirow{4}{*}{45ms} & Est. & 10,08 & 0,49 & 39,51 & 0,48 & 0,69 \\
 &  & Din. & 82,97 & 0,53 & 34,85 & 0,75 & 0,93 \\
 &  & Est. & 41,49 & 0,53 & 33,54 & 0,65 & 0,90 \\
 &  & Din. & 84,75 & 9,48 & 14,72 & 0,79 & 1,24 \\ \cline{2-8} 
 & \multirow{4}{*}{60ms} & Est. & 41,69 & 0,53 & 6,76 & 0,54 & 24,97 \\
 &  & Din. & 33,94 & 0,50 & 35,49 & 0,64 & 1,14 \\
 &  & Est. & 88,21 & 0,42 & 22,80 & 0,67 & 0,83 \\
 &  & Din. & 41,38 & 7,62 & 13,07 & 0,70 & 0,90 \\ \hline
\multirow{12}{*}{\textless avg - 15} & \multirow{4}{*}{30ms} & Est. &0\% &0\% &0\% &0\% &0\% \\
 &  & Din. &0\% &0\% & 71,33\% &0\% &0\% \\
 &  & Est. &0\% &0\% & 54,00\% &0\% &0\% \\
 &  & Din. &0\% &0\% & 1,33\% &0\% &0\% \\ \cline{2-8} 
 & \multirow{4}{*}{45ms} & Est. &0\% &0\% & 62,00\% &0\% &0\% \\
 &  & Din. &0\% &0\% & 73,33\% &0\% &0\% \\
 &  & Est. &0\% &0\% & 75,33\% &0\% &0\% \\
 &  & Din. &0\% & 2,67\% &0\% &0\% &0\% \\ \cline{2-8} 
 & \multirow{4}{*}{60ms} & Est. &0\% &0\% &0\% &0\% &0\% \\
 &  & Din. &0\% &0\% & 72,00\% &0\% &0\% \\
 &  & Est. &0\% &0\% &0\% &0\% &0\% \\
 &  & Din. &0\% & 2,00\% &0\% &0\% &0\% \\ \hline
\multirow{12}{*}{\textgreater avg + 15} & \multirow{4}{*}{30ms} & Est. & 18,67\% &0\% & 4,00\% &0\% &0\% \\
 &  & Din. & 0,67\% &0\% & 26,67\% &0\% &0\% \\
 &  & Est. & 16,00\% &0\% & 20,00\% &0\% &0\% \\
 &  & Din. & 4,67\% &0\% & 14,67\% &0\% &0\% \\ \cline{2-8} 
 & \multirow{4}{*}{45ms} & Est. & 25,33\% &0\% & 38,00\% &0\% &0\% \\
 &  & Din. & 11,33\% &0\% & 24,00\% &0\% &0\% \\
 &  & Est. & 10,00\% &0\% & 22,67\% &0\% &0\% \\
 &  & Din. & 15,33\% &0\% & 3,33\% &0\% &0\% \\ \cline{2-8} 
 & \multirow{4}{*}{60ms} & Est. & 6,00\% &0\% & 0,67\% &0\% &0\% \\
 &  & Din. & 2,67\% &0\% & 27,33\% &0\% &0\% \\
 &  & Est. &0\% &0\% & 8,67\% &0\% &0\% \\
 &  & Din. &0\% &0\% & 2,67\% &0\% &0\% \\ \hline
\multirow{12}{*}{= 400} & \multirow{4}{*}{30ms} & Est. & 1,33\% &0\% &0\% &0\% &0\% \\
 &  & Din. & 4,67\% &0\% & 1,33\% &0\% &0\% \\
 &  & Est. & 10,67\% &0\% &0\% &0\% &0\% \\
 &  & Din. &0\% &0\% &0\% &0\% &0\% \\ \cline{2-8} 
 & \multirow{4}{*}{45ms} & Est. &0\% &0\% &0\% &0\% &0\% \\
 &  & Din. & 20,67\% &0\% &0\% &0\% &0\% \\
 &  & Est. & 4,00\% &0\% &0\% &0\% &0\% \\
 &  & Din. & 22,00\% &0\% &0\% &0\% &0\% \\ \cline{2-8} 
 & \multirow{4}{*}{60ms} & Est. & 4,00\% &0\% &0\% &0\% & 1,33\% \\
 &  & Din. & 2,67\% &0\% &0\% &0\% &0\% \\
 &  & Est. & 22,67\% &0\% &0\% &0\% &0\% \\
 &  & Din. & 4,00\% &0\% &0\% &0\% &0\% \\ \hline
\end{longtable}
\end{center}



% \subsection{Leituras Dinâmicas}
% \begin{table}[]
\centering
\caption{My caption}
\label{my-label}
\begin{tabular}{|c|c|ccccc|}
\hline
\textbf{Teste}            & \textbf{Medida}                                                            & \textbf{Sensor 0}         & \textbf{Sensor 1}         & \textbf{Sensor 2}         & \textbf{Sensor 3}         & \textbf{Sensor 4}          \\ \hline
1                         & Média                                                                      & 26,58                     & 30,57                     & 30,55                     & 35,67                     & 37,51                      \\
2                         & Média                                                                      & 26,42                     & 30,60                     & 32,17                     & 35,61                     & 37,52                      \\
3                         & Média                                                                      & 26,39                     & 30,86                     & 32,31                     & 35,76                     & 37,73                      \\
4                         & Média                                                                      & 26,42                     & 30,29                     & 29,11                     & 35,65                     & 37,36                      \\
5                         & Média                                                                      & 26,72                     & 30,48                     & 31,11                     & 35,71                     & 37,65                      \\
6                         & Média                                                                      & 26,52                     & 30,35                     & 28,47                     & 35,50                     & 37,33                      \\ \hline
1                         & Média Limpa                                                                & 26,58                     & 30,57                     & 30,55                     & 35,67                     & 37,51                      \\
2                         & Média Limpa                                                                & 26,42                     & 30,60                     & 32,17                     & 35,61                     & 37,52                      \\
3                         & Média Limpa                                                                & 26,39                     & 30,86                     & 32,31                     & 35,76                     & 37,73                      \\
4                         & Média Limpa                                                                & 26,42                     & 30,29                     & 29,11                     & 35,65                     & 37,36                      \\
5                         & Média Limpa                                                                & 26,72                     & 30,48                     & 31,11                     & 35,71                     & 37,65                      \\
6                         & Média Limpa                                                                & 26,52                     & 30,35                     & 28,47                     & 35,50                     & 37,33                      \\ \hline
1                         & Desvio Padrão                                                              & 0,50                      & 0,50                      & 4,19                      & 0,55                      & 0,53                       \\
2                         & Desvio Padrão                                                              & 0,50                      & 0,49                      & 4,35                      & 0,59                      & 0,55                       \\
3                         & Desvio Padrão                                                              & 0,49                      & 0,35                      & 4,12                      & 0,54                      & 0,62                       \\
4                         & Desvio Padrão                                                              & 0,50                      & 0,46                      & 8,18                      & 0,48                      & 0,63                       \\
5                         & Desvio Padrão                                                              & 0,45                      & 0,50                      & 4,10                      & 0,50                      & 0,69                       \\
6                         & Desvio Padrão                                                              & 0,50                      & 0,48                      & 7,47                      & 0,50                      & 0,49                       \\ \hline
1                         & Desvio Padrão Limpo                                                        & 0,50                      & 0,50                      & 4,19                      & 0,55                      & 0,53                       \\
2                         & Desvio Padrão Limpo                                                        & 0,50                      & 0,49                      & 4,35                      & 0,59                      & 0,55                       \\
3                         & Desvio Padrão Limpo                                                        & 0,49                      & 0,35                      & 4,12                      & 0,54                      & 0,62                       \\
4                         & Desvio Padrão Limpo                                                        & 0,50                      & 0,46                      & 8,18                      & 0,48                      & 0,63                       \\
5                         & Desvio Padrão Limpo                                                        & 0,45                      & 0,50                      & 4,10                      & 0,50                      & 0,69                       \\
6                         & Desvio Padrão Limpo                                                        & 0,50                      & 0,48                      & 7,47                      & 0,50                      & 0,49                       \\ \hline
1                         & \textless (Média)/2                                                        & 0\%                    & 0\%                    & 2,67\%                    & 0\%                    & 0\%                     \\
2                         & \textless (Média)/2                                                        & 0\%                    & 0\%                    & 2,00\%                    & 0\%                    & 0\%                     \\
3                         & \textless (Média)/2                                                        & 0\%                    & 0\%                    & 2,00\%                    & 0\%                    & 0\%                     \\
4                         & \textless (Média)/2                                                        & 0\%                    & 0\%                    & 11,33\%                   & 0\%                    & 0\%                     \\
5                         & \textless (Média)/2                                                        & 0\%                    & 0\%                    & 2,00\%                    & 0\%                    & 0\%                     \\
6                         & \textless (Média)/2                                                        & 0\%                    & 0\%                    & 10,00\%                   & 0\%                    & 0\%                     \\ \hline
1                         & Destoantes                                                                 & 0\%                    & 0\%                    & 0\%                    & 0\%                    & 0\%                     \\
2                         & Destoantes                                                                 & 0\%                    & 0\%                    & 0\%                    & 0\%                    & 0\%                     \\
3                         & Destoantes                                                                 & 0\%                    & 0\%                    & 0\%                    & 0\%                    & 0\%                     \\
4                         & Destoantes                                                                 & 0\%                    & 0\%                    & 0\%                    & 0\%                    & 0\%                     \\
5                         & Destoantes                                                                 & 0\%                    & 0\%                    & 0\%                    & 0\%                    & 0\%                     \\
6                         & Destoantes                                                                 & 0\%                    & 0\%                    & 0\%                    & 0\%                    & 0\%                     \\ \hline
1-6 & Média dos Testes                                                           & 26,51 & 30,53 & 30,62 & 35,65 & 37,52 \\ \hline
1-6 & \begin{tabular}[c]{@{}c@{}}Média Limpa \\ dos Testes\end{tabular}          & 26,51 & 30,53 & 30,62 & 35,65 & 37,52 \\ \hline
1-6 & \begin{tabular}[c]{@{}c@{}}Média dos \\ Desvios Padrão\end{tabular}        & 0,02  & 0,06  & 1,89  & 0,04  & 0,07  \\ \hline
1-6 & \begin{tabular}[c]{@{}c@{}}Média dos \\ Desvios Padrão Limpos\end{tabular} & 0,02  & 0,06  & 1,89  & 0,04  & 0,07  \\ \hline
\end{tabular}
\end{table}

% \begin{table}[]
\centering
\caption{My caption}
\label{my-label}
\begin{tabular}{|c|c|ccccc|}
\hline
\textbf{Teste}            & \textbf{Medida}                                                            & \textbf{Sensor 0}         & \textbf{Sensor 1}         & \textbf{Sensor 2}         & \textbf{Sensor 3}         & \textbf{Sensor 4}          \\ \hline
1                         & Média                                                                      & 44,57                     & 47,87                     & 50,13                     & 48,59                     & 53,65                      \\
2                         & Média                                                                      & 44,95                     & 47,83                     & 50,65                     & 48,77                     & 52,65                      \\
3                         & Média                                                                      & 44,85                     & 47,85                     & 52,61                     & 48,92                     & 63,21                      \\
4                         & Média                                                                      & 44,79                     & 47,91                     & 50,87                     & 48,67                     & 60,42                      \\
5                         & Média                                                                      & 45,27                     & 47,85                     & 50,74                     & 48,94                     & 52,55                      \\
6                         & Média                                                                      & 44,78                     & 47,63                     & 51,29                     & 48,73                     & 51,33                      \\ \hline
1                         & Média Limpa                                                                & 44,57                     & 47,87                     & 50,13                     & 48,59                     & 53,65                      \\
2                         & Média Limpa                                                                & 44,95                     & 47,83                     & 50,65                     & 48,77                     & 52,65                      \\
3                         & Média Limpa                                                                & 44,85                     & 47,85                     & 52,61                     & 48,92                     & 63,21                      \\
4                         & Média Limpa                                                                & 44,79                     & 47,91                     & 50,87                     & 48,67                     & 60,42                      \\
5                         & Média Limpa                                                                & 45,27                     & 47,85                     & 50,74                     & 48,94                     & 52,55                      \\
6                         & Média Limpa                                                                & 44,78                     & 47,63                     & 51,29                     & 48,73                     & 51,33                      \\ \hline
1                         & Desvio Padrão                                                              & 0,56                      & 0,33                      & 1,48                      & 0,53                      & 12,93                      \\
2                         & Desvio Padrão                                                              & 1,00                      & 0,37                      & 1,17                      & 0,49                      & 11,11                      \\
3                         & Desvio Padrão                                                              & 0,63                      & 0,36                      & 1,16                      & 0,54                      & 28,66                      \\
4                         & Desvio Padrão                                                              & 0,53                      & 0,28                      & 0,70                      & 0,51                      & 26,26                      \\
5                         & Desvio Padrão                                                              & 2,98                      & 0,35                      & 0,81                      & 0,48                      & 8,89                       \\
6                         & Desvio Padrão                                                              & 0,53                      & 0,48                      & 1,30                      & 0,49                      & 5,41                       \\ \hline
1                         & Desvio Padrão Limpo                                                        & 0,56                      & 0,33                      & 1,48                      & 0,53                      & 12,93                      \\
2                         & Desvio Padrão Limpo                                                        & 1,00                      & 0,37                      & 1,17                      & 0,49                      & 11,11                      \\
3                         & Desvio Padrão Limpo                                                        & 0,63                      & 0,36                      & 1,16                      & 0,54                      & 28,66                      \\
4                         & Desvio Padrão Limpo                                                        & 0,53                      & 0,28                      & 0,70                      & 0,51                      & 26,26                      \\
5                         & Desvio Padrão Limpo                                                        & 2,98                      & 0,35                      & 0,81                      & 0,48                      & 8,89                       \\
6                         & Desvio Padrão Limpo                                                        & 0,53                      & 0,48                      & 1,30                      & 0,49                      & 5,41                       \\ \hline
1                         & \textless (Média)/2                                                        & 0\%                    & 0\%                    & 0\%                    & 0\%                    & 0\%                     \\
2                         & \textless (Média)/2                                                        & 0\%                    & 0\%                    & 0\%                    & 0\%                    & 0\%                     \\
3                         & \textless (Média)/2                                                        & 0\%                    & 0\%                    & 0\%                    & 0\%                    & 0\%                     \\
4                         & \textless (Média)/2                                                        & 0\%                    & 0\%                    & 0\%                    & 0\%                    & 0\%                     \\
5                         & \textless (Média)/2                                                        & 0\%                    & 0\%                    & 0\%                    & 0\%                    & 0\%                     \\
6                         & \textless (Média)/2                                                        & 0\%                    & 0\%                    & 0\%                    & 0\%                    & 0\%                     \\ \hline
1                         & Destoantes                                                                 & 0\%                    & 0\%                    & 0\%                    & 0\%                    & 2,00\%                     \\
2                         & Destoantes                                                                 & 0,67\%                    & 0\%                    & 0\%                    & 0\%                    & 2,67\%                     \\
3                         & Destoantes                                                                 & 0\%                    & 0\%                    & 0\%                    & 0\%                    & 0\%                     \\
4                         & Destoantes                                                                 & 0\%                    & 0\%                    & 0\%                    & 0\%                    & 0\%                     \\
5                         & Destoantes                                                                 & 0,67\%                    & 0\%                    & 0\%                    & 0\%                    & 2,67\%                     \\
6                         & Destoantes                                                                 & 0\%                    & 0\%                    & 0\%                    & 0\%                    & 0,67\%                     \\ \hline
1-6 & Média dos Testes                                                           & 44,87 & 47,83 & 51,05 & 48,77 & 55,63 \\ \hline
1-6 & \begin{tabular}[c]{@{}c@{}}Média Limpa \\ dos Testes\end{tabular}          & 44,87 & 47,83 & 51,05 & 48,77 & 55,63 \\ \hline
1-6 & \begin{tabular}[c]{@{}c@{}}Média dos \\ Desvios Padrão\end{tabular}        & 0,97  & 0,07  & 0,29  & 0,02  & 9,59  \\ \hline
1-6 & \begin{tabular}[c]{@{}c@{}}Média dos \\ Desvios Padrão Limpos\end{tabular} & 0,97  & 0,07  & 0,29  & 0,02  & 9,59  \\ \hline
\end{tabular}
\end{table}

% \begin{table}[]
\centering
\caption{My caption}
\label{my-label}
\begin{tabular}{|c|c|ccccc|}
\hline
\textbf{Teste}            & \textbf{Medida}                                                            & \textbf{Sensor 0}          & \textbf{Sensor 1}          & \textbf{Sensor 2}         & \textbf{Sensor 3}         & \textbf{Sensor 4}          \\ \hline
1                         & Média                                                                      & 102,71                     & 107,05                     & 97,07                     & 92,46                     & 94,74                      \\
2                         & Média                                                                      & 102,46                     & 106,91                     & 97,19                     & 92,54                     & 94,54                      \\
3                         & Média                                                                      & 102,31                     & 106,82                     & 97,10                     & 92,37                     & 94,49                      \\
4                         & Média                                                                      & 102,37                     & 106,83                     & 97,07                     & 92,23                     & 94,20                      \\
5                         & Média                                                                      & 102,34                     & 106,81                     & 97,23                     & 92,46                     & 94,53                      \\
6                         & Média                                                                      & 102,07                     & 106,77                     & 97,18                     & 92,51                     & 94,65                      \\ \hline
1                         & Média Limpa                                                                & 102,71                     & 107,05                     & 97,07                     & 92,46                     & 94,74                      \\
2                         & Média Limpa                                                                & 102,46                     & 106,91                     & 97,19                     & 92,54                     & 94,54                      \\
3                         & Média Limpa                                                                & 102,31                     & 106,82                     & 97,10                     & 92,37                     & 94,49                      \\
4                         & Média Limpa                                                                & 102,37                     & 106,83                     & 97,07                     & 92,23                     & 94,20                      \\
5                         & Média Limpa                                                                & 102,34                     & 106,81                     & 97,23                     & 92,46                     & 94,53                      \\
6                         & Média Limpa                                                                & 102,07                     & 106,77                     & 97,18                     & 92,51                     & 94,65                      \\ \hline
1                         & Desvio Padrão                                                              & 0,99                       & 0,44                       & 0,73                      & 0,53                      & 1,12                       \\
2                         & Desvio Padrão                                                              & 0,66                       & 0,29                       & 0,75                      & 0,57                      & 0,98                       \\
3                         & Desvio Padrão                                                              & 0,60                       & 0,39                       & 0,78                      & 0,56                      & 0,99                       \\
4                         & Desvio Padrão                                                              & 0,66                       & 0,40                       & 0,81                      & 0,44                      & 0,62                       \\
5                         & Desvio Padrão                                                              & 0,61                       & 0,39                       & 0,76                      & 0,54                      & 0,95                       \\
6                         & Desvio Padrão                                                              & 0,53                       & 0,42                       & 0,71                      & 0,58                      & 0,99                       \\ \hline
1                         & Desvio Padrão Limpo                                                        & 0,99                       & 0,44                       & 0,73                      & 0,53                      & 1,12                       \\
2                         & Desvio Padrão Limpo                                                        & 0,66                       & 0,29                       & 0,75                      & 0,57                      & 0,98                       \\
3                         & Desvio Padrão Limpo                                                        & 0,60                       & 0,39                       & 0,78                      & 0,56                      & 0,99                       \\
4                         & Desvio Padrão Limpo                                                        & 0,66                       & 0,40                       & 0,81                      & 0,44                      & 0,62                       \\
5                         & Desvio Padrão Limpo                                                        & 0,61                       & 0,39                       & 0,76                      & 0,54                      & 0,95                       \\
6                         & Desvio Padrão Limpo                                                        & 0,53                       & 0,42                       & 0,71                      & 0,58                      & 0,99                       \\ \hline
1                         & \textless (Média)/2                                                        & 0\%                     & 0\%                     & 0\%                    & 0\%                    & 0\%                     \\
2                         & \textless (Média)/2                                                        & 0\%                     & 0\%                     & 0\%                    & 0\%                    & 0\%                     \\
3                         & \textless (Média)/2                                                        & 0\%                     & 0\%                     & 0\%                    & 0\%                    & 0\%                     \\
4                         & \textless (Média)/2                                                        & 0\%                     & 0\%                     & 0\%                    & 0\%                    & 0\%                     \\
5                         & \textless (Média)/2                                                        & 0\%                     & 0\%                     & 0\%                    & 0\%                    & 0\%                     \\
6                         & \textless (Média)/2                                                        & 0\%                     & 0\%                     & 0\%                    & 0\%                    & 0\%                     \\ \hline
1                         & Destoantes                                                                 & 0\%                     & 0\%                     & 0\%                    & 0\%                    & 0\%                     \\
2                         & Destoantes                                                                 & 0\%                     & 0\%                     & 0\%                    & 0\%                    & 0\%                     \\
3                         & Destoantes                                                                 & 0\%                     & 0\%                     & 0\%                    & 0\%                    & 0\%                     \\
4                         & Destoantes                                                                 & 0\%                     & 0\%                     & 0\%                    & 0\%                    & 0\%                     \\
5                         & Destoantes                                                                 & 0\%                     & 0\%                     & 0\%                    & 0\%                    & 0\%                     \\
6                         & Destoantes                                                                 & 0\%                     & 0\%                     & 0\%                    & 0\%                    & 0\%                     \\ \hline
1-6 & Média dos Testes                                                           & 102,38 & 106,86 & 97,14 & 92,43 & 94,52 \\ \hline
1-6 & \begin{tabular}[c]{@{}c@{}}Média Limpa \\ dos Testes\end{tabular}          & 102,38 & 106,86 & 97,14 & 92,43 & 94,52 \\ \hline
1-6 & \begin{tabular}[c]{@{}c@{}}Média dos \\ Desvios Padrão\end{tabular}        & 0,16   & 0,05   & 0,04  & 0,05  & 0,17  \\ \hline
1-6 & \begin{tabular}[c]{@{}c@{}}Média dos \\ Desvios Padrão Limpos\end{tabular} & 0,16   & 0,05   & 0,04  & 0,05  & 0,17  \\ \hline
\end{tabular}
\end{table}

% \begin{table}[]
\centering
\caption{My caption}
\label{my-label}
\begin{tabular}{|c|c|ccccc|}
\hline
\textbf{Teste}            & \textbf{Medida}                                                            & \textbf{Sensor 0}          & \textbf{Sensor 1}          & \textbf{Sensor 2}          & \textbf{Sensor 3}          & \textbf{Sensor 4}           \\ \hline
1                         & Média                                                                      & 148,70                     & 153,18                     & 153,45                     & 156,41                     & 174,46                      \\
2                         & Média                                                                      & 149,05                     & 153,01                     & 153,63                     & 156,95                     & 176,01                      \\
3                         & Média                                                                      & 148,62                     & 153,01                     & 153,10                     & 156,97                     & 226,32                      \\
4                         & Média                                                                      & 148,90                     & 153,04                     & 154,24                     & 157,13                     & 192,36                      \\
5                         & Média                                                                      & 148,53                     & 152,96                     & 153,33                     & 157,28                     & 161,61                      \\
6                         & Média                                                                      & 148,45                     & 152,57                     & 152,18                     & 156,33                     & 229,58                      \\ \hline
1                         & Média Limpa                                                                & 148,70                     & 153,18                     & 153,45                     & 156,41                     & 158,35                      \\
2                         & Média Limpa                                                                & 149,05                     & 153,01                     & 153,63                     & 156,95                     & 158,28                      \\
3                         & Média Limpa                                                                & 148,62                     & 153,01                     & 153,10                     & 156,97                     & 158,78                      \\
4                         & Média Limpa                                                                & 148,90                     & 153,04                     & 154,24                     & 157,13                     & 158,56                      \\
5                         & Média Limpa                                                                & 148,53                     & 152,96                     & 153,33                     & 157,28                     & 158,39                      \\
6                         & Média Limpa                                                                & 148,45                     & 152,57                     & 152,18                     & 156,33                     & 158,84                      \\ \hline
1                         & Desvio Padrão                                                              & 0,63                       & 0,39                       & 0,64                       & 0,88                       & 60,49                       \\
2                         & Desvio Padrão                                                              & 0,47                       & 0,08                       & 0,61                       & 0,68                       & 63,23                       \\
3                         & Desvio Padrão                                                              & 0,63                       & 0,23                       & 0,86                       & 0,82                       & 108,67                      \\
4                         & Desvio Padrão                                                              & 0,47                       & 0,30                       & 6,01                       & 0,72                       & 84,06                       \\
5                         & Desvio Padrão                                                              & 0,62                       & 0,20                       & 0,62                       & 0,65                       & 27,81                       \\
6                         & Desvio Padrão                                                              & 0,59                       & 4,90                       & 5,68                       & 0,89                       & 110,18                      \\ \hline
1                         & Desvio Padrão Limpo                                                        & 0,63                       & 0,39                       & 0,64                       & 0,88                       & 0,88                        \\
2                         & Desvio Padrão Limpo                                                        & 0,47                       & 0,08                       & 0,61                       & 0,68                       & 0,86                        \\
3                         & Desvio Padrão Limpo                                                        & 0,63                       & 0,23                       & 0,86                       & 0,82                       & 0,80                        \\
4                         & Desvio Padrão Limpo                                                        & 0,47                       & 0,30                       & 6,01                       & 0,72                       & 0,79                        \\
5                         & Desvio Padrão Limpo                                                        & 0,62                       & 0,20                       & 0,62                       & 0,65                       & 0,72                        \\
6                         & Desvio Padrão Limpo                                                        & 0,59                       & 4,90                       & 5,68                       & 0,89                       & 2,09                        \\ \hline
1                         & \textless (Média)/2                                                        & 0\%                     & 0\%                     & 0\%                     & 0\%                     & 0\%                      \\
2                         & \textless (Média)/2                                                        & 0\%                     & 0\%                     & 0\%                     & 0\%                     & 0\%                      \\
3                         & \textless (Média)/2                                                        & 0\%                     & 0\%                     & 0\%                     & 0\%                     & 0\%                      \\
4                         & \textless (Média)/2                                                        & 0\%                     & 0\%                     & 0\%                     & 0\%                     & 0\%                      \\
5                         & \textless (Média)/2                                                        & 0\%                     & 0\%                     & 0\%                     & 0\%                     & 0\%                      \\
6                         & \textless (Média)/2                                                        & 0\%                     & 0\%                     & 0\%                     & 0\%                     & 0\%                      \\ \hline
1                         & Destoantes                                                                 & 0\%                     & 0\%                     & 0\%                     & 0\%                     & 0\%                      \\
2                         & Destoantes                                                                 & 0\%                     & 0\%                     & 0\%                     & 0\%                     & 0\%                      \\
3                         & Destoantes                                                                 & 0\%                     & 0\%                     & 0\%                     & 0\%                     & 0\%                      \\
4                         & Destoantes                                                                 & 0\%                     & 0\%                     & 0,67\%                     & 0\%                     & 0\%                      \\
5                         & Destoantes                                                                 & 0\%                     & 0\%                     & 0\%                     & 0\%                     & 0\%                      \\
6                         & Destoantes                                                                 & 0\%                     & 0\%                     & 0\%                     & 0\%                     & 0\%                      \\ \hline
1-6 & Média dos Testes                                                           & 148,71 & 152,96 & 153,32 & 156,84 & 193,39 \\ \hline
1-6 & \begin{tabular}[c]{@{}c@{}}Média Limpa \\ dos Testes\end{tabular}          & 148,71 & 152,96 & 153,32 & 156,84 & 158,53 \\ \hline
1-6 & \begin{tabular}[c]{@{}c@{}}Média dos \\ Desvios Padrão\end{tabular}        & 0,08   & 1,91   & 2,67   & 0,10   & 31,70  \\ \hline
1-6 & \begin{tabular}[c]{@{}c@{}}Média dos \\ Desvios Padrão Limpos\end{tabular} & 0,08   & 1,91   & 2,67   & 0,10   & 0,52   \\ \hline
\end{tabular}
\end{table}

% \input{../Tabelas/200_din.tex}
% 
% \subsection{Leituras Estáticas}
% \begin{table}[]
\centering
\caption{My caption}
\label{my-label}
\begin{tabular}{|c|c|ccccc|}
\hline
\textbf{time out} & \textbf{Medida}     & \textbf{Sensor 0} & \textbf{Sensor 1} & \textbf{Sensor 2} & \textbf{Sensor 3} & \textbf{Sensor 4} \\ \hline
30ms              & Média               & 26,43             & 30,13             & 31,14             & 35,62             & 37,19             \\
30ms              & Média               & 26,37             & 30,12             & 30,99             & 35,66             & 37,21             \\
45ms              & Média               & 26,36             & 30,51             & 32,61             & 35,67             & 37,34             \\
45ms              & Média               & 26,45             & 30,26             & 31,59             & 35,75             & 37,17             \\
60ms              & Média               & 26,41             & 30,12             & 31,28             & 35,67             & 37,21             \\
60ms              & Média               & 26,46             & 30,01             & 30,38             & 35,61             & 37,15             \\ \hline
30ms              & Média Limpa         & 26,43             & 30,13             & 31,14             & 35,62             & 37,19             \\
30ms              & Média Limpa         & 26,37             & 30,12             & 30,99             & 35,66             & 37,21             \\
45ms              & Média Limpa         & 26,36             & 30,51             & 32,61             & 35,67             & 37,34             \\
45ms              & Média Limpa         & 26,45             & 30,26             & 31,59             & 35,75             & 37,17             \\
60ms              & Média Limpa         & 26,41             & 30,12             & 31,28             & 35,67             & 37,21             \\
60ms              & Média Limpa         & 26,46             & 30,01             & 30,38             & 35,61             & 37,15             \\ \hline
30ms              & Desvio Padrão       & 0,50              & 0,34              & 1,51              & 0,49              & 0,41              \\
30ms              & Desvio Padrão       & 0,48              & 0,33              & 1,19              & 0,48              & 0,43              \\
45ms              & Desvio Padrão       & 0,48              & 0,50              & 1,68              & 0,47              & 0,52              \\
45ms              & Desvio Padrão       & 0,50              & 0,44              & 1,63              & 0,44              & 0,41              \\
60ms              & Desvio Padrão       & 0,49              & 0,33              & 1,42              & 0,47              & 0,41              \\
60ms              & Desvio Padrão       & 0,50              & 0,12              & 0,70              & 0,49              & 0,38              \\ \hline
30ms              & Desvio Padrão Limpo & 0,50              & 0,34              & 1,51              & 0,49              & 0,41              \\
30ms              & Desvio Padrão Limpo & 0,48              & 0,33              & 1,19              & 0,48              & 0,43              \\
45ms              & Desvio Padrão Limpo & 0,48              & 0,50              & 1,68              & 0,47              & 0,52              \\
45ms              & Desvio Padrão Limpo & 0,50              & 0,44              & 1,63              & 0,44              & 0,41              \\
60ms              & Desvio Padrão Limpo & 0,49              & 0,33              & 1,42              & 0,47              & 0,41              \\
60ms              & Desvio Padrão Limpo & 0,50              & 0,12              & 0,70              & 0,49              & 0,38              \\ \hline
30ms              & \textless (Média)/2 & 0\%            & 0\%            & 0\%            & 0\%            & 0\%            \\
30ms              & \textless (Média)/2 & 0\%            & 0\%            & 0\%            & 0\%            & 0\%            \\
45ms              & \textless (Média)/2 & 0\%            & 0\%            & 0\%            & 0\%            & 0\%            \\
45ms              & \textless (Média)/2 & 0\%            & 0\%            & 0\%            & 0\%            & 0\%            \\
60ms              & \textless (Média)/2 & 0\%            & 0\%            & 0\%            & 0\%            & 0\%            \\
60ms              & \textless (Média)/2 & 0\%            & 0\%            & 0\%            & 0\%            & 0\%            \\ \hline
30ms              & Destoantes          & 0\%            & 0\%            & 0\%            & 0\%            & 0\%            \\
30ms              & Destoantes          & 0\%            & 0\%            & 0,67\%            & 0\%            & 0\%            \\
45ms              & Destoantes          & 0\%            & 0\%            & 0\%            & 0\%            & 0\%            \\
45ms              & Destoantes          & 0\%            & 0\%            & 0\%            & 0\%            & 0\%            \\
60ms              & Destoantes          & 0\%            & 0\%            & 0\%            & 0\%            & 0\%            \\
60ms              & Destoantes          & 0\%            & 0\%            & 0\%            & 0\%            & 0\%            \\ \hline
\end{tabular}
\end{table}

% \begin{table}[]
\centering
\caption{My caption}
\label{my-label}
\begin{tabular}{|c|c|ccccc|}
\hline
\textbf{time out} & \textbf{Medida}     & \textbf{Sensor 0} & \textbf{Sensor 1} & \textbf{Sensor 2} & \textbf{Sensor 3} & \textbf{Sensor 4} \\ \hline
30ms              & Média               & 44,80             & 47,77             & 50,86             & 48,39             & 53,25             \\
30ms              & Média               & 45,21             & 47,91             & 50,86             & 48,35             & 52,49             \\
45ms              & Média               & 45,35             & 47,46             & 50,45             & 48,53             & 53,61             \\
45ms              & Média               & 44,70             & 47,72             & 51,35             & 48,56             & 53,06             \\
60ms              & Média               & 45,86             & 47,76             & 50,81             & 48,49             & 51,69             \\
60ms              & Média               & 45,68             & 47,89             & 51,54             & 48,61             & 53,26             \\ \hline
30ms              & Média Limpa         & 44,80             & 47,77             & 50,86             & 48,39             & 53,25             \\
30ms              & Média Limpa         & 45,21             & 47,91             & 50,86             & 48,35             & 52,49             \\
45ms              & Média Limpa         & 45,35             & 47,46             & 50,45             & 48,53             & 53,61             \\
45ms              & Média Limpa         & 44,70             & 47,72             & 51,35             & 48,56             & 53,06             \\
60ms              & Média Limpa         & 45,86             & 47,76             & 50,81             & 48,49             & 51,69             \\
60ms              & Média Limpa         & 45,68             & 47,89             & 51,54             & 48,61             & 53,26             \\ \hline
30ms              & Desvio Padrão       & 0,65              & 0,42              & 0,74              & 0,50              & 11,99             \\
30ms              & Desvio Padrão       & 2,96              & 0,29              & 0,74              & 0,48              & 11,24             \\
45ms              & Desvio Padrão       & 3,38              & 0,50              & 0,97              & 0,51              & 13,46             \\
45ms              & Desvio Padrão       & 0,76              & 0,45              & 0,81              & 0,50              & 12,03             \\
60ms              & Desvio Padrão       & 7,40              & 0,43              & 0,87              & 0,50              & 7,44              \\
60ms              & Desvio Padrão       & 6,25              & 0,31              & 0,81              & 0,50              & 11,92             \\ \hline
30ms              & Desvio Padrão Limpo & 0,65              & 0,42              & 0,74              & 0,50              & 11,99             \\
30ms              & Desvio Padrão Limpo & 2,96              & 0,29              & 0,74              & 0,48              & 11,24             \\
45ms              & Desvio Padrão Limpo & 3,38              & 0,50              & 0,97              & 0,51              & 13,46             \\
45ms              & Desvio Padrão Limpo & 0,76              & 0,45              & 0,81              & 0,50              & 12,03             \\
60ms              & Desvio Padrão Limpo & 7,40              & 0,43              & 0,87              & 0,50              & 7,44              \\
60ms              & Desvio Padrão Limpo & 6,25              & 0,31              & 0,81              & 0,50              & 11,92             \\ \hline
30ms              & \textless (Média)/2 & 0\%            & 0\%            & 0\%            & 0\%            & 0\%            \\
30ms              & \textless (Média)/2 & 0\%            & 0\%            & 0\%            & 0\%            & 0\%            \\
45ms              & \textless (Média)/2 & 0\%            & 0\%            & 0\%            & 0\%            & 0\%            \\
45ms              & \textless (Média)/2 & 0\%            & 0\%            & 0\%            & 0\%            & 0\%            \\
60ms              & \textless (Média)/2 & 0\%            & 0\%            & 0\%            & 0\%            & 0\%            \\
60ms              & \textless (Média)/2 & 0\%            & 0\%            & 0\%            & 0\%            & 0\%            \\ \hline
30ms              & Destoantes          & 0\%            & 0\%            & 0\%            & 0\%            & 4,00\%            \\
30ms              & Destoantes          & 0,67\%            & 0\%            & 0\%            & 0\%            & 2,67\%            \\
45ms              & Destoantes          & 0,67\%            & 0\%            & 0\%            & 0\%            & 4,00\%            \\
45ms              & Destoantes          & 0,67\%            & 0\%            & 0\%            & 0\%            & 3,33\%            \\
60ms              & Destoantes          & 1,33\%            & 0\%            & 0\%            & 0\%            & 1,33\%            \\
60ms              & Destoantes          & 1,33\%            & 0\%            & 0\%            & 0\%            & 4,00\%            \\ \hline
\end{tabular}
\end{table}

% \begin{table}[]
\centering
\caption{My caption}
\label{my-label}
\begin{tabular}{|c|c|ccccc|}
\hline
\textbf{time out} & \textbf{Medida}     & \textbf{Sensor 0} & \textbf{Sensor 1} & \textbf{Sensor 2} & \textbf{Sensor 3} & \textbf{Sensor 4} \\ \hline
30ms              & Média               & 102,49            & 106,83            & 96,83             & 92,05             & 94,19             \\
30ms              & Média               & 102,41            & 106,85            & 96,83             & 92,21             & 94,49             \\
45ms              & Média               & 102,78            & 106,86            & 96,96             & 92,05             & 94,29             \\
45ms              & Média               & 102,41            & 106,89            & 97,07             & 92,20             & 94,27             \\
60ms              & Média               & 102,63            & 106,96            & 97,46             & 92,14             & 94,24             \\
60ms              & Média               & 102,59            & 106,90            & 97,07             & 92,43             & 94,60             \\ \hline
30ms              & Média Limpa         & 102,49            & 106,83            & 96,83             & 92,05             & 94,19             \\
30ms              & Média Limpa         & 102,41            & 106,85            & 96,83             & 92,21             & 94,49             \\
45ms              & Média Limpa         & 102,78            & 106,86            & 96,96             & 92,05             & 94,29             \\
45ms              & Média Limpa         & 102,41            & 106,89            & 97,07             & 92,20             & 94,27             \\
60ms              & Média Limpa         & 102,63            & 106,96            & 97,46             & 92,14             & 94,24             \\
60ms              & Média Limpa         & 102,59            & 106,90            & 97,07             & 92,43             & 94,60             \\ \hline
30ms              & Desvio Padrão       & 0,66              & 0,39              & 0,78              & 0,24              & 1,02              \\
30ms              & Desvio Padrão       & 0,66              & 0,38              & 0,77              & 0,41              & 1,01              \\
45ms              & Desvio Padrão       & 0,83              & 0,35              & 0,72              & 0,33              & 1,01              \\
45ms              & Desvio Padrão       & 0,66              & 0,31              & 0,77              & 0,42              & 0,83              \\
60ms              & Desvio Padrão       & 0,69              & 0,33              & 2,52              & 0,37              & 1,07              \\
60ms              & Desvio Padrão       & 0,73              & 0,34              & 0,72              & 0,50              & 1,22              \\ \hline
30ms              & Desvio Padrão Limpo & 0,66              & 0,39              & 0,78              & 0,24              & 1,02              \\
30ms              & Desvio Padrão Limpo & 0,66              & 0,38              & 0,77              & 0,41              & 1,01              \\
45ms              & Desvio Padrão Limpo & 0,83              & 0,35              & 0,72              & 0,33              & 1,01              \\
45ms              & Desvio Padrão Limpo & 0,66              & 0,31              & 0,77              & 0,42              & 0,83              \\
60ms              & Desvio Padrão Limpo & 0,69              & 0,33              & 2,52              & 0,37              & 1,07              \\
60ms              & Desvio Padrão Limpo & 0,73              & 0,34              & 0,72              & 0,50              & 1,22              \\ \hline
30ms              & \textless (Média)/2 & 0\%            & 0\%            & 0\%            & 0\%            & 0\%            \\
30ms              & \textless (Média)/2 & 0\%            & 0\%            & 0\%            & 0\%            & 0\%            \\
45ms              & \textless (Média)/2 & 0\%            & 0\%            & 0\%            & 0\%            & 0\%            \\
45ms              & \textless (Média)/2 & 0\%            & 0\%            & 0\%            & 0\%            & 0\%            \\
60ms              & \textless (Média)/2 & 0\%            & 0\%            & 0\%            & 0\%            & 0\%            \\
60ms              & \textless (Média)/2 & 0\%            & 0\%            & 0\%            & 0\%            & 0\%            \\ \hline
30ms              & Destoantes          & 0\%            & 0\%            & 0\%            & 0\%            & 0\%            \\
30ms              & Destoantes          & 0\%            & 0\%            & 0\%            & 0\%            & 0\%            \\
45ms              & Destoantes          & 0\%            & 0\%            & 0\%            & 0\%            & 0\%            \\
45ms              & Destoantes          & 0\%            & 0\%            & 0\%            & 0\%            & 0\%            \\
60ms              & Destoantes          & 0\%            & 0\%            & 1,33\%            & 0\%            & 0\%            \\
60ms              & Destoantes          & 0\%            & 0\%            & 0\%            & 0\%            & 0,67\%            \\ \hline
\end{tabular}
\end{table}

% \input{../Tabelas/150_est.tex}
% \begin{table}[]
\centering
\caption{My caption}
\label{my-label}
\begin{tabular}{|c|c|ccccc|}
\hline
\textbf{time out} & \textbf{Medida}     & \textbf{Sensor 0} & \textbf{Sensor 1} & \textbf{Sensor 2} & \textbf{Sensor 3} & \textbf{Sensor 4} \\ \hline
30ms              & Média               & 148,52            & 152,17            & 152,97            & 156,65            & 230,77            \\
30ms              & Média               & 148,97            & 153,53            & 153,65            & 156,53            & 196,81            \\
45ms              & Média               & 149,13            & 153,13            & 153,95            & 156,42            & 184,03            \\
45ms              & Média               & 148,63            & 151,45            & 153,23            & 157,11            & 210,04            \\
60ms              & Média               & 148,84            & 153,09            & 153,42            & 156,79            & 256,77            \\
60ms              & Média               & 148,49            & 152,95            & 153,03            & 156,81            & 164,53            \\ \hline
30ms              & Média Limpa         & 148,52            & 152,17            & 152,97            & 156,65            & 158,25            \\
30ms              & Média Limpa         & 148,97            & 153,53            & 153,65            & 156,53            & 158,10            \\
45ms              & Média Limpa         & 149,13            & 153,13            & 153,95            & 156,42            & 158,25            \\
45ms              & Média Limpa         & 148,63            & 151,45            & 153,23            & 157,11            & 158,53            \\
60ms              & Média Limpa         & 148,84            & 153,09            & 153,42            & 156,79            & 158,60            \\
60ms              & Média Limpa         & 148,49            & 152,95            & 153,03            & 156,81            & 158,08            \\ \hline
30ms              & Desvio Padrão       & 0,60              & 7,48              & 6,44              & 0,76              & 111,16            \\
30ms              & Desvio Padrão       & 0,52              & 0,50              & 0,65              & 0,68              & 88,98             \\
45ms              & Desvio Padrão       & 0,42              & 0,33              & 0,73              & 0,74              & 74,88             \\
45ms              & Desvio Padrão       & 0,61              & 9,46              & 0,89              & 0,64              & 99,26             \\
60ms              & Desvio Padrão       & 0,57              & 0,28              & 0,63              & 0,57              & 118,98            \\
60ms              & Desvio Padrão       & 0,54              & 0,21              & 0,68              & 0,61              & 39,11             \\ \hline
30ms              & Desvio Padrão Limpo & 0,60              & 7,48              & 6,44              & 0,76              & 0,96              \\
30ms              & Desvio Padrão Limpo & 0,52              & 0,50              & 0,65              & 0,68              & 0,93              \\
45ms              & Desvio Padrão Limpo & 0,42              & 0,33              & 0,73              & 0,74              & 0,69              \\
45ms              & Desvio Padrão Limpo & 0,61              & 9,46              & 0,89              & 0,64              & 0,69              \\
60ms              & Desvio Padrão Limpo & 0,57              & 0,28              & 0,63              & 0,57              & 0,65              \\
60ms              & Desvio Padrão Limpo & 0,54              & 0,21              & 0,68              & 0,61              & 0,70              \\ \hline
30ms              & \textless (Média)/2 & 0\%            & 0,67\%            & 0,67\%            & 0\%            & 0\%            \\
30ms              & \textless (Média)/2 & 0\%            & 0\%            & 0\%            & 0\%            & 0\%            \\
45ms              & \textless (Média)/2 & 0\%            & 0\%            & 0\%            & 0\%            & 0\%            \\
45ms              & \textless (Média)/2 & 0\%            & 0\%            & 0\%            & 0\%            & 0\%            \\
60ms              & \textless (Média)/2 & 0\%            & 0\%            & 0\%            & 0\%            & 0\%            \\
60ms              & \textless (Média)/2 & 0\%            & 0\%            & 0\%            & 0\%            & 0\%            \\ \hline
30ms              & Destoantes          & 0\%            & 0\%            & 0\%            & 0\%            & 0\%            \\
30ms              & Destoantes          & 0\%            & 0\%            & 0\%            & 0\%            & 0\%            \\
45ms              & Destoantes          & 0\%            & 0\%            & 0\%            & 0\%            & 0\%            \\
45ms              & Destoantes          & 0\%            & 0\%            & 0\%            & 0\%            & 0\%            \\
60ms              & Destoantes          & 0\%            & 0\%            & 0\%            & 0\%            & 0\%            \\
60ms              & Destoantes          & 0\%            & 0\%            & 0\%            & 0\%            & 0\%            \\ \hline
\end{tabular}
\end{table}


 \subsection{Interfência entre BLDC e sonares}
% % Please add the following required packages to your document preamble:
% \usepackage{multirow}
\begin{table}[]
\centering
\caption{My caption}
\label{my-label}
\begin{tabular}{|c|c|ccccc|}
\hline
\textbf{Motores}                & \textbf{Medida}  & \textbf{Sensor 0} & \textbf{Sensor 1} & \textbf{Sensor 2} & \textbf{Sensor 3} & \textbf{Sensor 4} \\ \hline
\multirow{3}{*}{STP}            & Média            & 399               & 339               & 400               & 400               & 400               \\
                                & \textless 400    & 0,43\%            & 21,26\%           & 0\%            & 0\%            & 0\%            \\
                                & \textgreater 400 & 0\%            & 0\%            & 0\%            & 0\%            & 0\%            \\ \hline
\multirow{3}{*}{E\_L}           & Média            & 399               & 236               & 400               & 400               & 400               \\
                                & \textless 400    & 0,30\%            & 59,34\%           & 0\%            & 0\%            & 0\%            \\
                                & \textgreater 400 & 0\%            & 0\%            & 0\%            & 0\%            & 0\%            \\ \hline
\multirow{3}{*}{E\_M}           & Média            & 400               & 219               & 400               & 400               & 400               \\
                                & \textless 400    & 0\%            & 66,56\%           & 0\%            & 0\%            & 0\%            \\
                                & \textgreater 400 & 0\%            & 0\%            & 0\%            & 0\%            & 0\%            \\ \hline
\multirow{3}{*}{E\_F}           & Média            & 400               & 233               & 400               & 400               & 400               \\
                                & \textless 400    & 0\%            & 61,28\%           & 0\%            & 0\%            & 0\%            \\
                                & \textgreater 400 & 0\%            & 0\%            & 0\%            & 0\%            & 0\%            \\ \hline
\multirow{3}{*}{D\_L}           & Média            & 399               & 316               & 400               & 400               & 400               \\
                                & \textless 400    & 0,34\%            & 31,54\%           & 0\%            & 0\%            & 0\%            \\
                                & \textgreater 400 & 0\%            & 0\%            & 0\%            & 0\%            & 0\%            \\ \hline
\multirow{3}{*}{D\_M}           & Média            & 400               & 217               & 400               & 400               & 400               \\
                                & \textless 400    & 0\%            & 66,78\%           & 0\%            & 0\%            & 0\%            \\
                                & \textgreater 400 & 0\%            & 0\%            & 0\%            & 0\%            & 0\%            \\ \hline
\multirow{3}{*}{D\_F}           & Média            & 400               & 247               & 400               & 400               & 400               \\
                                & \textless 400    & 0\%            & 55,56\%           & 0\%            & 0\%            & 0\%            \\
                                & \textgreater 400 & 0\%            & 0\%            & 0\%            & 0\%            & 0\%            \\ \hline
\multirow{3}{*}{FS}             & Média            & 400               & 290               & 400               & 400               & 400               \\
                                & \textless 400    & 0\%            & 40,27\%           & 0\%            & 0\%            & 0\%            \\
                                & \textgreater 400 & 0\%            & 0\%            & 0\%            & 0\%            & 0\%            \\ \hline
\multirow{3}{*}{Fre}            & Média            & 400               & 156               & 400               & 400               & 400               \\
                                & \textless 400    & 0\%            & 89,23\%           & 0\%            & 0\%            & 0\%            \\
                                & \textgreater 400 & 0\%            & 0\%            & 0\%            & 0\%            & 0\%            \\ \hline
\multirow{4}{*}{\textbf{TOTAL}} & Média            & 399               & 261               & 400               & 400               & 400               \\
                                & Desvio Padrão    & 13                & 138               & 0                 & 0                 & 0                 \\
                                & \textless 400    & 0,16\%            & 50,42\%           & 0\%            & 0\%            & 0\%            \\
                                & \textgreater 400 & 0\%            & 0\%            & 0\%            & 0\%            & 0\%            \\ \hline
\end{tabular}
\end{table}

% Please add the following required packages to your document preamble:
% \usepackage{multirow}
\begin{table}[]
\centering
\caption{My caption}
\label{my-label}
\begin{tabular}{|c|c|ccccc|}
\hline
\textbf{Motores}                & \textbf{Medida}  & \textbf{Sensor 0} & \textbf{Sensor 1} & \textbf{Sensor 2} & \textbf{Sensor 3} & \textbf{Sensor 4} \\ \hline
\multirow{3}{*}{STP}            & Média            & 400               & 373               & 400               & 400               & 400               \\
                                & \textless 400    & 0\%            & 11,09\%           & 0\%            & 0\%            & 0\%            \\
                                & \textgreater 400 & 0\%            & 0\%            & 0\%            & 0\%            & 0\%            \\ \hline
\multirow{3}{*}{E\_L}           & Média            & 400               & 141               & 400               & 400               & 400               \\
                                & \textless 400    & 0\%            & 89,55\%           & 0\%            & 0\%            & 0\%            \\
                                & \textgreater 400 & 0\%            & 0\%            & 0\%            & 0\%            & 0\%            \\ \hline
\multirow{3}{*}{E\_M}           & Média            & 400               & 121               & 400               & 400               & 400               \\
                                & \textless 400    & 0\%            & 96,97\%           & 0\%            & 0\%            & 0\%            \\
                                & \textgreater 400 & 0\%            & 0\%            & 0\%            & 0\%            & 0\%            \\ \hline
\multirow{3}{*}{E\_F}           & Média            & 400               & 116               & 400               & 400               & 400               \\
                                & \textless 400    & 0\%            & 98,48\%           & 0\%            & 0\%            & 0\%            \\
                                & \textgreater 400 & 0\%            & 0\%            & 0\%            & 0\%            & 0\%            \\ \hline
\multirow{3}{*}{D\_L}           & Média            & 400               & 207               & 400               & 400               & 400               \\
                                & \textless 400    & 0\%            & 67,84\%           & 0\%            & 0\%            & 0\%            \\
                                & \textgreater 400 & 0\%            & 0\%            & 0\%            & 0\%            & 0\%            \\ \hline
\multirow{3}{*}{D\_M}           & Média            & 400               & 148               & 400               & 400               & 400               \\
                                & \textless 400    & 0\%            & 87,94\%           & 0\%            & 0\%            & 0\%            \\
                                & \textgreater 400 & 0\%            & 0\%            & 0\%            & 0\%            & 0\%            \\ \hline
\multirow{3}{*}{D\_F}           & Média            & 400               & 242               & 400               & 400               & 400               \\
                                & \textless 400    & 0\%            & 56,57\%           & 0\%            & 0\%            & 0\%            \\
                                & \textgreater 400 & 0\%            & 0\%            & 0\%            & 0\%            & 0\%            \\ \hline
\multirow{3}{*}{FS}             & Média            & 400               & 218               & 400               & 400               & 400               \\
                                & \textless 400    & 0\%            & 64,65\%           & 0\%            & 0\%            & 0\%            \\
                                & \textgreater 400 & 0\%            & 0\%            & 0\%            & 0\%            & 0\%            \\ \hline
\multirow{3}{*}{Fre}            & Média            & 400               & 123               & 400               & 400               & 400               \\
                                & \textless 400    & 0\%            & 96,48\%           & 0\%            & 0\%            & 0\%            \\
                                & \textgreater 400 & 0\%            & 0\%            & 0\%            & 0\%            & 0\%            \\ \hline
\multirow{4}{*}{\textbf{TOTAL}} & Média            & 400               & 210               & 400               & 400               & 400               \\
                                & Desvio Padrão    & 0                 & 135               & 0                 & 0                 & 0                 \\
                                & \textless 400    & 0\%            & 66,83\%           & 0\%            & 0\%            & 0\%            \\
                                & \textgreater 400 & 0\%            & 0\%            & 0\%            & 0\%            & 0\%            \\ \hline
\end{tabular}
\end{table}

% % Please add the following required packages to your document preamble:
% \usepackage{multirow}
\begin{table}[]
\centering
\caption{Teste 3}
\label{teste_3}
\begin{tabular}{|c|c|ccccc|}
\hline
\textbf{Motores}                & \textbf{Medida}  & \textbf{Sensor 0} & \textbf{Sensor 1} & \textbf{Sensor 2} & \textbf{Sensor 3} & \textbf{Sensor 4} \\ \hline
\multirow{3}{*}{STP}            & Média            & 400               & 384               & 400               & 400               & 399               \\
                                & \textless 400    & 0\%            & 8,14\%            & 0\%            & 0\%            & 0,68\%            \\
                                & \textgreater 400 & 0\%            & 0\%            & 0\%            & 0\%            & 0\%            \\ \hline
\multirow{3}{*}{E\_L}           & Média            & 400               & 121               & 400               & 400               & 400               \\
                                & \textless 400    & 0\%            & 98,50\%           & 0\%            & 0\%            & 0,50\%            \\
                                & \textgreater 400 & 0\%            & 0\%            & 0\%            & 0\%            & 0\%            \\ \hline
\multirow{3}{*}{E\_M}           & Média            & 400               & 115               & 400               & 400               & 400               \\
                                & \textless 400    & 0\%            & 100\%          & 0\%            & 0\%            & 0\%            \\
                                & \textgreater 400 & 0\%            & 0\%            & 0\%            & 0\%            & 0\%            \\ \hline
\multirow{3}{*}{E\_F}           & Média            & 400               & 117               & 400               & 400               & 400               \\
                                & \textless 400    & 0\%            & 100\%          & 0\%            & 0\%            & 0\%            \\
                                & \textgreater 400 & 0\%            & 0\%            & 0\%            & 0\%            & 0\%            \\ \hline
\multirow{3}{*}{D\_L}           & Média            & 400               & 149               & 400               & 400               & 400               \\
                                & \textless 400    & 0\%            & 88,94\%           & 0\%            & 0\%            & 0,50\%            \\
                                & \textgreater 400 & 0\%            & 0\%            & 0\%            & 0\%            & 0\%            \\ \hline
\multirow{3}{*}{D\_M}           & Média            & 400               & 187               & 400               & 400               & 399               \\
                                & \textless 400    & 0\%            & 80,30\%           & 0\%            & 0\%            & 1,52\%            \\
                                & \textgreater 400 & 0\%            & 0\%            & 0\%            & 0\%            & 0\%            \\ \hline
\multirow{3}{*}{D\_F}           & Média            & 400               & 281               & 400               & 400               & 400               \\
                                & \textless 400    & 0\%            & 45,23\%           & 0\%            & 0\%            & 0,50\%            \\
                                & \textgreater 400 & 0\%            & 0\%            & 0\%            & 0\%            & 0\%            \\ \hline
\multirow{3}{*}{FS}             & Média            & 400               & 163               & 400               & 400               & 400               \\
                                & \textless 400    & 0\%            & 84,85\%           & 0\%            & 0\%            & 0\%            \\
                                & \textgreater 400 & 0\%            & 0\%            & 0\%            & 0\%            & 0\%            \\ \hline
\multirow{3}{*}{Fre}            & Média            & 400               & 125               & 400               & 400               & 400               \\
                                & \textless 400    & 0\%            & 97,49\%           & 0\%            & 0\%            & 0\%            \\
                                & \textgreater 400 & 0\%            & 0\%            & 0\%            & 0\%            & 0\%            \\ \hline
\multirow{4}{*}{\textbf{TOTAL}} & Média            & 400               & 207               & 400               & 400               & 400               \\
                                & Desvio Padrão    & 0                 & 129               & 0                 & 0                 & 5                 \\
                                & \textless 400    & 0\%            & 69,78\%           & 0\%            & 0\%            & 0,44\%            \\
                                & \textgreater 400 & 0\%            & 0\%            & 0\%            & 0\%            & 0\%            \\ \hline
\end{tabular}
\end{table}

% Please add the following required packages to your document preamble:
% \usepackage{multirow}
\begin{table}[H]
\centering
\caption{Teste com os Sonares Acoplados ao Veículo}
\label{teste_4}
\begin{tabular}{|c|c|ccccc|}
\hline
\textbf{Motores}                & \textbf{Medida}  & \textbf{Sensor 0} & \textbf{Sensor 1} & \textbf{Sensor 2} & \textbf{Sensor 3} & \textbf{Sensor 4} \\ \hline
\multirow{3}{*}{STP}            & Média            & 400               & 400               & 400               & 398               & 400               \\
                                & \textless 400    & 0,23\%            & 0\%            & 0\%            & 0,70\%            & 0\%            \\
                                & \textgreater 400 & 0\%            & 0\%            & 0\%            & 0\%            & 0\%            \\ \hline
\multirow{3}{*}{E\_L}           & Média            & 163               & 222               & 400               & 90                & 232               \\
                                & \textless 400    & 91,58\%           & 66,34\%           & 0\%            & 98,51\%           & 71,29\%           \\
                                & \textgreater 400 & 3,47\%            & 0\%            & 0\%            & 0\%            & 4,46\%            \\ \hline
\multirow{3}{*}{E\_M}           & Média            & 143               & 196               & 400               & 90                & 209               \\
                                & \textless 400    & 98,48\%           & 77,78\%           & 0\%            & 100\%          & 83,33\%           \\
                                & \textgreater 400 & 1,01\%            & 0\%            & 0\%            & 0\%            & 4,55\%            \\ \hline
\multirow{3}{*}{E\_F}           & Média            & 176               & 198               & 400               & 91                & 250               \\
                                & \textless 400    & 93,43\%           & 77,27\%           & 0\%            & 100\%          & 61,62\%           \\
                                & \textgreater 400 & 2,02\%            & 0\%            & 0\%            & 0\%            & 2,53\%            \\ \hline
\multirow{3}{*}{D\_L}           & Média            & 174               & 229               & 400               & 84                & 238               \\
                                & \textless 400    & 89,95\%           & 62,81\%           & 0\%            & 100\%          & 66,33\%           \\
                                & \textgreater 400 & 2,51\%            & 0\%            & 0\%            & 0\%            & 4,02\%            \\ \hline
\multirow{3}{*}{D\_M}           & Média            & 180               & 332               & 400               & 99                & 369               \\
                                & \textless 400    & 94,95\%           & 27,27\%           & 0\%            & 100\%          & 21,21\%           \\
                                & \textgreater 400 & 1,52\%            & 0\%            & 0\%            & 0\%            & 3,03\%            \\ \hline
\multirow{3}{*}{D\_F}           & Média            & 292               & 386               & 400               & 132               & 380               \\
                                & \textless 400    & 50,25\%           & 6,03\%            & 0\%            & 99,50\%           & 14,57\%           \\
                                & \textgreater 400 & 5,53\%            & 0\%            & 0\%            & 0,50\%            & 2,01\%            \\ \hline
\multirow{3}{*}{FS}             & Média            & 192               & 232               & 400               & 90                & 253               \\
                                & \textless 400    & 77,89\%           & 60,80\%           & 0\%            & 99,50\%           & 57,29\%           \\
                                & \textgreater 400 & 0,50\%            & 0\%            & 0\%            & 0,50\%            & 0,50\%            \\ \hline
\multirow{3}{*}{Fre}            & Média            & 181               & 252               & 400               & 86                & 274               \\
                                & \textless 400    & 81,82\%           & 53,03\%           & 0\%            & 100\%          & 56,06\%           \\
                                & \textgreater 400 & 2,02\%            & 0\%            & 0\%            & 0\%            & 2,53\%            \\ \hline
\multirow{4}{*}{\textbf{TOTAL}} & Média            & 233               & 287               & 400               & 160               & 302               \\
                                & Desvio Padrão    & 132               & 134               & 0                 & 128               & 128               \\
                                & \textless 400    & 66,80\%           & 42,45\%           & 0\%            & 78,62\%           & 42,50\%           \\
                                & \textgreater 400 & 1,83\%            & 0\%            & 0\%            & 0,10\%            & 2,33\%            \\ \hline
\end{tabular}
\end{table}

% % Please add the following required packages to your document preamble:
% \usepackage{multirow}
\begin{table}[]
\centering
\caption{My caption}
\label{my-label}
\begin{tabular}{|c|c|ccccc|}
\hline
\textbf{Motores}                & \textbf{Medida}  & \textbf{Sensor 0} & \textbf{Sensor 1} & \textbf{Sensor 2} & \textbf{Sensor 3} & \textbf{Sensor 4} \\ \hline
\multirow{3}{*}{STP}            & Média            & 399               & 399               & 400               & 398               & 399               \\
                                & \textless 400    & 0,37\%            & 0,37\%            & 0\%            & 0,75\%            & 0,37\%            \\
                                & \textgreater 400 & 0\%            & 0\%            & 0\%            & 0\%            & 0\%            \\ \hline
\multirow{3}{*}{E\_L}           & Média            & 205               & 276               & 400               & 154               & 383               \\
                                & \textless 400    & 84,85\%           & 45,96\%           & 0\%            & 95,45\%           & 10,10\%           \\
                                & \textgreater 400 & 8,08\%            & 0\%            & 0\%            & 2,53\%            & 3,54\%            \\ \hline
\multirow{3}{*}{E\_M}           & Média            & 148               & 326               & 400               & 220               & 331               \\
                                & \textless 400    & 99,49\%           & 27,78\%           & 0\%            & 73,23\%           & 32,32\%           \\
                                & \textgreater 400 & 0,51\%            & 0\%            & 0\%            & 1,52\%            & 3,54\%            \\ \hline
\multirow{3}{*}{E\_F}           & Média            & 204               & 363               & 400               & 247               & 365               \\
                                & \textless 400    & 83,33\%           & 14,65\%           & 0\%            & 56,06\%           & 21,21\%           \\
                                & \textgreater 400 & 3,54\%            & 0\%            & 0\%            & 3,03\%            & 4,04\%            \\ \hline
\multirow{3}{*}{D\_L}           & Média            & 203               & 293               & 400               & 179               & 367               \\
                                & \textless 400    & 83,42\%           & 41,21\%           & 0\%            & 81,41\%           & 14,57\%           \\
                                & \textgreater 400 & 5,03\%            & 0\%            & 0\%            & 2,51\%            & 0,50\%            \\ \hline
\multirow{3}{*}{D\_M}           & Média            & 210               & 397               & 400               & 285               & 400               \\
                                & \textless 400    & 88,24\%           & 0,98\%            & 0\%            & 63,73\%           & 0\%            \\
                                & \textgreater 400 & 2,94\%            & 0\%            & 0\%            & 6,86\%            & 0\%            \\ \hline
\multirow{3}{*}{D\_F}           & Média            & 316               & 400               & 400               & 400               & 400               \\
                                & \textless 400    & 47,47\%           & 0\%            & 0\%            & 0\%            & 0\%            \\
                                & \textgreater 400 & 6,06\%            & 0\%            & 0\%            & 0\%            & 0\%            \\ \hline
\multirow{3}{*}{FS}             & Média            & 247               & 325               & 400               & 211               & 355               \\
                                & \textless 400    & 63,64\%           & 30,30\%           & 0\%            & 73,74\%           & 17,17\%           \\
                                & \textgreater 400 & 2,02\%            & 0\%            & 0\%            & 4,04\%            & 0\%            \\ \hline
\multirow{3}{*}{Fre}            & Média            & 200               & 306               & 400               & 186               & 395               \\
                                & \textless 400    & 83,84\%           & 33,33\%           & 0\%            & 84,85\%           & 2,02\%            \\
                                & \textgreater 400 & 4,04\%            & 1,01\%            & 0\%            & 6,06\%            & 2,02\%            \\ \hline
\multirow{4}{*}{\textbf{TOTAL}} & Média            & 243               & 342               & 400               & 255               & 376               \\
                                & Desvio Padrão    & 128               & 111               & 0                 & 139               & 75                \\
                                & \textless 400    & 67,17\%           & 22,07\%           & 0\%            & 56,96\%           & 11,99\%           \\
                                & \textgreater 400 & 3,36\%            & 0,07\%            & 0\%            & 2,47\%            & 1,71\%            \\ \hline
\end{tabular}
\end{table}

% % Please add the following required packages to your document preamble:
% \usepackage{multirow}
\begin{table}[]
\centering
\caption{My caption}
\label{my-label}
\begin{tabular}{|c|c|ccccc|}
\hline
\textbf{Motores}                & \textbf{Medida}  & \textbf{Sensor 0} & \textbf{Sensor 1} & \textbf{Sensor 2} & \textbf{Sensor 3} & \textbf{Sensor 4} \\ \hline
\multirow{3}{*}{STP}            & Média            & 395               & 397               & 400               & 398               & 303               \\
                                & \textless 400    & 1,72\%            & 1,15\%            & 0\%            & 0,57\%            & 53,45\%           \\
                                & \textgreater 400 & 0\%            & 0\%            & 0\%            & 0\%            & 0\%            \\ \hline
\multirow{3}{*}{E\_L}           & Média            & 182               & 263               & 400               & 339               & 247               \\
                                & \textless 400    & 90,91\%           & 49,49\%           & 0\%            & 36,36\%           & 88,89\%           \\
                                & \textgreater 400 & 3,03\%            & 0\%            & 0\%            & 12,12\%           & 0\%            \\ \hline
\multirow{3}{*}{E\_M}           & Média            & 179               & 158               & 400               & 388               & 258               \\
                                & \textless 400    & 98,00\%           & 90\%           & 0\%            & 7,00\%            & 85,00\%           \\
                                & \textgreater 400 & 2,00\%            & 0\%            & 0\%            & 1,00\%            & 0\%            \\ \hline
\multirow{3}{*}{E\_F}           & Média            & 140               & 386               & 396               & 395               & 260               \\
                                & \textless 400    & 98,99\%           & 5,05\%            & 1,01\%            & 4,04\%            & 84,85\%           \\
                                & \textgreater 400 & 1,01\%            & 0\%            & 0\%            & 3,03\%            & 0\%            \\ \hline
\multirow{3}{*}{D\_L}           & Média            & 180               & 184               & 400               & 395               & 316               \\
                                & \textless 400    & 89,90\%           & 80,81\%           & 0\%            & 5,05\%            & 50,51\%           \\
                                & \textgreater 400 & 4,04\%            & 0\%            & 0\%            & 2,02\%            & 0\%            \\ \hline
\multirow{3}{*}{D\_M}           & Média            & 394               & 397               & 400               & 394               & 242               \\
                                & \textless 400    & 2,02\%            & 1,01\%            & 0\%            & 3,03\%            & 94,95\%           \\
                                & \textgreater 400 & 0\%            & 0\%            & 0\%            & 0\%            & 0\%            \\ \hline
\multirow{3}{*}{D\_F}           & Média            & 355               & 400               & 400               & 400               & 251               \\
                                & \textless 400    & 23,00\%           & 0\%            & 0\%            & 0\%            & 89,00\%           \\
                                & \textgreater 400 & 0\%            & 0\%            & 0\%            & 0\%            & 0\%            \\ \hline
\multirow{3}{*}{FS}             & Média            & 221               & 292               & 400               & 397               & 266               \\
                                & \textless 400    & 69,70\%           & 40,40\%           & 0\%            & 4,04\%            & 80,81\%           \\
                                & \textgreater 400 & 2,02\%            & 0\%            & 0\%            & 2,02\%            & 0\%            \\ \hline
\multirow{3}{*}{Fre}            & Média            & 217               & 181               & 400               & 245               & 229               \\
                                & \textless 400    & 76,77\%           & 75,76\%           & 0\%            & 75,76\%           & 87,88\%           \\
                                & \textgreater 400 & 9,09\%            & 0\%            & 0\%            & 5,05\%            & 0\%            \\ \hline
\multirow{4}{*}{\textbf{TOTAL}} & Média            & 263               & 303               & 400               & 374               & 267               \\
                                & Desvio Padrão    & 132               & 133               & 13                & 78                & 75                \\
                                & \textless 400    & 56,61\%           & 35,33\%           & 0,10\%            & 13,95\%           & 77,48\%           \\
                                & \textgreater 400 & 2,17\%            & 0\%            & 0\%            & 2,58\%            & 0\%            \\ \hline
\end{tabular}
\end{table}


 \pagebreak
 \newpage
 \section{Fluxograma da Rotina de Desvio de Obstáculos}
  \begin{figure}[H]
    \centering
    \includegraphics[width=0.8 \linewidth]{../../Imagens/ObstAvoid.png}
    \caption{Fluxograma da Rotina de Desvio de Obstáculos}
    \label{ObstAvoid}
  \end{figure}
 \pagebreak

 \section{Tabela Verdade}

\begin{table}[H]
\centering
\caption{Tabela Verdade}
\label{IA}
\begin{tabular}{|ccccc|c|}
\hline
 \textbf{Sensor 0} & \textbf{Sensor 1} & \textbf{Sensor 2} & 
 \textbf{Sensor 3} & \textbf{Sensor 4} & \textbf{Ação} \\ 
\hline
 0 & 0 & 0 & 0 & 0 & $FS$ \\ \hline
 0 & 0 & 0 & 0 & 1 & $E_L$ \\ \hline
 0 & 0 & 0 & 1 & 0 & $E_M$ \\ \hline
 0 & 0 & 0 & 1 & 1 & $E_M$ \\ \hline
 0 & 0 & 1 & 0 & 0 & $E_F$ \\ \hline
 0 & 0 & 1 & 0 & 1 & $E_F$ \\ \hline
 0 & 0 & 1 & 1 & 0 & $E_F$ \\ \hline
 0 & 0 & 1 & 1 & 1 & $E_F$ \\ \hline
 0 & 1 & 0 & 0 & 0 & $D_M$ \\ \hline
 0 & 1 & 0 & 0 & 1 & $D_M$ \\ \hline
 0 & 1 & 0 & 1 & 0 & $E_F$ \\ \hline
 0 & 1 & 0 & 1 & 1 & $E_M$ \\ \hline
 0 & 1 & 1 & 0 & 0 & $D_F$ \\ \hline
 0 & 1 & 1 & 0 & 1 & $E_F$ \\ \hline
 0 & 1 & 1 & 1 & 0 & $E_F$ \\ \hline
 0 & 1 & 1 & 1 & 1 & $E_F$ \\ \hline
 1 & 0 & 0 & 0 & 0 & $D_L$ \\ \hline
 1 & 0 & 0 & 0 & 1 & $Fre$ \\ \hline
 1 & 0 & 0 & 1 & 0 & $E_M$ \\ \hline
 1 & 0 & 0 & 1 & 1 & $E_M$ \\ \hline
 1 & 0 & 1 & 0 & 0 & $D_F$ \\ \hline
 1 & 0 & 1 & 0 & 1 & $E_F$ \\ \hline
 1 & 0 & 1 & 1 & 0 & $E_F$ \\ \hline
 1 & 0 & 1 & 1 & 1 & $E_F$ \\ \hline
 1 & 1 & 0 & 0 & 0 & $D_M$ \\ \hline
 1 & 1 & 0 & 0 & 1 & $D_M$ \\ \hline
 1 & 1 & 0 & 1 & 0 & $D_M$ \\ \hline
 1 & 1 & 0 & 1 & 1 & $Fre$ \\ \hline
 1 & 1 & 1 & 0 & 0 & $D_F$ \\ \hline
 1 & 1 & 1 & 0 & 1 & $D_F$ \\ \hline
 1 & 1 & 1 & 1 & 0 & $D_F$ \\ \hline
 1 & 1 & 1 & 1 & 1 & $E_F$ \\ \hline
\end{tabular} 
\end{table}



\section{Sistemas de Tempo Real}
São sistemas computacionais que dependem não somente da correção dos dados computados, mas que sejam obtidos dentro de um intervalo 
de tempo 
pré determinado, que pode ser maior ou menor de acordo com a aplicação.
Na literatura, este período em que se espera que a resposta do sistema se dê é denominado \textit{deadline}.
Sistemas de tempo real podem ser classificados em dois tipos: \textit{soft} ou \textit{hard}.

Sistemas \textit{soft} são menos restritivos, tolerando eventuais perdas de \textit{deadline}; 
ao contrário dos sistemas \textit{hard}, em que estas perdas não são aceitáveis.  

Algumas características típicas, apesar de não obrigatórias, de sistemas de tempo real são limitações com relação ao tamanho, propósito específico e 
baixo custo \cite{silberschatz}.

\section{CRC}
Método de detecção de erros aleatórios, isto é, de dados corrompidos ao longo do processo de transmissão ou armazenamento da informação por exemplo 
por ruídos, mas não por um agente \textquoteleft inteligente\textquoteright{} externo que modifique os dados transmitidos, tal qual um 
\textit{malware} \cite{stigge}.

Consiste essencialmente em uma divisão polinomial \cite{stigge}, logo, pode ser implementado em \textit{hardware}, utilizando-se apenas registradores 
de deslocamento com conexões realimentadas \cite{peterson}, assim como em \textit{software}. 
Em suma, trata-se de acrescentar à mensagem digital original um sufixo, que tem seu valor definido por operações realizadas em função da mensagem 
binária que se intenta enviar e de um polinômio gerador.
Para o \textit{transceiver} nRF24L01+, dois polinômios geradores são utilizados: Eq. \ref{CRC_1} quando o dado cíclico adicionado é de 1 
\textit{byte} , e Eq. \ref{CRC_2}, para 2 \textit{bytes} \cite{nRF}.

Para uma descrição completa de como é implementado este método, vide \cite{stigge,peterson}.

\begin{equation}
\label{CRC_1}
G(X) = X^8 + X^2 + X + 1 
\end{equation}

\begin{equation}
\label{CRC_2}
G(X) = X^{16} + X^{12} + X^5 + 1 
\end{equation}

\section{PWM}
A modulação por largura de pulso é uma técnica de modulação que consiste em amostrar e codificar o sinal correspondente à mensagem na largura de um 
trem de pulsos de amplitude fixa, i.e. cada amostra da mensagem é convertida em um pulso retangular cuja duração expressa a amplitude do sinal 
modulante.

Um modulador PWM pode ser implementado utilizando-se um circuito comparador não inversor cuja entrada inversora liga-se à saída de um gerador de 
ondas tipo dente de serra (\textit{trailing edge modulation}), dente de serra invertida (\textit{leading edge modulation}  ou triangular 
(\textit{modulation on both edges}) e na entrada não inversora, o sinal modulante. 
Desta forma, quando a tensão da mensagem excede a amplitude da onda gerada, observa-se um sinal alto na saída, caso contrário, baixo, conforme 
ilustra a Fig. \ref{pwm_modulation_types}.

  \begin{figure}[H] %% TODO fonte: https://en.wikipedia.org/wiki/File:Three_PWM_types.svg
    \centering
    \includegraphics[width=0.7\linewidth]{../../Imagens/PWM_modulation_types.png}
    \caption{Três tipos de modulação PWM: \textit{trailing edge}, \textit{leading edge} e  \textit{both edges}, de cima pra baixo, respectivamente.}
    \label{pwm_modulation_types}
  \end{figure}
  
para uma descrição detalhada do circuito e simulações vide \cite{pwm_modulator}
  
  %TODO circuito corretor de erros:
%    First, the error amplifier accommodates feedback of the output PWM waveform in order to correct for any errors in the
% output voltage introduced by the comparator. Second, it adds a dc offset to the input voltage so that negative input voltages can be accommodated 
% by 
% the circuit.
%  
%  Because the supply voltage of the comparator directly impacts the output voltage, PWM circuits without feedback have no power supply rejection. In 
% this TI Design, the error amplifier acts as an inverting amplifier to the input signal, shown as a dc-coupled source VIN. By including the 
% comparator 
% inside the feedback loop of the error amplifier, and adding integration capacitor C1, the error amplifier now directly controls the average output 
% voltage
% 
% 
