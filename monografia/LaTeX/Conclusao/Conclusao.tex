\chapter{Conclusão}

Com o processamento dos dados lidos pelos sonares, foi possível reduzir os efeitos
de falsas detecções de obstáculos. Quanto à decisão entre optar por intervalos estáticos
ou dinâmicos de medição, foi comprovado que há de fato menor confiabilidade nas medidas
obtidas utilizando-se intervalos dinâmicos no que concerne a obstáculos mais próximos
ao veículo enquanto que, para obstáculos mais longínquos, a diferença não é tão notória.
De forma que a diminuição no tempo de resposta dos sonares acaba sendo maior do que
o aumento da incidência de erros de leitura, o que faz com que seja melhor se sujeitar a
obtenção de dados menos confiáveis porém mais atualizados, pois é possível tratar esses
eventuais erros utilizando métodos estatísticos como a teoria Dempster-Shafer, conforme foi feito em
 \citeonline{Artigo_11}, ou inferência Bayesiana, conforme \citeonline{Artigo_7}.
 
No entanto, o problema da interferência dos motores nos sonares ainda não foi
resolvido e constitui um contratempo grave ao bom funcionamento do veículo e que,
portanto, precisa ser resolvido para poder dar prosseguimento ao projeto. É preciso reduzir
o impacto das vibrações mecânicas dos BLDC no circuito que liga a placa de prototipação
aos sonares. Uma abordagem que poderia mitigar o fenômeno seria fazer uma placa de
circuito impresso, na qual os sonares seriam soldados diretamente, eliminando totalmente
a utilização de fios para fazer contato entre os dispositivos.

Quanto ao subsistema de comunicação, por se tratar de um módulo de baixo
consumo de potência, há uma limitação no alcance do dispositivo. No entanto, apesar
de perceptível, essa restrição não constituiu um problema nos testes feitos no veículo,
que manteve a comunicação funcionando mesmo em distâncias de aproximadamente 10
metros as custas de um aumento na perda de pacotes.

O subsistema de navegação ainda tem muito o que melhorar pois, como há uma
gama de apenas 8 possíveis medidas de evasão, a resposta do veículo é pouco adaptável ao
ambiente externo, de modo que muitas vezes o comportamento adotado é muito suave,
causando colisões, ou muito brusco, causando desvios de rota desnecessários. Utilizando-se
estratégias de desvio de obstáculos em que a rotação dos motores é obtida por meio de
um controlador PID, ou Virtual Force Field Method, conforme \citeonline{2016_artigo_2}.




% Com o processamento dos dados lidos pelos sonares, foi possível reduzir os efeitos de falsas detecções de obstáculos.
% Quanto à decisão entre optar-se por intervalos estáticos ou dinâmicos de medição, foi comprovado que há de fato menor confiabilidade nas medidas 
% obtidas utilizando-se intervalos dinâmicos no que concerne a obstáculos mais próximos ao veículo enquanto que para obstáculos mais longínquos a 
% diferença não é tão notória.
% De forma que a diminuição no tempo de resposta dos sonares acaba sendo maior do que o aumento da incidência de erros de leitura, o que faz com que 
% seja melhor se sujeitar a obtenção de dados menos confiáveis porém mais atualizados, pois é possível tratar esses eventuais erros utilizando 
% métodos 
% estatísticos como a teoria Dempster-Shafer, conforme foi feito em \cite{Artigo_11}, ou inferência Bayesiana, conforme \cite{Artigo_7}.
% 
% No entanto, o problema da interferência dos motores nos sonares ainda não foi resolvido e constitui um contratempo grave ao bom funcionamento do 
% veículo e que, portanto, precisa ser resolvido para poder dar prosseguimento ao projeto. 
% É preciso reduzir o impacto das vibrações mecânicas dos BLDC no circuito que liga a placa de prototipação aos sonares.
% Uma abordagem que poderia mitigar o fenômeno seria fazer uma placa de circuito impresso, na qual os sonares seriam soldados diretamente, eliminando 
% totalmente a utilização de fios para fazer contato entre os dispositivos.
% 
% Quanto ao subsistema de comunicação, por se tratar de um módulo de baixo consumo de potência, há uma limitação no alcance do dispositivo. 
% No entanto, apesar de perceptível, essa restrição não constituiu um problema nos testes feitos no veículo, que mantinha a comunicação funcionando 
% mesmo em distâncias de aproximadamente 10 metros as custas de um aumento na perda de pacotes.
% 
% O subsistema de navegação ainda tem muito o que melhorar pois, como há uma gama de apenas 8 possíveis medidas de evasão, a resposta do 
% veículo é pouco adaptável ao ambiente externo, de modo que muitas vezes o comportamento adotado é muito suave, causando colisões, ou muito brusco, 
% causando desvios de rota desnecessários.
% Utilizando-se estratégias de desvio de obstáculos em que a rotação dos motores é obtida por meio de um controlador PID, ou \textit{Virtual Force 
% Field Method}, conforme \cite{2016_artigo_2}.
