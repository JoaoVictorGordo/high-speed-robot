 \chapter{Introdução}
O presente trabalho consiste na implementação de um veículo autônomo de alta
velocidade, desenvolvido sob o paradigma de sistemas puramente reativos, cuja finalidade
é servir como uma plataforma terrestre de testes de estratégias de navegação de Veículos
Aereos Não Tripulados, VANTs, dentro da arquitetura MOSA, \textit{Mission Oriented Sensor
Array}. Isto é, na arquitetura MOSA, o escopo deste projeto consiste na porção relativa
ao controlador de navegação, que é responsável por cuidar da integridade da aeronave, enquanto a parte
orientada à missão é o que pretende-se testar utilizando o veículo. A necessidade de
se desenvolver um veículo terrestre para testar estratégias de voo se dá pelo fato de que este é
um sistema embarcado crítico cuja falha pode resultar em acidentes graves, logo, caso o
comportamento do MOSA desenvolvido seja extensivamente testado em um ambiente livre
de riscos utilizando-se o veículo, há menos risco de eventuais danos provocados por mal
funcionamento. 
Como o intuito do projeto é simular um \textit{drone} para testar novos algoritmos de navegação para que 
se tenha uma segurança maior no funcionamento desejado do \textit{software} antes de serem feitos os devidos 
testes utilizando de fato uma aeronave, o veículo desenvido opera em velocidades similares às de VANTs,
podendo atingir até \unitfrac[100]{km}{h}.
Maiores detalhes acerca da arquitetura MOSA serão abordados na seção
de Embasamento Teórico.

O projeto pode ser dividido em 4 subsistemas: percepção, locomoção, comunicação
e navegação. A percepção é feita através de uma matriz de cinco sensores ultrassônicos
de baixo custo, que são disparados simultaneamente a fim de reduzir o tempo gasto
com obtenção dos dados do ambiente externo. Como consequência disso, obtém-se dados
menos confiáveis pois são intensificados efeitos colaterais como o crosstalk, que não pode ser
eliminado processando medidas consecutivas, conforme \citeonline{2016_artigo_1}. Além disso, buscou-se verificar
o quão significativo é o impacto na confiabilidade das medidas caso os ciclos de leitura dos
sonares seja determinado de acordo com as circunstâncias do meio, isto é, caso não haja
um intervalo de medição fixo e pré-determinado, denominado intervalo estático de agora
em diante, de modo que o período gasto na percepção estivesse atrelado apenas ao tempo
gasto pelo sonar que detectou o obstáculo mais distante do veículo, denominado intervalo
dinâmico. Ao adotar intervalos dinâmicos, reduz-se o tempo entre leituras consecutivas,
o que pode fazer com que vibrações residuais da membrana responsável por emitir as
ondas acústicas sensibilize o elemento receptor provocando falsas leituras, conforme \citeonline{jones}.
Maiores detalhes acerca do princípio de funcionamento dos sensores ultrassônicos, assim
como vantagens e desvantagens deste sensor encontram-se na seção de Materiais desta
monografia.

Para a locomoção do veículo, são utilizados dois motores \textit{brushless} de corrente
contínua, de modo que a tração é dianteira. Logo, o veículo faz curvas em razão da diferença
de velocidade entre os BLDC e, por se tratar de motores usualmente utilizados em VANTs,
atinge velocidade de até  \unitfrac[80]{km}{h}.
Como o controle de motores sem escovas não é trivial de
ser feito, foram utilizados controladores eletrônicos de velocidade, ESCs, como elemento
intermediador entre o microcontrolador e os motores. Desta forma, todo o tratamento de
mais baixo nível no que concerne o acionamento das bobinas dos motores foi designado
a este dispositivo, enquanto o microcontrolador se responsabiliza por fornecer aos ESCs
a velocidade que pretende-se obter do seu respectivo motor.
O sistema de comunicação tem o propósito de: fornecer feedback dos dados internos
do veículo, como percepção e velocidade dos motores; regular o acionamento dos motores,
pois repassa ao microcontrolador informações de um botão de segurança, que emite sinais
ao veículo, autorizando-o ou não a navegar; e, finalmente, de servir como uma interface de
comandos na qual é possível controlar remotamente o veículo.
\footnote{Entenda-se por veículo não só a parte física como rodas, motores e dispositivos mas, também, todo o \textit{software} responsável por gerir 
o automóvel.}.

Quanto ao subsistema de navegação, temos que este consiste na inteligência artificial
do veículo, isto é, qual é a estratégia utilizada para efetuar o desvio de obstáculos com
base nos dados colhidos pelos sonares. A técnica de desvio de obstáculos implementada
é a mais simples possível: trata-se de uma função que mapeia as leituras dos sonares -
categorizadas em três regiões: perigo, atenção e distante - em um par de velocidades
angulares - categorizados quanto a intensidade em forte, médio e leve, e quanto à direção
em esquerda ou direita - a serem impostas aos motores de acordo com qual a combinação
de regiões lida pela matriz de sonares, conforme a Eq. \ref{nav}. A tabela verdade que relaciona
domínio e contradomínio da função de desvio de obstáculos consta no Apêndice, vide Tabela \ref{IA}.

\begin{equation}
\label{nav}
T:R^5 \rightarrow S
\end{equation}
$$\quad \textrm{em que:} \quad R=\{distante, perigo, atenção\}, \quad S=\{E_L, E_M, E_F, D_L, D_M, D_F\}$$

Note que cada um dos elementos do conjunto S são pares ordenados que representam a
velocidade angular dos motores.


\begin{table}[H]
\centering
\caption{Tabela Verdade}
\label{IA}
\begin{tabular}{|ccccc|c|}
\hline
 \textbf{Sensor 0} & \textbf{Sensor 1} & \textbf{Sensor 2} & 
 \textbf{Sensor 3} & \textbf{Sensor 4} & \textbf{Ação} \\ 
\hline
 0 & 0 & 0 & 0 & 0 & $FS$ \\ \hline
 0 & 0 & 0 & 0 & 1 & $E_L$ \\ \hline
 0 & 0 & 0 & 1 & 0 & $E_M$ \\ \hline
 0 & 0 & 0 & 1 & 1 & $E_M$ \\ \hline
 0 & 0 & 1 & 0 & 0 & $E_F$ \\ \hline
 0 & 0 & 1 & 0 & 1 & $E_F$ \\ \hline
 0 & 0 & 1 & 1 & 0 & $E_F$ \\ \hline
 0 & 0 & 1 & 1 & 1 & $E_F$ \\ \hline
 0 & 1 & 0 & 0 & 0 & $D_M$ \\ \hline
 0 & 1 & 0 & 0 & 1 & $D_M$ \\ \hline
 0 & 1 & 0 & 1 & 0 & $E_F$ \\ \hline
 0 & 1 & 0 & 1 & 1 & $E_M$ \\ \hline
 0 & 1 & 1 & 0 & 0 & $D_F$ \\ \hline
 0 & 1 & 1 & 0 & 1 & $E_F$ \\ \hline
 0 & 1 & 1 & 1 & 0 & $E_F$ \\ \hline
 0 & 1 & 1 & 1 & 1 & $E_F$ \\ \hline
 1 & 0 & 0 & 0 & 0 & $D_L$ \\ \hline
 1 & 0 & 0 & 0 & 1 & $Fre$ \\ \hline
 1 & 0 & 0 & 1 & 0 & $E_M$ \\ \hline
 1 & 0 & 0 & 1 & 1 & $E_M$ \\ \hline
 1 & 0 & 1 & 0 & 0 & $D_F$ \\ \hline
 1 & 0 & 1 & 0 & 1 & $E_F$ \\ \hline
 1 & 0 & 1 & 1 & 0 & $E_F$ \\ \hline
 1 & 0 & 1 & 1 & 1 & $E_F$ \\ \hline
 1 & 1 & 0 & 0 & 0 & $D_M$ \\ \hline
 1 & 1 & 0 & 0 & 1 & $D_M$ \\ \hline
 1 & 1 & 0 & 1 & 0 & $D_M$ \\ \hline
 1 & 1 & 0 & 1 & 1 & $Fre$ \\ \hline
 1 & 1 & 1 & 0 & 0 & $D_F$ \\ \hline
 1 & 1 & 1 & 0 & 1 & $D_F$ \\ \hline
 1 & 1 & 1 & 1 & 0 & $D_F$ \\ \hline
 1 & 1 & 1 & 1 & 1 & $E_F$ \\ \hline
\end{tabular} 
\end{table}



A título de ilustração, para deixar mais claro o propósito deste projeto, podemos
supor que um MOSA esteja sendo desenvolvido com o propósito de percorrer uma rota
previamente selecionada, utilizando-se de GPS, acelerômetro e magnetômetro para a
missão. Este poderia ser testado no veículo antes de ser colocado para voar, de forma que
o veículo autônomo desempenharia a função da aeronave, sendo responsável por desviar
de eventuais obstáculos quando necessário, e quando o caminho estiver livre, o controle
do veículo seria repassado ao MOSA, mas ainda sob sua supervisão. Dessa forma, quando
em posse do veículo, o MOSA desempenharia sua missão, que no exemplo citado seria
percorrer \textit{checkpoints}.