
% resumo em português
\setlength{\absparsep}{18pt} % ajusta o espaçamento dos parágrafos do resumo
\begin{resumo}
Foi implementado um veículo de navegação autônoma de alta velocidade, cuja técnica de desvio de obstáculos é puramente reativa, com o propósito de 
servir como uma plataforma de testes de navegação de VANTs desenvolvidos na arquitetura MOSA, que desacopla a porção crítica da não crítica do 
sistema embarcado de tempo real.
A percepção do veículo consiste em uma matriz de cinco sensores ultrassônicos, o subsistema de locomoção funciona por tração dianteira feita por 
motores de corrente contínua sem escovas, o subsistema de comunicação é feito via rádio com as funcionalidades de: obtenção remota de dados 
internos do veículo durante a navegação autônoma, de uma interface de comando que também possibilitasse a mudança de variáveis internas do veículo sem 
a necessidade de reprogramar o microcontrolador e, o mais importante, de dar ou não ao veículo o aval para navegar com base no \textit{status} de um 
botão de segurança acoplado a outro rádio.
A estratégia de navegação consiste em classificar as leituras de cada um dos sonares em três regiões: distante, atenção e próxima; com base em qual 
região encontra-se cada sonar, toma-se a decisão de qual a medida de evasão a ser realizada, isto é, foi gravada na memória do microcontrolador uma 
tabela que correlaciona cada uma das combinações de regiões a um par ordenado de velocidades que deve ser imposta aos motores para desviar do 
obstáculo em questão.
Contudo, as leituras obtidas pelos sonares foram fortemente afetadas em decorrência de vibrações mecânicas dos motores, que causam leituras 
espúrias, o que, por sua vez, provoca a adoção de comportamentos errados pelo veículo.

 \textbf{Palavras-chaves}: navegação autônoma. sistemas reativos. sensores ultrassônicos. desvio de obstáculos.
\end{resumo}

 \begin{resumo}[Abstract]
  \begin{otherlanguage*}{english}
 A high speed autonomous mobile robot with a purely reactive obstacle avoidance technique was implemented. Its purpose is to serve as a navigation 
test platform for UVAs in MOSA architecture which decouples the critical from the non-critical portion in the embedded system at real time. The 
vehicle’s perception is composed of an array of five ultrasonic sensors, the locomotion subsystem operates by front-wheel drive made through brushless 
DC motors, the communication subsystem is made through radio and its functions are: obtaining remote internal data during the autonomous navigation as 
well as a command interface which could also enable changes on the internal variables of the vehicle without reprograming the microcontroller and, 
most important, endorse the vehicle’s navigation based on the status of a security button attached to another radio. The navigation strategy consists 
in classifying the readings from each of the sonars in three regions: distant, warning and close. Based on the region where the sonar is located, a 
decision is made on which evasion measure is to be carried out, i.e., in the microcontroller’s memory was recorded a table that correlates each 
combinations of regions to an ordered pair of speed which has to be imposed to the motors so they are able to deviate from the obstacle. However, the 
readings obtained by the sonars were strongly affected by the mechanical vibrations from the motors, causing spurious readings which led the vehicle 
to adopt wrong behaviors.

    \vspace{\onelineskip}
  
    \noindent 
    \textbf{Key-words}: autonomous navigation. reactive systems. ultrasonic sensors. obstacle avoidance
  \end{otherlanguage*}
 \end{resumo}
