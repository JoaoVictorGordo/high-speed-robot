\chapter{Embasamento Teórico}
\section{Sistemas Reativos}
\epigraph{ 
	  \textit{``Representações explícitas e modelos atrapalham. No fim das contas, a melhor representação do mundo é ele mesmo.''} 
	 }
	  { Brooks, R.A \cite{brooks} - tradução livre -} 
	  
%   \begin{flushright}
% 	  \textit{``Representações explícitas e modelos atrapalham.\\
% 	  No fim das contas, a melhor representação \\
% 	  do mundo é ele mesmo.", Brooks, R.A. 
% 	  \cite{brooks} \footnote{tradução livre}}
%   \end{flushright}

\subsection{Paradigma Reativo como Robótica Bioinspirada} 

De acordo com Rodney Brooks \cite{brooks}, para o desenvolvimento da inteligência no seu sentido mais estrito e genuíno, são condições 
suficientes que o indivíduo, que denominaremos agente, tenha as seguintes faculdades: mobilidade dentro de um ambiente dinâmico no qual esteja 
inserido, percepção do que se passa nas suas adjacências e, por fim, manutenção da própria sobrevivência.
Em suma, habilidades como o raciocinar, comunicar-se, gerar conhecimento nada mais são do que comportamentos complexos, consequências simples do fato 
de existirmos e termos o poder de reagir dentro do meio em que vivemos.

Para aprofundarmos a discussão e esclarecermos como se daria esse processo de aprimoramento dos agentes, é preciso fornecer uma  definição 
mais rigorosa do termo ``comportamento''.
Numa tradução livre: ``comportamentos são mapeamentos diretos de informações sensoriais recebidas em  
padrões de ações motoras, desempenhadas para se cumprir uma tarefa. Matematicamente, seria uma função transferência que transforma dados dos 
sensores em comandos para os atuadores'' \cite{murphy}.

Nos animais, a transformação de percepção em ação está subordinada à existência de estímulos específicos, de natureza interna ou externa 
ao agente, que podem ser entendidos como sinais de controle que que permitem ou inibem determinados comportamentos \cite{murphy}.
A título de ilustração: ao avistar uma presa - informação sensorial - o predador somente a ataca - comportamento - caso esteja com fome - estímulo 
interno; ou quando afastamos a mão - comportamento - ao tocarmos uma panela quente - a informação sensorial seria a temperatura da panela enquanto o 
estímulo externo é o fato de que ela excede uma dada temperatura.

Com base em estudos da etologia, o comportamentos dos animais podem ser inatos ou aprendidos, e a sua inteligência pode ser decomposta verticalmente 
em camadas de comportamentos, cada qual acessa os sensores e atuadores do agente de maneira independente das demais \cite{murphy}.
Isto é, o indivíduo inicia sua existência com um conjunto comportamentos inatos de autopreservação mas, ao longo da sua vida, outros 
novos vão surgindo, podendo: refinar comportamentos pré-existentes, negá-los (completamente) ou agregar-se a eles sem produzir conflitos, 
i.e. trabalhando paralelamente com os que lhe são ancestrais.
Desta forma, os dois primeiros casos podem ser entendidos como uma reutilização de camadas inferiores da inteligência, enquanto o 
último consiste na adição de mais uma camada.

\subsection{Características}
Por estar embasado em ideias da etologia discutidas na seção anterior, o paradigma reativo simplesmente desconsidera a etapa de planejamento 
existente na tríade 'percepção, planejamento, ação', que sumariza o ciclo de tarefas realizadas por um sistema sob o paradigma hierárquico 
\cite{murphy,roseli}.
Em suma, os comportamentos se dão de acordo com o que o agente percebe que está acontecendo no seu entorno, não são feitas modelagens ou 
representações do ambiente externo, apenas medições locais e orientadas a comportamentos.

Em decorrência da exclusão da etapa de planejamento, robôs desenvolvidos sob o paradigma reativo costumam ser simples e apresentarem 
respostas rápidas \cite{roseli}.
Com boas práticas de projeto é possível construir uma robô com: alta coesão, pois comportamentos podem ter acesso direto aos sensores de que 
necessitam para tomar suas decisões, o que possibilita um alto grau de independência em relação a operações e dados externos entre diferentes módulos 
ou subsistemas do robô; e baixo acoplamento, pois comportamentos são independentes entre si e, portanto, há pouca ou nenhuma dependência de ligações 
e 
interfaces externas a um dado módulo \cite{murphy}.

% TODO: \subsection{Arquitetura de Subsumpção}

\section{Arquitetura MOSA}

Arquitetura que propõe dividir o sistema aéreos de navegação autônoma em dois módulos: aeronave e MOSA \cite{mosa_proposal}.

O primeiro constitui a porção crítica do sistema embarcado, i.e. segmento cuja falha pode resultar em ao menos um dos seguintes desastres: morte 
ou lesão de pessoas; destruição ou danos a propriedades, patrimônios ou equipamentos; danos ambientais \cite{safety}.
Veículos Aéreos Não Tripulados, VANTs, apresentam a tolerância de um erro grave a cada $10^5$ ou $10^9$ horas de voo \cite{hard}, o que as 
caracteriza 
como sistemas computacionais de tempo real do tipo \textit{hard}. 
Maiores esclarecimentos acerca destes jargões podem ser encontrados no apêndice.

O segundo corresponde à parte não crítica à segurança, encarregada do controle da navegação e, por conseguinte, da determinação da maior parte dos 
parâmetros de voo. É caracterizado como um conjunto de sensores inteligentes capazes de cumprir uma missão específica, ou seja, 
existe uma relação biunívoca entre missão e MOSA, dado que ele consiste no melhor arranjo de sensores para o cenário em questão. Neste contexto, a 
aeronave é vista unicamente como o meio de transporte dos sensores, enquanto que o módulo MOSA constituiria o \textquoteleft cérebro\textquoteright{}  
da plataforma, responsável pelo cumprimento da missão e por guiar a aeronave até a sua realização.

%  \begin{figure}[h]
%   \includegraphics[scale=0.5]{./Resources/MOSA.png}
%   \caption{MOSA} \label{MOSA}
%  \end{figure}

No entanto, como a aeronave é o elemento responsável pela garantia da segurança, cabe a ela acatar ou não os comandos do MOSA. E pode, inclusive, 
optar por readaptar a missão em tempo de voo para se ajustar ao cenário, o que inclui a seleção dos sensores que melhor se encaixam na dada 
conjuntura.

Isso se dá através de uma matriz de reconfiguração dinamicamente adaptável denominada \textit{Knowledge Based Framework}, seu papel é comparável à 
expertise de um piloto.
Ou seja, um elemento inteligente capaz de escolher o melhor serviço  a ser executado com base em regras e critérios de seleção como resposta em tempo 
real, segurança, performance.


% diferentes missões, definidas pelo mosa e diferentes sensores, podem ser integrados, possibilitando a escolha do melhor arranjo de sensores que se 
% ajuste ao cenário de utilização do sistema. este é o mecanismo básico do mosa, fazendo com que a missão possa ser adaptativa. durante uma missão, 
% com 
% base em uma matriz de reconfiguração o vant pode se adaptar dinamicamente às características da missão, escolhendo os sensores que melhor se 
% encaixam 
% dependendo da situação. além do hardware, um sistema mosa deve contemplar também o software capaz de realizar uma missão, comunicar-se com todos os 
% sensores que o compõe, enviar e receber dados para a aeronave .
% 
% o sistema mosa deve ser capaz de determinar se a missão prevista pode ou não ser realizada.
% 
% the aircraft can, for safety reasons, not follow the flight sensors
% commands, eventually terminating the flight
% 
% flight sensors can provide data for mission controllers but aircraft
% % % % % % % % controllers must not use data provided by mission sensors
