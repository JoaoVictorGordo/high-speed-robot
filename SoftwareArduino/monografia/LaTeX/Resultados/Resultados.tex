\chapter{Resultados}

\section{Testes Unitários}

\subsection{Intervalo de Medição Dinâmico}
A fim de verificar a viabilidade de impregar-se intervalos dinâmicos na percepão do veículo, este foi disposto em diferentes posições nas quais todos 
os sonares apresentavam obstáculos visíveis, foram realizadas medições dinâmicas e estáticas, esta última com três intervalos distintos: 30ms, 45ms e 
60ms. Em cada uma das medições foram armazenadas 150 leituras ininterruptas armazenadas em um \textit{buffer} local que, ao ser preenchido, cessava 
o acionamento dos sonares e imprimia os dados via interface serial antes de começar a próxima medição. 
Nos dados obtidos em cada uma dessas medições foram feitos cálculos de média e desvio padrão sujos e limpos\footnote{foram consideradas medidas 
limpas aquelas cuja métrica foi calculada descartando-se as leituras em que o respectivo sonar não encontrou o obstáculo}, contagem dos resultados 
que desviassem em mais do que quatro vezes o desvio padrão em relação à média limpa das leituras.

\subsection{Leituras Espúrias nos Sonares}
%% testes feitos com o veículo parado nas quais surgiam valores conflitantes
Ao serem feitos testes unitários no \textit{software} de controle dos sensores ultrassônicos, notou-se a existência de leituras espúrias, sobretudo 
em condições em que não havia obstáculos dentre da região visível do dispositivo.
A fim de documentar o problema, posiciou-se o veículo num ângulo e distância conhecidos em relação a um obstáculo para em seguida medir as respostas 
dos sonares; foram feitos diversos testes variando o ângulo e a distância dos obstáculos, incluindo o caso de não haver obstáculos dentro do 
alcance de detecção dos sensores ultrassônicos, variou-se também o \textit{blanking time}.
Os valores coletados constam nas Tabelas \ref{espurias-1}, \ref{espurias-2}.

Podemos notar que não há uma correlação entre o \textit{dead time} e a aparição das leituras espúrias, o que motivou a adoção de intervalos dinâmicos 
de leitura, isto é, o \textit{blanking time} corresponde ao intervalo do sonar que demorou mais a receber uma resposta no pino de \textit{echo} ou um 
 valor máximo pré-determinado, caso não haja obstáculos ao alcance de algum deles.
A solução concebida para remediar este problema foi basear o comportamento a ser adotado na média aritmética das últimas 5 leituras, excluindo os 
valores extremos.

\section{Testes de Integração}
\subsection{Interferência dos BLDC nos Sonares}
Notou-se que em determinadas condições o robô ia de encontro ao obstáculo ao invés de efetuar o desvio. 
Após serem analisados os dados dos sensores nessas circunstâncias, observou-se a existência de ruídos em sensores específicos, que causavam a adoção 
destes comportamentos errados.

Ao perceber o problema, novos testes foram engendrados a fim de descobrir a natureza da falha.
As hipóteses concebidas eram as seguintes: \textit{crosstalk}, curto circuito entre pinos do Arduino, falha na lógica do \textit{software} ou 
interferência dos motores nos sensores.

Para eliminar a possibilidade de que uma das portas da placa de prototipação estivesse interferindo na outra de alguma maneira, mudou-se a 
ligação dos sensores ultrassônicos e o problema se manteve.
O mesmo foi feito no \textit{software}, i.e. foi alterada a disposição dos sensores ultrassônicos no código. 
Especificamente falando, foram trocados os parâmetros que correlacionam a ligação física do pino de \textit{echo} dos dispositivos à sua variável 
correspondente na matriz de estruturas do tipo \textit{sensor\_t}, denominada no programa por USS, e, mais uma vez, o defeito persistiu.
Adicionalmente a essa modificação no \textit{software}, foram feitas medições com os motores desligados, nas quais a falha em questão não ocorreu, 
evidenciando que a natureza do problema não era do código.

O teste seguinte consistiu em desacoplar os sensores ultrassônicos da carcaça do veículo e ligar os motores com os sonares sendo segurados na mão, o 
no resultado observou-se a desaparição das leituras espúrias, confirmando a hipótese de que os BLDC causavam a distorção na percepção do robô.
Em vista disso, supôs-se que a natureza da interferência seria em razão da proximidade entre os dispositivos, como interferência eletromagnética nos 
pinos de \textit{echo} dos sonares ou então de ondas acústica na banda de funcionamento do sensor, e optou-se por erguer os sensores a uma altura 
na qual não houvesse interferência suficiente a ponto de provocar erros de medição.

Após terem sido feitas as devidas modificações, foi engendrado um novo teste a fim de constatar se com a nova disposição o problema havia sido 
sanado: cada uma das combinações de velocidades dos motores foi mantida por dez segundos enquanto os sonares faziam as medições.
Os dados obtidos, vide Tabela \ref{tabela-interf}, apontaram que o problema persistia; enquanto era realizado o teste, notou-se que os fios oscilavam
 em decorrência da vibração dos motores, o que levou a cogitar a hipótese de que essa seria de fato a natureza da interferência entre os sensores 
ultrassônicos e os motores.

%%% ----------------------------------------------------------------------------------------------------------------------------------------------
\cite{jones}(página 149) vibração residual do trigger pode causar falsas leituras no echo, talvez esse seja o problema no caso de detecções dinâmicas 
deixado de lado.

\cite{siegwart} (páginas 126-127) talvez estendendo o \textit{blanking time} consigamos manter as detecções dinâmicas. Como fazer isso:
\begin{itemize}
 \item colocar um delay antes de entrar no loop
 \item mexer diretamente no myPulsein() para garantir medidas abaixo de 5 cm não sejam processadas trabalhando nos IFs. 
\end{itemize}

 While comparison of consecutive readings is an efficient way for rejecting external
 erroneous readings, it is unsuitable for reducing crosstalk. \cite{2016_artigo_1}

