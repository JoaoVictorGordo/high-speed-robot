\chapter{Embasamento Teórico} %% ou {Método} ???
\section{Arquitetura Reativa}
\epigraph{ 
	  \textit{``Representações explícitas e modelos atrapalham. No fim das contas, a melhor representação do mundo é ele mesmo.''} 
	 }
	  { Brooks, R.A \cite{brooks} - tradução livre -} 
	  
%   \begin{flushright}
% 	  \textit{``Representações explícitas e modelos atrapalham.\\
% 	  No fim das contas, a melhor representação \\
% 	  do mundo é ele mesmo.", Brooks, R.A. 
% 	  \cite{brooks} \footnote{tradução livre}}
%   \end{flushright}
A ideia básica que fundamenta esta arquitetura é de que tudo aquilo conquistado até hoje pelos seres vivos, incluindo a própria racionalidade e 
inteligência humana, se deu exclusivamente graças às nossas faculdades de percepção, mobilidade e manutenção da própria sobrevivência \cite{brooks}.
Denominemos agentes primitivos aqueles indivíduos cujos comportamentos se reduziam exclusivamente aos seus instintos de auto-preservação.
Sendo que, de acordo com \cite{murphy} numa tradução livre, ``comportamentos são mapeamentos diretos das informações sensoriais recebidas em um 
padrão de ações motoras, utilizadas a fim de se cumprir uma tarefa''

Sob esta ótica, temos que estes agentes primitivos desenvolveram novos comportamentos ao longo de seu processo evolutivo, que nada mais é do 
que uma consequência das faculdades deste indivíduo de se mover e perceber o mundo, de forma que estes novos comportamentos obtidos se relacionam com 
os demais de três maneiras possíveis.
A primeira consistiria na reutilzação de outros comportamentos que lhe são ancestrais, o que seria um refinamento do seu proceder: a criação de uma 
nova camada comportamental que envolve as ancestrais, análogo à constituição de uma pérola.
O segundo meio seria negando comportamentos antecessores, o que constituiria uma mudança de postura. 
Enquanto que a última maneira seria a simples criação de uma nova conduta paralela, o que pode ser visto como um comportamento mais avançado que se 
empilha sobre os demais ao invés de envolvê-los ou negá-los\cite{murphy}.

A noção de que toda ação é fruto de um comportamento alicerça o paradigma reativo.
Temos portanto que a ação está intimamente ligada à percepção através do comportamento.
Nesta perspectiva, quando afastamos a mão ao tocarmos uma panela quente, podemos entender que a nossa ação foi consequência de um comportamento que 
nos impele a distanciarmo-nos de objetos quentes, detectados pelos nossos sensores de calor da pele.

Como consequência desta prerrogativa, a etapa de planejamento existente na tríade 'percepção, planejamento, ação', que sumariza 
o ciclo de tarefas realizadas por um sistema sob o paradigma hierárquico \cite{murphy,roseli}, pode ser descartada por não ser necessária.
Em suma, os comportamentos se dão de acordo com o que está acontecendo no ambiente externo, informação obtida através da leitura dos sensores que 
compõem a percepção do robô.
Nenhuma modelagem ou representação deste ambiente externo é feita, apenas medições locais e orientadas a comportamentos são adotadas.

\section{Arquitetura MOSA}

Arquitetura que propõe dividir o sistema aéreos de navegação autônoma em dois módulos: aeronave e MOSA \cite{mosa_proposal}.

O primeiro constitui a porção crítica do sistema embarcado, i.e. segmento cuja falha pode resultar em ao menos um dos seguintes desastres: morte 
ou lesão de pessoas; destruição ou danos a propriedades, patrimônios ou equipamentos; danos ambientais \cite{safety}.
Veículos Aéreos Não Tripulados, VANTs, apresentam a tolerância de um erro grave a cada $10^5$ ou $10^9$ horas de voo \cite{hard}, o que as 
caracteriza 
como sistemas computacionais de tempo real do tipo \textit{hard}. 
Maiores esclarecimentos acerca destes jargões podem ser encontrados no apêndice.

O segundo corresponde à parte não crítica à segurança, encarregada do controle da navegação e, por conseguinte, da determinação da maior parte dos 
parâmetros de voo. É caracterizado como um conjunto de sensores inteligentes capazes de cumprir uma missão específica, ou seja, 
existe uma relação biunívoca entre missão e MOSA, dado que ele consiste no melhor arranjo de sensores para o cenário em questão. Neste contexto, a 
aeronave é vista unicamente como o meio de transporte dos sensores, enquanto que o módulo MOSA constituiria o \textquoteleft cérebro\textquoteright{}  
da plataforma, responsável pelo cumprimento da missão e por guiar a aeronave até a sua realização.

%  \begin{figure}[h]
%   \includegraphics[scale=0.5]{./Resources/MOSA.png}
%   \caption{MOSA} \label{MOSA}
%  \end{figure}

No entanto, como a aeronave é o elemento responsável pela garantia da segurança, cabe a ela acatar ou não os comandos do MOSA. E pode, inclusive, 
optar por readaptar a missão em tempo de voo para se ajustar ao cenário, o que inclui a seleção dos sensores que melhor se encaixam na dada 
conjuntura.

Isso se dá através de uma matriz de reconfiguração dinamicamente adaptável denominada \textit{Knowledge Based Framework}, seu papel é comparável à 
expertise de um piloto.
Ou seja, um elemento inteligente capaz de escolher o melhor serviço  a ser executado com base em regras e critérios de seleção como resposta em tempo 
real, segurança, performance.


% diferentes missões, definidas pelo mosa e diferentes sensores, podem ser integrados, possibilitando a escolha do melhor arranjo de sensores que se 
% ajuste ao cenário de utilização do sistema. este é o mecanismo básico do mosa, fazendo com que a missão possa ser adaptativa. durante uma missão, 
% com 
% base em uma matriz de reconfiguração o vant pode se adaptar dinamicamente às características da missão, escolhendo os sensores que melhor se 
% encaixam 
% dependendo da situação. além do hardware, um sistema mosa deve contemplar também o software capaz de realizar uma missão, comunicar-se com todos os 
% sensores que o compõe, enviar e receber dados para a aeronave .
% 
% o sistema mosa deve ser capaz de determinar se a missão prevista pode ou não ser realizada.
% 
% the aircraft can, for safety reasons, not follow the flight sensors
% commands, eventually terminating the flight
% 
% flight sensors can provide data for mission controllers but aircraft
% % % % % % % % controllers must not use data provided by mission sensors
